\documentclass[oneside]{book}
\usepackage{hyperref}
\hypersetup{
  colorlinks   = true,    % Colours links instead of ugly boxes
  urlcolor     = blue,    % Colour for external hyperlinks
  linkcolor    = blue,    % Colour of internal links
  citecolor    = red      % Colour of citations
}
\usepackage[table,xcdraw]{xcolor}
\usepackage{multicol}
\setcounter{chapter}{-1}
\setcounter{secnumdepth}{0}
\makeatletter
\renewcommand{\@dotsep}{10000} 
\makeatother
\begin{document}


\title{ {\Huge THIZ} \\ \vspace{2 mm} {\large THIZ is Heavily Inspired by ZeFRS}}
\date{}
\author{Big Tex}
\maketitle

\pagenumbering{gobble}
\paragraph{As the subtitle suggests, this work is inspired by ZeFRS, a role-playing system that is adapted from an original body of rules written by Dave "Zeb" Cook. This rulebook has been written and compiled by Big Tex} 

\paragraph{The author would like to extend a great thanks to Zeb Cook, as well as the author of ZeFRS, Mark Krawec. Additional thanks are due to everyone else involved in the ZeFRS project: \\}

\begin{minipage}[t]{0.5\textwidth}
Jason Vey \\
Drake2000 \\ 
E.T. Smith \\
The Evil DM \\
Max \\
M\"{u}nch \\
Artikid \\
Marius \\
hive\_mind \\
Harmast
\end{minipage}
\begin{minipage}[t]{0.5\textwidth}
Insect King \\
blu\_sponge \\
DMAndrew \\
MountZionRyan \\
\"{A}rk\"{a}s \\
hatheg-kla \\
Vagabond \\
Christopher V. Brady \\
Spinachcat
\end{minipage}

\paragraph{If you're interested in reading more about ZeFRS, please visit the following links: \\
\url{http://www.midcoast.com/~ricekrwc/zefrs/} The ZeFRS project home\\
\url{http://zefrs.proboards.com/} The ZeFRS Project Forums \\
\url{http://forum.rpg.net/showthread.php?327143} ZeFRS Project thread \\
\url{http://forum.rpg.net/showthread.php?206835} Where it all started }


\paragraph{The author was not personally involved in the ZeFRS project and lays no claims of ownership to said body of work. The same can be said for the original work by Zeb Cook. Parts of this work, most especially the skills/conflict resolution system are taken directly from ZeFRS itself. The author feels many other areas of the ZeFRS project have been modified enough to warrant a distinct publication. \\ \\
This work may be freely distributed as long as it remains complete and intact, which includes this credits page. The text of this work is designated as "copyleft," which means it can be used, modified, and reproduced at no cost. It would be appreciated, however, if credit would be given to the original author, Dave "Zeb" Cook, and those involved in the ZeFRS project.}

\tableofcontents

\chapter{Introduction}

While THIZ was inspired by a set of rules intended for a Swords n' Sorcery role-playing experience, THIZ strives to be a system that facilitates any and all settings and genres. The fluff is left to the Referee, while THIZ provides a full-fledged set of rules that strives to leave the players feeling powerful without bogging down gameplay. There are no races, very few restrictions with regard to magic, and a potentially limitless number of skills for player characters to master. 

\section*{The Obligatory Explanation of Role-playing Games}

Many people reading this rulebook likely already know what a role-playing game is. For those who are net yet part of this fortunate group, a role-playing game is a shared narrative, in which all participants have an opportunity (and perhaps an obligation) to tell the story as it unfolds. Each character has a role not only in the party but in 
the world as a whole. Each player's actions, reactions to events, and interactions with non-player characters (NPCs for the uninitiated) coalesce to form a living world, a compelling backdrop for a grand adventure (or series of adventures). \\
THIZ fits into this picture by adding an element of order and providing a means to not only define each character but also their ability to act upon the world around them. This rulebook provide a means for the referee (sometimes referred to as the Game Master) to arbitrate contests or conflicts between different characters. The Referee is the player who tells most of the story, providing the backdrop and setting the stage for the other players. Each player is in control of one or more characters, who act as their avatars in the game. 

\section*{Requisite Supplies}

You'll need a referee and at least one other player. A pencil and paper will be useful, as will the character sheet included at the end of this book. Additionally, you will need to be able to randomly generate a number between 1 and 100. This is normally achieved using a pair of ten-sided dice, often referred to as a d100, or percentile dice.  This set of dice typically has one die with numbers from 10 to 100 and one die that ranges from 0 to 9, but alternately you could use two dice with a similar range of numbers, as long as they are of a different color, and you consistently designate one color to serve as one part of the percentage. Finally, you will need a copy of the resolution table that is included in this book. This is used to resolve any conflicts or contests between characters. Once a d100 has been rolled and a random number generated, this number will be used in conjunction with the resolution table to determine the level of success of a character's actions. Enough talk. Let's get to the game!

\chapter{Character Creation}

\section{What Exactly is a \emph{Character?}}
As you may have guessed, this chapter intends to outline the process of creating a character. In order to create a character, it is necessary to understand what exactly defines a character in the world of THIZ. Characters in THIZ have a name and a homeland, both of which are mostly fluff; and a set of skills, which defines the limits of what they can do.

\section{Skills?}
Skills in THIZ are probably the most complicated aspect of the game. This complication arises from the many subgroups they are divided into. Skills are first divided into three broad categories: \textbf{Hard Skills}, which are the skills that directly affect a character's combat prowess, and \textbf{Soft Skills}, which are the skills that help a character perform useful actions outside of combat, and \textbf{Magic Skills}, which are the skills that a character utilizes to manifest (hopefully) useful magical effects. Hard Skills are then divided into two categories: \textbf{Mettle}, which are the skills that a character uses to avoid or endure attacks and \textbf{Efficacy}, which are the skills that help a character perform physical attacks. Soft Skills are divided into three categories as well: \textbf{Expertise}, which are the skills that govern a character's intellectual ability and critical thinking skills; \textbf{Awareness}, which are the skills that a character uses to perceive the world around them; and \textbf{Body}, which are skills that delineate a character's physical abilities and athleticism.

 \newpage
\begin{multicols}{2}
\section{Creating a Character}
Usually a character begins with a concept. Would you like to play as a taciturn warrior who charges into the battlefield wielding two spiked shields, bashing their enemies until either he or they stop moving? A fast-talking necromancer with a weakness for drink and a fear of the sun? An unusually short archer with a seething hatred for the fair-skinned folk of \emph{Dwyrain}?  Coming up with a basic concept can sometimes be the most difficult part of character creation. Sometimes is may be easier to choose which skills you'd like your next character to have and come up with the concept later, but establishing a back-story for your character is an integral step in the character creation process. It may not be very fun for the other players if they have to adventure with "Guard \#12." Record this back-story in the "The Story Begins" section of your character sheet. The following three sections will help you to flesh out this section.

\section{Your Character's Name}
Coming up with a name for your character is also an important step in creating a character. As your character's fame (or infamy) grows, their name will often precede them, spreading across the land with tales of their exploits and derring-do. As such, it would be a good idea to ensure that the name is enjoyable to you and fits into the setting of your campaign. Often, a character with a single name and an eponym is sufficient. If you need inspiration, as either a player or referee, \emph{Gary Gygax's Extraordinary Book of Names} is a great resource.

\section{Homeland}
Your character's homeland is the place of their birth. This will affect the way they look, their customs, and their native tongue. Their homeland may also restrict the skills they start the game with. A character born in the capital city of a very industrialized nation may not have any need to learn the arts of Tracking, for example. 

The Homeland of your character is determined as much by the physical location of their birth as by the heritage(s) of their parents. 

Your character is automatically proficient in the common language of their homeland, as well as any one language which their parents use to communicate. They can speak both of these languages, but they do not necessarily have to be proficient in reading or writing either of them. At the referee's discretion, one of these languages may be a sort of \emph{lingua franca} used by all of the neighboring nations. 

\section{Parents}
Your character should have at least one parent or legal guardian, and while this book may refer to parents in the plural, it is perfectly acceptable for a character to have a single parent. Even it's simply the matron (or patron) of a rural orphanage, at some point your character had someone else to care for or mentor them. Decide on a name for your character's parents, as well as their careers. Careers can be based on any of the skills listed Chapter 2, or can be some profession of your own creation. Keep in mind that, at character creation, your character will have to invest at least one skill point in a skill that represents one of their parent's careers. This is meant to represent the career path that your character would have chosen had they not heeded the call of adventure. As such, if their back-story dictates that they have some sort of mentor that isn't one of their parents, a skill may be chosen to represent that person's career.

Now that you've finished choosing a name, homeland, and some details regarding your character's parents, you can fill in the section on your character sheet that reads "The Story Begins."

\section{Your Character's Skills}
Skills are the real meat of your character. They define the sum of your character's experiences and knowledge. At character creation, you have 20 skills points to allocate to your character. Each point may be exchanged at a 1:1 ratio to increase your rating in that skill. Each character begins with all skills at a rating of zero. You may allocate points to any skill in the table below as well as any skills that your referee may have added to the skill pool, with the following restrictions:
	\begin{itemize}
		\item{ \small You must allocate at least five points each on both \textbf{Hard Skills} and \textbf{Soft Skills}}
		\item{ \small You may allocate no more than 5 points on any one skill}
		\item{ \small You must allocate at least one point to skill representing a parent or mentor's profession}
		\item{ \small You must not allocate points to any skills which your referee has deemed incompatible with the setting}
\end{itemize}	

There is an additional restriction on choosing Magical Skills: for each unique magical skill your character begins play with, you must choose a \textbf{Weakness} for that character to also possess. For magical skills denoted in the chart with a $\alpha$ symbol, you must choose a Spiritual Weakness. For magical skills marked in the chart with a $\beta$ symbol, you must choose a Mental Weakness. More information on weaknesses can be found in Chapter 3.

When choosing skills, you should also keep in mind that for your character to learn a new skill after character creation, you must spend 5 skill points to get a rating of 1 in that skill. To learn more about each of these skills, refer to Chapter 2, where each is described in detail.

To further customize your character, you may choose to give them \textbf{Weaknesses} or \textbf{Quirks}. These are described in detail in Chapter 3. Choosing a Weakness (beyond those you may have incurred for learning a magical skill) grants you 5 more skill points to allocate to your character. Alternatively, adding a Quirk to your character costs 5 skill points. As you may have guessed, Weaknesses negatively affect your character: they may restrict what other skills they may learn or impose penalties under certain conditions. Quirks, on the other hand, may provide small perks or skill boosts when making certain resolution checks.
\end{multicols}
\newpage
\section{Table of Skills}


\begin{table}[h]
\begin{tabular}{|c|c|}
\hline
\rowcolor[HTML]{000000} 
{\color[HTML]{FFFFFF} \textbf{Hard Skills}} & {\color[HTML]{FFFFFF} Soft Skills}         \\ \hline
\rowcolor[HTML]{656565} 
{\color[HTML]{FFFFFF} \textbf{Mettle}}      & {\color[HTML]{FFFFFF} \textbf{Expertise}}  \\ 
\rowcolor[HTML]{9B9B9B} 
\multicolumn{1}{|l|}{\cellcolor[HTML]{9B9B9B}{\color[HTML]{000000} \textbf{\begin{tabular}[c]{@{}l@{}}\\ Reflexes {[}P{]}\\ Speed {[}P{]}\\ Strength {[}P{]}\\ Toughness {[}P{]}\\ Willpower {[}M{]}\end{tabular}}}} & \multicolumn{1}{l|}{\cellcolor[HTML]{9B9B9B}\textbf{\begin{tabular}[c]{@{}l@{}}Appraise {[}M{]}\\ Artisan (Choose one) {[}M{]} \\ Forgery {[}M{]} \\ Knowledge (Choose one) {[}M{]}\\ Lock-picking {[}M{]}\\ Persuasion {[}M{]}\\ Profession (Choose one) {[}M{]}\\Trade (Choose one) {[}P{]}\\ Trapping {[}M{]}\end{tabular}}}   \\ \hline
\rowcolor[HTML]{656565} 
{\color[HTML]{FFFFFF} \textbf{Efficacy}}    & {\color[HTML]{FFFFFF} \textbf{Body}}       \\ 
\rowcolor[HTML]{9B9B9B} 
\multicolumn{1}{|l|}{\cellcolor[HTML]{9B9B9B}\textbf{\begin{tabular}[c]{@{}l@{}}Dual-Wield [P]\\ Grappling {[}P{]}\\ Tactics {[}M{]} \\ Unarmed Combat {[}P{]}\\ Weapon Skill (Choose one) {[}P{]}\end{tabular}}}                                                & \multicolumn{1}{l|}{\cellcolor[HTML]{9B9B9B}\textbf{\begin{tabular}[c]{@{}l@{}}Acrobatics {[}P{]}\\ Climbing{[}P{]}\\  Sleight of Hand {[}P{]}\\ Stealth {[}P{]}\\ Swimming{[}P{]}\\ Throwing {[}P{]}\end{tabular}}}                                                      \\ \hline
\rowcolor[HTML]{333333} 
{\color[HTML]{FFFFFF} \textbf{Magic Skills}}  & \cellcolor[HTML]{656565}{\color[HTML]{FFFFFF} \textbf{Awareness}}    \\
\rowcolor[HTML]{9B9B9B} 
\multicolumn{1}{|l|}{\cellcolor[HTML]{9B9B9B}\textbf{\begin{tabular}[c]{@{}l@{}}Alchemy\\ Artifice\\ Druidism\\ Insight\\ Mentalism\\ Necromancy\\ Psychokinesis\\ Reiki\\ Shamanism\\ Summoning\end{tabular}}}                          & \multicolumn{1}{l|}{\cellcolor[HTML]{9B9B9B}\textbf{\begin{tabular}[c]{@{}l@{}}Animal Handling {[}M{]}\\ Fast Talk {[}M{]}\\ First Aid {[}M{]}\\ Magic Sense {[}P{]}\\ Navigation {[}M{]}\\ Perception {[}P{]}\\ Pilot (Choose One) {[}M{]}\\ Pocket-Picking {[}P{]}\\ Scrutiny {[}M{]}\\ Seduce {[}M{]}\\ Weather Sense {[}M{]}\end{tabular}}} \\ \hline
\end{tabular}
\end{table}

\section{General Skill Ratings}
In addition to the actual skill ratings your character possesses, they also have three general skill ratings: \textbf{Physical}, \textbf{Mental}, and \textbf{Magical}. These scores are calculated by taking your character's rating in the appropriate skills, summing these ratings, and dividing the result by ten, rounding down. The skills in the above chart marked with a [P] contribute to your character's Physical rating, whereas those marked with a [M] contribute to their Mental rating. All Magic Skills contribute to their Magical rating. 

\onecolumn

\chapter{Skills}
This chapter discusses the various skills available to your character in greater detail. Those marked with an [M] or [P] contribute to a character's Magical or Physical general skill rating. All Magic Skills contribute to their Magical general rating.

With the exception of Magic Skills, which cannot be used unless a character has learned them, a character makes resolution checks using the higher between their rating in the appropriate skill or that skill's governing general rating (Physical or Mental). For example, a character with no rating in Unarmed Combat and Physical rating of 4 makes resolution checks for Unarmed Combat against a base rating of 4 (situational modifiers may affect this rating as any other).

\begin{multicols}{2}

\section{Hard Skills}
\subsection{Mettle}
\subsubsection{Reflexes [P]}
This is the skill that a character uses to quickly avoid traps or react to the events that occur around them, most usually in battle. This skill also increases a character's initiative score. Traps are discussed further in the chapter entitled \textbf{Hazards}.

If a character delays their action in combat and wishes to take an action \emph{before} the event that they were waiting for, they must make a check against their Reflexes rating at the moment they wish to act. On a Yellow success or better, this character may make that action. Otherwise, they take their action at the end of the initiative pass.

\subsubsection{Speed [P]}
This skill determines a character's movement speed, and is also used by characters to avoid enemy attacks. Movement speed is discussed in the chapter entitled \textbf{Movement}.

\subsubsection{Strength [P]}
This skill determines how heavy of an object a character can lift, as well as how heavy of a load they can carry on their backs as they travel. A character can overhead lift 100 pounds plus an additional 10 pounds for every point of Strength. A character can lift double this amount off the ground, moving as if they had a Speed rating of 0. They can move in this way for the same amount of time they can Sprint (see the chapter on Movement).

\subsubsection{Toughness [P]}
This is the skill that characters use to shrug off the effects of attacks they receive in combat, as well as the effects of poison or other hazards. The chapters entitled \textbf{Combat} and \textbf{Dangers} explain the use of this skill in more detail. 

\subsubsection{Willpower [M]}
Willpower governs a character's ability to resist their internal compulsions and desires, as well as those that others may attempt to impose upon them. Characters usually use Willpower to negate or lessen the effects of their Weaknesses. Willpower may also help a character ignore certain magical attacks.
\\
\\
\subsection{Efficacy}
\subsubsection{Dual-Wield [P]}
This skill allows a character to use a pair of any two one-handed weapons together in combat, usually attacking with both of them at once (see the chapter entitled Combat). A new instance of this skill must be learned and allocated its own points for each combination of weapons the character wishes to use. For example, a character can put 3 points into Dual-Wielding [Rapier, Pistol] and 5 points into Dual-Wielding [Falchion, Axe] to get a rating of 3 and 5 in those skills respectively. When taking this skill, the character must list the two weapons such that the weapon they are wielding in their main hand is listed first. A character suffers a penalty for wielding weapons in the wrong hand.

\subsubsection{Grappling [P]}
This skill is used by characters when they grapple with or attempt to pin an opponent. Please see the chapter on Combat for more information on grappling.

\subsubsection{Tactics [M]}
This skill is for characters who wish to direct their ally's actions on the battlefield, informing them how best to defeat their enemies in any situation. Whenever three or more allies are in combat with a specific enemy, a character may use their combat turn to make a check against their Tactics skill. Based on their level of success, each ally in combat with that specific enemy at the time of this check receive a bonus on their next turn on any resolution checks involving combat with that enemy. This bonus is lost at the end of the next turn, and bonuses from multiple Tactics checks do not stack. The magnitude of this bonus ranges from 1 to 4, for Green through Black successes respectively. 

\subsubsection{Unarmed Combat [P]}
This skill is used by characters when they are attacking an enemy using a part of their body, eschewing the use of any weapons. Attacking in this manner normally cannot damage armored Adversaries. If a character has a Strength rating that is triple the armor rating of the targeted area, however, they can deal damage to an armored Adversary. See the chapter on Combat for more information regarding Adversaries. 

\subsubsection{Weapon Skill [P]}
This skill is used by a character when they wish to attack or perform maneuvers with a weapon. A new instance of this skill should be chosen for every weapon the character is proficient in, each with its own rating determined by points that have been allocated to it specifically.

\section{Soft Skills}
\subsection{Expertise}
\subsubsection{Appraise [M]}
This skill is used to appraise the value of trade goods and the works of artisans and craftsmen. A character that has a particular Artisan or Profession skill can check against their skill rating to appraise works (or materials) that fall under that skill's domain, but any character that does not have the appropriate profession or craft must use the Appraise skill to judge the value of an item. A character that has both Appraise and an appropriate Artisan or Profession skill may check against the higher of the two skill ratings, with a +2 bonus, in order to appraise something. 

The level of success of this check determines the width of the price range a character may determine for a particular object. A Green success would result in a range that is $\pm$ 20\% of the object's value, Yellow would result in $\pm$ 10\%, Red $\pm$ 5\%, and a Black success would give the object's exact value.
\subsubsection{Artisan [M]}
This skill is intended for character's that intend to pursue a more artistic profession. When learning this profession, a character must choose an artistic calling, such as Goldsmithing (making jewelry from gems and precious metals), Silversmithing (making silverware, flatware, hollowware, and other such items from gems and precious metals), Painting, Sculpture etc... A character may have multiple instances of the Artisan skill for each artistic pursuit they wish to follow. Of course, they must allocate points to each of these skills separately in order to increase their ratings.

See the chapter entitled \textbf{Character Advancement and Time Management} in order to see how a character may earn money by pursuing a career. 
\subsubsection{Forgery [M]}
This skill governs a character's ability to create forged documents in any language they can read and write. In order for a character to forge a specific document, they must have a copy of that document to reference while creating the document. For each full 4-hour period a character has to work on their forgery, with a minimum of 1 (8 hours or less) and up to a maximum of 6 (24 hours or more), they may make a resolution check against their Forgery rating. For each check, the player should write down a 1, 2, 3, or 4 for a Green, Yellow, Red, or Black success, respectively. After all checks have been made, the player should sum the numbers they have written down. This sum is the forged document's \textbf{Forgery Score}. A Forgery Score of 0 means the character has failed; otherwise, they have successfully created a forgery. The Forgery Score is important if another character attempts to discover the ruse.

A character wishing to create a forgery of the product of a artisan or tradesman, they must have the appropriate Artisan or Trade skill. The process of creating a forgery with one of those skills is identical to that of creating a forgery using this skill.
\subsubsection{Knowledge [M]}
This skill represents the character's breadth of knowledge and experience with a subject. When learning this skill, pick any broad topic, such as a particular nation's (or world) history, battlefield tactics, herbalism, literature, etc... A character may have multiple instances of the Knowledge skill, representing their knowledge of different domains. As usual, they must allocate skill points to these skills separately in order to increase their ratings.

Whenever a character would search their breadth of knowledge for a particular fact, they may make a resolution check against their Knowledge rating for the topic under which that fact falls. The level of success required to remember a fact is determined by how specialized this particular fact is. For example, if a character wished to remember the minutiae of a particular year in a relatively unknown Classical poet's life, they would be required to make a Black success. At any other level of success, while they may remember something about that author's life or works from that year, they won't recall anything at the level of detail they may have hoped for.

\subsubsection{Lock-picking [M]}
This skill is used by characters in order to open locked doors and containers. A character should make a check against their Lock-picking rating for each tumbler in the lock. Any level of success means that the character has successfully set the pin in that tumbler. A Green success means that this process took 1 minute, Yellow 45 seconds, Red 30 seconds, and Black 10 seconds (or lower, at the referee's discretion). A single failure simply means that the character must try to set that pin again. Two failures in a row resets all the pins that had been set, forcing the character to start picking the lock over again. If the setting supports this, three failures in a row triggers an external alarm of some sort, possibly alerting nearby characters.
\subsubsection{Persuasion [M]}
This skill is used in order to convince a character (usually an NPC) to take an action that they normally wouldn't be inclined to, or to permanently change their viewpoint on a particular matter. A character usually refuses to take an action because of material, moral, or mortal concerns. 

Convincing a character to take an action that would cause them a small material loss will take a Green success, whereas an action that would incur a greater loss of wealth or other possessions would take a Yellow or Red success, depending on the magnitude of the cost. A Black success would only be required to convince a character to take an action that would cause them to lose the entirety of their wealth. 

Convincing a character to take an action that is against the tenets of their culture or religion might require anywhere from a Green to Black success, depending on the fervency of their beliefs, and the level of taboo that particular action holds. Convincing a zealot to make even a small slight against their god might take a Black success, while convincing a casual believer to condemn themselves to damnation might take a Red success. Cultural norms are usually easier to get a character to ignore, and so a Black success would likely only be required for something as grave to them as murder is to most cultures.

Convincing a character to take an action that would put themselves or a family member in mortal danger might take a Red or Black success, depending on how assured their destruction seems. Convincing them to put a stranger in mortal danger might take a Yellow or Red success, also depending on how assured their destruction seems. 
If a player roleplays their speech to the character particularly well, the referee may decide to lower the required level of success, at their discretion. Also at the referee's discretion, this skill may be checked against in order to move a crowd with a speech.

\subsubsection{Profession [M]}
This skill is for characters who wish to pursue a career in a more "learned" field, such as Education (educating others), Mathematics, Engineering, Architecture, Surgery, etc... A character may have multiple instances of the Profession skill for each intellectual pursuit they wish to follow. Of course, they must allocate points to each of these skills separately in order to increase their ratings. 

As Surgery falls under this skill, please see the chapter entitled Combat in order to see its possible uses. 

See the chapter entitled \textbf{Character Advancement and Time Management} in order to see how a character may earn money by pursuing a career. 

\subsubsection{Trade [P]}
This skill represents a character's knowledge of and ability in a tradeskill, such as Woodworking, Carpentry, Plumbing (if setting-appropriate), Blacksmithing, etc... Professions that fall under this skill's purview are those that are typically considered "blue-collar."  A character may have multiple instances of the Trade skill for each trade they wish to pursue. Of course, they must allocate points to each of these skills separately in order to increase their ratings. 

See the chapter entitled "Character Advancement and Time Management" in order to see how a character may earn money by pursuing a career. 

\subsubsection{Trapping}
This skill is used by characters who wish to construct and lay traps. This skill also represents a character's knowledge and intuition of where best to lay traps. This skill can be used to construct or detect outdoor traps for creatures up to human size. Spike pits, snares, and net traps fall under the umbrella of this skill. Indoor or mechanical traps are not governed by this skill, which would more likely fall under Engineering (a subset of Profession).

If the trap is intended to catch unintelligent prey, the referee makes a check against the character's Trapping rating, and they catch some sort of prey on any degree of success. The quality and/or quantity of prey increases as the level of success gets greater. 
\\
\\
\subsection{Body}
\subsubsection{Acrobatics [P]}
This is a skill that is used by characters that wish to jump across gaps or over obstacles. This skill would also be used if a character wished to swing from a rope or perform any other such stunts. Please see the chapter entitled \textbf{Movement} for more information regarding Acrobatics.

A character wishing to long jump may make a resolution check against their Acrobatics skill. For a standing long jump, the character can jump 8, 9, 10, or 11 feet, for Green through Black successes respectively.
If the character can run at least their jogging distance before attempting the long jump, they can jump 20, 22, 24, or 26 feet, for Green through Black successes respectively.

A character attempting a vertical jump from standing can reach heights of 32, 34, 36, or 38 inches, for Green through Black successes. A character that can run at least their jogging distance before attempting a high jump can reach a height of 60, 68, 76, or 84 inches for Green through Black successes.
\subsubsection{Climbing}
This skill is used by a character when they are scaling towers or large walls. Any character can climb a short distance, but for a character to successfully climb for an extended period of time over a large distance, they need some knowledge and experience. The chapter on Movement has more information regarding climbing. 

\subsubsection{Sleight of Hand}
This skill governs the ability of a character to successfully make an object appear or disappear from within a pocket. The difficulty required for a character to hide or reveal a particular object successfully depends on the size of the object, according to the referee's discretion. Something discrete like a coin may require only a Green Success, whereas something the size of a human head or larger would require a Black success. Any level of success below the threshold established by the referee means that the item was spotted, at least in passing, by a nearby character that was actively observing the character attempting to use Sleight of Hand. A failure means that the character has dropped the item. 

A character may also attempt to cheat at a game of chance with their Sleight of Hand skill. Any level of success means that their cheating has so far gone undetected. 
Whenever a character makes a Sleight of Hand check, subtract the highest Perception rating among all the characters who are actively observing the character making the check. 

\subsubsection{Stealth}
This skill allows a character to attempt to hide from others or move unseen. A character wishing to use stealth to move must make a check against their stealth rating minus the highest Perception rating among all the characters actively searching for them (but not currently looking at them). If they succeed, they may move up to 1, 2, 3, or 4 times their walking speed before they must make a stealth check again. On a failure, they have been spotted. A character automatically fails a stealth attempt if they are being directly observed by a character that is searching for them or would immediately reveal their location to someone searching for them. If they are currently being observed, they must first move out of sight before attempting stealth. 

If a character is attempting to hide from someone pursuing them, that person doesn't currently have them in sight, they make make a check against their stealth rating to enter a hiding spot, hide an object they are holding, or hide a character that is adjacent to them into a hiding spot of sufficient size that is nearby.

 For each full one-minute period the character has to attempt to hide, they may make a resolution check against their Stealth skill. They may make a minimum of 1 check (if they have less than two minutes to hide) and a maximum of 6 (they have 6 minutes or more to hide). As the player makes each of these resolution checks, they should write down a 1, 2, 3, or 4 if they achieved a Green, Yellow, Red, or Black success, respectively. After all checks are made, the player should sum these numbers. This sum is the character's \textbf{Hide Score}. A Hide Score greater than 0 means the character has successfully hidden whatever it is they were attempting to hide. A Hide Score of 0 means the character has failed. The Hide Score is important if another character attempts to find the hidden character or object.

See the description of the skills \textbf{Perception} and \textbf{Scrutiny} to learn more about spotting a character that is hidden. 

\subsubsection{Swimming}
Without this skill, a character can only tread water or doggy paddle in the water. In order to move any appreciable distance in the water, or to stay afloat for an extended amount of time, a character must have a rating above 0 in the Swimming skill. Please see the chapter on Movement for more information.

\subsubsection{Throwing}
This skill is used whenever a character attempts to throw a non-weapon object across any distance. A character can make a check against this skill to throw a 16-pound object 64, 68, 72, or 76 inches on Green through Black successes, respectively. This distance is inversely correlated to the object's weight. If the object were to double in weight, the possible distance it could be thrown would be halved. 

This skill is meant to represent the maximum possible distance a character could throw an object, with ample time to prepare themselves for the throw. If this skill were to be used for combat in an improvised manner, an object of any size can only be thrown a quarter the distance it could normally be thrown.

\subsection{Awareness}
\subsubsection{Animal Handling [M]}
This skill is used by a character to train and command animals. This skill is most often used to control a mount in abnormal situations. To get a mounted creature to move towards a foe or obstacle that it is terrified of (most usually an exotic or supernatural beast), the character makes a check against their Animal Handling rating. On any level of success, the creature will move towards the foe, but at a maximum rate of 1/4, 1/2, 1, or twice its movement speed.

A character with this skill can rear an animal from birth, training it for any sort of purpose they wish (as long as the referee feels that purpose is not too complex for its intelligence). This training takes a period of one year for the creature to obey without fault or hesitation. This training can occur as the character travels on adventures, as long as the creature accompanies the character. 

\subsubsection{Fast Talk [M]}
This skill is perfect for characters who wish to master the art of the grift. This skill is used by a character to convince another character of a fact that they would not otherwise believe to be true. A character who has been duped by a successful use of the Fast Talk skill will realize the deception after 10+1d10 minutes. The level of success require to successfully swindle another character depends on the unbelievability of the lie, at the referee's discretion. The difficulty can also be lowered one step if the character has role-played their attempt particularly well, also at the referee's discretion. 

\subsubsection{First Aid [M]}
This skill is used by characters to quickly heal minor wounds that they or other characters have incurred during their adventures. Please see the chapter on Combat to read more about recovering from damage.

\subsubsection{Magic Sense [P]}
This skill is used by characters who, by some means or another (most likely due to some physical mutation or knowledge of telltale physical clues), are able to identify magical effects and magic users themselves. A character with points in this skill will be able to detect magical effects or objects at a distance of 10 feet * their rating in this skill. Additionally, a character may make a check against this skill in order to determine if another character has any magical abilities. A Black success can detect a character with the tiniest vestige of magic (a rating of 1 in a single magical skill), whereas a Yellow success can detect a character with a rating of 5 or higher in a single magical skill, Red can detect a character with a rating of 9 or higher, and Green  can detect a character with a rating of 15 or higher in a single magical skill or a general Magical rating of 1 or higher. A character using this skill simply knows that something is magical, they have no knowledge (through of this skill) of the nature or strength of that magic. 

\subsubsection{Navigation}
This skill allows characters to find their bearings while traveling. The level of success required to successfully reach a destination is determined by the number of known landmarks along the way and how cloudy the sky is (whether the sun is visible if traveling by day or an appropriate number of stars is visible if traveling by night). This skill can also be used to determine which direction the character is currently facing at night (a trivial task during the day if the sun is visible, impossible by use of this skill otherwise). To gain their bearings at night, a character must simply make a check against their Navigation rating at any level of success. Otherwise, on a failure, the character mistakenly believes they are going the direction they desire. The referee randomly determines which direction they are actually traveling.

\subsubsection{Perception [P]}
This skill is used whenever a character needs to quickly spot any feature in their surroundings, such as an approaching character at the top of a hill, an ambusher's elbow poking out of a nearby bush, or the telltale signs of a trap. The level of success required to spot an approaching character or party depends on the size of the party and the distance and environmental conditions between them. It is up to the referee's discretion if a character's check against their Perception check is sufficiently successful to spot the approaching party immediately. 

A character may also attempt to use Perception to immediately spot a character or object that they believe is hiding in the room. A character wishing to do so must make a check against their Perception rating. The player should write down a 1, 2, 3, or 4, for a Green, Yellow, Red, or Black success, respectively. If this number exceeds the Hide Score (see the section on the Stealth score) of the hidden object or character, they have found it. Otherwise, they must resort to use of the Scrutiny skill.

A character can also check against their Perception rating to survey their surroundings for interesting information. The referee should describe their surroundings in greater detail as their level of success increases. 


\subsubsection{Pilot [M]}
This skill governs a character's ability to pilot a craft or vehicle. Upon learning this skill, the character should name a specific class of vehicle, such as schooner, brigandine, or perhaps helicopter or tank in a more technologically advanced setting. For vehicles that the referee decides are sufficiently complicated, a character without the appropriate Pilot skill cannot pilot them whatsoever.

A character may have multiple instances of the Pilot skill for each type of vehicle they wish to be able to pilot. Of course, they must allocate points to each of these skills separately in order to increase their ratings.
See the chapter entitled Movement to learn more about the specific rules concerning vehicles. 

\subsubsection{Pocket-Picking [P]}
This skill is used to remove (or occasionally add) objects from the pocket of another character, most often coins or other valuables. A character can make a resolution check against this skill in order to pilfer an object from another character. On any level of success, they have successfully picked that character's pockets. In the case of coins, on a Green success, they have stolen 60\% of the coin that character is currently holding; on a Yellow success, 70\%; on a Red success, 80\%; on a Black success, 100\%.

A character can attempt to use pocket-picking to place an object small enough to fit in the palm of their hand into the pockets of another character by making a check against their Pocket-Picking rating. They successfully do so at any level of success on this check.

Normal failure simply means that the character was not swift enough and could not reach into the other character's pocket. If the player rolls 96-100, however, their pocket-picking attempt was not only unsuccessful, but was also detected by their intended target.

\subsubsection{Scrutiny [M]}
This skill represents the character's ability to search for clues, find hidden objects, and examine objects or documents to determine their authenticity. If a character wishes to search for a specific clue, they should use this skill.  A character that does not know what they are searching for cannot find it with this skill, and should use Perception to survey their surroundings for anything. To represent the difficulty of finding the clue, the referee should establish an integer, with 1 representing a trivial search and 24 representing a monumental undertaking. The player should make a check against the character's Scrutiny rating, noting a 1, 2, 3, or 4 for a Green, Yellow, Red, or Black success, respectively. Subtract this number from that representing the difficulty of the search. If the difficulty is now 0 or lower, the search was successful. Else, the character may continue making checks in this manner until the difficulty has been reduced to 0 or lower, at which point they have found the clue. Each check represents 1 + 1d4 minutes passing as the character searches for the clue. Of course, at the referee's discretion, an event that prevents the character from continuing to search may happen at any time, and so the passage of time should be calculated between each resolution check. 

If a character wishes to find a hidden character or object, or discover a forgery, the process is very similar to that of finding a clue, the only difference being that the difficulty is replaced with a Hide Score or Forgery Score, as appropriate. Also, in the case of discovering forgeries, each check represents 1+1d4 hours of Scrutiny, rather than minutes. 

\subsubsection{Seduce [M]}
This skill is used by a character to use their appearance and wiles to get a character to lower their guard or act in a manner that may get them into legal or personal trouble afterwards. A character cannot seduce another character that is not normally attracted to their gender or race, and they cannot use Seduce to get them to do something they would never possibly do otherwise. A character requires a Green success to Seduce a character that is already friendly towards them, a Yellow success to seduce an acquaintance, a Red success to seduce a stranger, and Black success to Seduce someone that dislikes them. Someone who has a deep personal hatred for the character or is currently in combat with the character cannot be seduced by them.

\subsubsection{Weather Sense [M]}
A character may make a resolution check against their Weather Sense rating to determine the weather conditions for the next 12 hours. A Green success or better will allow them to determine the type of precipitation coming (if any), and a Red success or better will also inform them of the general amount of precipitation (if any). A Yellow success or better will also allow them to determine wind speed and direction. A character may make a Weather Sense check every 6 hours.

\section{Magic Skills}
Magic skills are placed into their own category since they have utility both inside and outside of battle. Each Magic Skill's description will give a brief summary of the sort of effects that fall under the scope of that skill, as well as a list of example effects and the level of success required to manifest that effect. This level of success will be noted by a [G] for a Green Success, [Y] for Yellow, [R] for Red, and a [B] for Black. Players should note that the stresses of battle are not conducive to intricate spells or rituals, and that magic used in combat generally causes diminished effects for a shorter duration than magic used outside of combat. Please see the Chapter entitled \textbf{Magic} in order to learn more about how magic is used both in and out of combat.

\subsubsection{Alchemy}
This skill is the one of the few magic skills that has little use in combat. It allows a character to transmute metals from one to another, and to brew potions and other magical concoctions. In order to transmute a metal, the character must have a significant amount of the metal in question, and any appropriate reagents dictated by the referee. Depending on how exotic the base metal and the desired metal are, these reagents may be rare enough to require the alchemist to pay an exorbitant sum or go on a quest.

If the character has the proper materials for a transmutation, they may make a check against their Alchemy skill. Their transmutation attempt has an efficiency as indicated in the list below, according to their level of success. This efficiency is given as a percentage of mass transmuted. If, for example, a character is attempting to transmute 30 ounces of \emph{Oolite} into Platinum, a Green success would result in 3 ounces being transmuted, and the rest destroyed in the attempt. 

\textbf{Example Alchemy Effects}: 
	\begin{itemize}
		\setlength{\itemsep}{0cm}%
  		\setlength{\parskip}{0cm}%
		\item{ \small 10\% transmutation efficiency [G]}
		\item{ \small Sleeping potion or powder [G]}
		\item{ \small Vomit-inducing philter [G]}
		\item{ \small Gas that evokes feelings of love [Y]}
		\item{ \small Gas that causes intense vertigo [Y]}
		\item{ \small 40\% transmutation efficiency [Y]}
		\item{ \small Knockout gas [R]}
		\item{ \small 70\% transmutation efficiency [R]}
		\item{ \small Deadly vapors [B]}
		\item{ \small 100\% transmutation efficiency [B]}
	\end{itemize}

\subsubsection{Artifice}
A character can use Artifice to imbue crafted objects with magical effects. They can also use Artifice to create Golems, mindless objects that are suited for automatically performing simple tasks, and Egos, items that have a mind of their own. Using artifice in combat imbues items that already exist with temporary effects that last on the order of a few combat turns. Using artifice outside of combat is done alongside another skill, usually Blacksmithing, Goldsmithing, Sculpture, or Woodworking, and results in an object that is permanently imbued with some magical effect.

\textbf{Example Out-of-combat Artifice Effects}: 
	\begin{itemize}
		\setlength{\itemsep}{0cm}%
  		\setlength{\parskip}{0cm}%
		\item{ \small A child-sized Golem that slowly moves light objects from one place to another [G]}
		\item{ \small An object that permanently casts off a dim light [G]}
		\item{ \small An object that glows in the immediate presence of a particular creature [Y]}
		\item{ \small A ring that gives a +1 to a single soft skill [R]}
		\item{ \small A weapon that always adds +2 to its wielder's attack roll [R]}
		\item{ \small Armor that provides a +2 against a specific damage type (like piercing) [R]}
	\end{itemize}

\textbf{Example In-combat Artifice Effects}: 
	\begin{itemize}
		\setlength{\itemsep}{0cm}%
  		\setlength{\parskip}{0cm}%
  		\item{ \small Repairing a nearby broken weapon or piece of armor [G]}
  		\item{ \small Making a single piece of armor provide +1 against a specific damage type for 1+1d4 turns [G]}
  		\item{ \small Making a single weapon add +1 to its wielder's attack roll for 2+1d6 turns [Y]}
  		\item{ \small Making all nearby swords sharper, adding +1 to their wielders' attack rolls for 1+1d4 turns [R]}
  		\item{ \small Imbuing a weapon with fire, dealing +2 fire damage for 1+1d4 turns [R]}
	\end{itemize}
Black successes could cause effects similar to those above (in or out of combat), but at greater magnitudes, left to the referee's discretion.

\subsubsection{Druidism}
Druidism is borne of a spellcaster spending much time in nature, communing with the spirits of the forest. As such, Druids have a modicum of control over the plants and animals they encounter in the wild. Of course, creatures that are disharmonious with nature, such a demons, creatures from another plane of existence, or manufactured creatures do not fall under the purview of the Druidism skill.

\textbf{Example Out-of-combat Druidism Effects}: 
	\begin{itemize}
		\setlength{\itemsep}{0cm}%
  		\setlength{\parskip}{0cm}%
		\item{ \small Sense nearby animals [G]}
		\item{ \small Ask a single simple question of the surrounding trees [G]}
		\item{ \small Command a nearby animal to carry out a simple duty [Y]}
		\item{ \small Repel all nearby animals from your presence as you travel [Y]}
		\item{ \small Command a nearby monstrous creature to carry out a simple duty [R]}
		\item{ \small See through the eyes of a nearby creature [R]}
		\item{ \small Communicate through the mouth of a nearby creature [B]}
	\end{itemize}
	
\textbf{Example Out-of-combat Druidism Effects}: 
	\begin{itemize}
		\setlength{\itemsep}{0cm}%
  		\setlength{\parskip}{0cm}%
		\item{ \small Pacify a nearby animal [G]}
		\item{ \small Cause a nearby root to trip a character [G]}
		\item{ \small Cause roots and grasses to reduce the Speed of nearby characters by 2 for 1+1d4 turns [Y]}
		\item{ \small Cause a nearby animal to fight for your side [Y]}
		\item{ \small Cause roots and grasses to reduce the Speed of nearby enemies by 2 for 1+1d4 turns [R]}
		\item{ \small Cause a nearby neutral monstrous creature to fight for your side [R]}
		\item{ \small Cause a nearby tree to rise and fight for your side [B]}
	\end{itemize}

\subsubsection{Insight}
Insight is the art of gleaning information about events that are distant in space, time, or both. This information could be the result of divine revelation or just a spellcaster's natural ability. In order to affect an Insight spell effect that targets a specific person, the caster must have an object that is connected to that person in some way (they previously owned it, it was once a part of their body). If a spellcaster uses Insight to glean information about the future, it is dependent on the referee's discretion and the setting as to whether or not this future can be changed.

\textbf{Example Insight Effects}:
	\begin{itemize}
		\setlength{\itemsep}{0cm}%
  		\setlength{\parskip}{0cm}%
		\item{ \small See OR hear an event distant in space OR time [G]}
		\item{ \small See OR hear the events currently happening around a specific target [Y]}
		\item{ \small See AND hear an event distant in space OR time[Y]}
		\item{ \small Ask a deity or other external force for advice [R]}
		\item{ \small Predict a future event or the outcome of an impending choice/conflict [B]}
	\end{itemize}

\subsubsection{Mentalism}
Mentalism is what allows a spellcaster to move \emph{information} from one mind to another. This includes, but is not necessarily limited to, reading minds, communicating telepathically, mind control, and creating illusions. The range in which a spellcaster can use Mentalism on another character is about the size of a large city. If the spellcaster has a very strong personal connection with the intended target, they may be able to communicate telepathically outside of this range. The complexity of a task, as well as a target's inclination to carry out such a task, may increase or decrease the required level of success to force them to perform that task, according to the referee's discretion.

In combat, a Mentalist can also make a target feel as if they have suffered a specific wound. This ability can be used on any character within range. Such a wound wound is identical to any other wound, except that its effects last until the character has has spent 10+1d10 minutes resting outside of combat. A character who wishes to use mentalism on any character in combat must make a check against (Mentalism - target's Willpower). The spellcaster may then inflict a wound on the target equal in magnitude to their level of success, on a body area of their choice. Alternately, they may impart an effect similar to those in the "In-combat" chart below.  

\textbf{Example Out-of-Combat Mentalism Effects}:
	\begin{itemize}
		\setlength{\itemsep}{0cm}%
  		\setlength{\parskip}{0cm}%
		\item{ \small Speak to a specific target telepathically [G]}
		\item{ \small Read the mind of a willing target [G]}
		\item{ \small Detect the presence of nearby sentient creatures [G]}
		\item{ \small Visit a target in their dreams, capable of only speech [Y]}
		\item{ \small Cause a character to carry out a simple task of your choice [R]}
		\item{ \small Read the mind of an unwilling target [R]}
		\item{ \small Cause a lingering illusion to manifest itself in an area [B]}
	\end{itemize}
	
\textbf{Example In-Combat Mentalism Effects}:
	\begin{itemize}
		\setlength{\itemsep}{0cm}%
  		\setlength{\parskip}{0cm}%
		\item{ \small Grant yourself a +2 to avoid a specific character's next action [G]}
		\item{ \small Cause a character to ignore you as an attack target the next 1+1d4 turns [Y]}
		\item{ \small Grant yourself a +2 to avoid all attacks for the next 1+1d4 turns [R]}
		\item{ \small Pacify a character for 1+1d4 turns [B]}
	\end{itemize}
	
	
\subsubsection{Necromancy}
Necromancy is one of the more perverse arts a spellcaster can pursue. It is the art of raising and forcibly commanding the dead. Necromancers can resurrect recently fallen foes and force them to fight by their side for a short time. Or they can resurrect a long-dead pirate king, forcing him to reveal the location of his buried hoard. Necromancers can also use their dark arts to weaken the living in melee range, sapping their life force with nigromantic magics. In order to contact a specific spirit, or to raise any creature bodily from the dead, the spellcaster must have a piece of their body or an object they were very strongly emotionally attached to.

If a spirit is contacted, they will be compelled to answer any three questions asked by the spellcaster, but they may choose to answer additional questions beyond the first three. Spirits are tied to a physical location or object, and if the spellcaster leaves the area or object, the spirit will dissipate and must be re-conjured.
 
If a dead character (human or animal) is resurrected bodily, they will perform a single deed (limited by the mental, physical, and magical characteristics they possessed in life) for a limited amount of time (if they are repeating a simple task or it has end state, such as collecting firewood indefinitely or standing guard) or until completion (if the task has an end state, such as moving a set amount of cargo to a set location or painting a house), after which they will die once more. Human characters that are resurrected again will be hostile to the spellcaster, and must be convinced or coerced to act (perhaps through Mentalism). 

Attempting to cast any Necromancy spells, regardless of success, causes the character to suffer a Green wound.

\textbf{Example Out-of-Combat Necromancy Effects}:
	\begin{itemize}
		\setlength{\itemsep}{0cm}%
  		\setlength{\parskip}{0cm}%
		\item{ \small Contact a nearby random spirit, capable of answering Yes/No questions  [G]}
		\item{ \small Conjure a specific spirit, capable of answering Yes/No questions [Y]}
		\item{ \small Contact a random nearby spirit, capable of answering any question [Y]}
		\item{ \small Resurrect a large natural animal or monster for 1+1d4 hours [R]}
		\item{ \small Resurrect a human for 1+1d4 hours [B]}
	\end{itemize}
	
\textbf{Example In-Combat Necromancy Effects}:
	\begin{itemize}
		\setlength{\itemsep}{0cm}%
  		\setlength{\parskip}{0cm}%
		\item{ \small Cause a nearby skull or other skeletal remains to move about, filling nearby sensible foes with Terror (see the chapter on Magic)  [G]}
		\item{ \small Cause a character to suffer a Yellow wound  [Y]}
		\item{ \small Conjure a black cloud with a 15-foot radius that causes all characters to suffer -2 to their Speed while within and for 1+1d4 turns afterward [Y]}
		\item{ \small Cause a nearby uncontrolled undead or spiritual character to come under your command for 1+1d4 turns [R]}
		\item{ \small Resurrect a human foe that has died in this combat; it will attack the nearest character, whether former friend or foe [B]}
	\end{itemize}
 
\subsubsection{Psychokinesis}
This skill governs a character's ability to move objects with sheer force of will. This skill is unique in that its use in and out of combat is essentially the same. Characters with this skill are theoretically capable of moving objects of any size at any speed, but their ability to do so is limited by their confidence, experience, and ability to visualize such movements. Characters using this ability are not normally capable of fine manipulations, such as engraving jewelry or defusing a bomb (although a Black success can achieve almost anything). Characters with this skill receive a +2 to all checks to attack with a thrown weapon. 

If a character desires to move (or stop) an object or character, they may make a check against Psychokinesis or (Psychokinesis - target's Speed) in the case of a moving target. They can effect a change in the object's momentum, in foot-pounds per second, equal to 50, 100, 200, or 400, whether they achieve a Green, Yellow, Red, or Black success, respectively. The object will maintain movement with their new momentum as long as the character can see the object and maintains concentration on the object, forgoing any other actions. If their concentration is broken forcibly, such as if they suffered a physical blow, the object begins moving as it would naturally, most likely under the effect of physical laws such as gravity. If they chose to end their concentration willingly, they may choose to let the object fall or make another check against Psychokinesis if they wish to hit a specific target. Of course, they must make a check against (Psychokinesis - target's Speed) if they wish to hit a target capable of avoiding the attack. The level of success to hit the target depends on the target's size relative to the projectile being moved, according to the referee's discretion. 

In times of great desire or desperation, a character can achieve a momentum far greater than those indicated on the table for their level of success, although they will incur a physical toll afterwards (such as a Yellow wound, a temporary stun, or falling unconscious) as adjudicated by the referee.

\subsubsection{Reiki}
Reiki is the skill of manipulating the flow of latent magical energy present in all beings. Characters using this skill must be within melee range in order to affect another character with it. If they wish to heal a character, they must get within melee range of that character and make a check against Reiki. They heal the character of one wound that is of equal or lower magnitude to their level of success (a Yellow success heals a Yellow or Green wound). Alternatively, a Green or higher success can alleviate the negative effects of a Specific Wound for 1+1d4 combat turns. 

In order to damage an enemy using Reiki, a character must first roll to hit the enemy character using their Unarmed Combat skill. They then make a check against their (Reiki - target's Toughness) skill, inflicting a wound to their enemy equal in magnitude to their level of success. 

\subsubsection{Shamanism}
Whereas Necromancy is the skill of dominating the undead and subjugating them to the spellcaster's will, Shamanism is the skill of interacting harmoniously with the undead. That being said, Shamans cannot coerce the dead to act in any way, so their relationships are often those of mutual benefit. They can, however, incapacitate unruly spirits or corporeal undead. They can do the same to controlled undead, but they must make a check against (Shamanism - the Willpower of the undead's controller).

Characters that practice Shamanism are skilled in the art of conferring with the spirits of the dead, and many Shamans often allow spirits with whom they have a deep personal bond to possess their bodies for a short time. Masters of this skill may also enable a spirit to manifest itself physically in their presence. In order for a Shaman to induce these magical effects, they must be holding an object to which a spirit has strong ties, and the spirit itself must allow the spellcaster to do so. 

The spirits that a Shaman interacts with are characters for all intents and purposes. They have skills, and probably quirks and weaknesses as well. When possessed by a spirit, a Shaman receives a bonus to all the skills in which a spirit has a rating (positive or negative), equal to 1/2 the magnitude of that rating. A spirit that is physically manifest has all the skills it possessed in life, and can use them to act on the outside world. A Shaman typically must go on a harrowing quest to find an object to which a spirit has strong ties. The referee (and possibly the player) should take time to flesh out a spirits skills, personality, and desires, as they will likely be the Shaman's close companion.

The effects that can be achieve by making a Shamanism check are similar whether the characters is in or out of combat. In combat, manifesting a spirit has a duration on the order of turns, whereas it lasts on the order of hours out of combat. If a spirit enters or leaves combat, convert the units of time they have left from hours to turns (or vice versa, as appropriate) at a one-to-one ratio. 

\textbf{Example Shamanism Effects}: 
	\begin{itemize}
		\setlength{\itemsep}{0cm}%
  		\setlength{\parskip}{0cm}%
		\item{ \small Detect nearby spirits [G]}
		\item{ \small Speak with nearby spirits [G]}
		\item{ \small Incapacitate undead for 1+1d4 turns [Y]}
		\item{ \small Allow a spirit to possess the caster [R]}
		\item{ \small Allow a spirit to physically manifest for 2+1d6 hours/turns [B]}
	\end{itemize}

\end{multicols}
\end{document}