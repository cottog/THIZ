\documentclass[oneside]{book}
\usepackage{hyperref}
\hypersetup{
  colorlinks   = true,    % Colours links instead of ugly boxes
  urlcolor     = blue,    % Colour for external hyperlinks
  linkcolor    = blue,    % Colour of internal links
  citecolor    = red      % Colour of citations
}
\usepackage[table,xcdraw]{xcolor}
\usepackage{nonfloat}
\usepackage{multicol}
\usepackage{graphicx}
\setcounter{secnumdepth}{0}
\makeatletter
\renewcommand{\@dotsep}{10000} 
\makeatother
\begin{document}

\frontmatter

\title{ {\Huge THIZ} \\ \vspace{2 mm} {\large THIZ is Heavily Inspired by ZeFRS}}
\date{}
\author{Big Tex}
\maketitle

\pagenumbering{gobble}
\paragraph{As the subtitle suggests, this work is inspired by ZeFRS, a role-playing system that is adapted from an original body of rules written by Dave "Zeb" Cook. This rulebook has been written and compiled by Big Tex} 

\paragraph{The author would like to extend a great thanks to Zeb Cook, as well as the author of ZeFRS, Mark Krawec. Additional thanks are due to everyone else involved in the ZeFRS project: \\}

\begin{minipage}[t]{0.5\textwidth}
Jason Vey \\
Drake2000 \\ 
E.T. Smith \\
The Evil DM \\
Max \\
M\"{u}nch \\
Artikid \\
Marius \\
hive\_mind \\
Harmast
\end{minipage}
\begin{minipage}[t]{0.5\textwidth}
Insect King \\
blu\_sponge \\
DMAndrew \\
MountZionRyan \\
\"{A}rk\"{a}s \\
hatheg-kla \\
Vagabond \\
Christopher V. Brady \\
Spinachcat
\end{minipage}

\paragraph{If you're interested in reading more about ZeFRS, please visit the following links: \\
\url{http://www.midcoast.com/~ricekrwc/zefrs/} The ZeFRS project home\\
\url{http://zefrs.proboards.com/} The ZeFRS Project Forums \\
\url{http://forum.rpg.net/showthread.php?327143} ZeFRS Project thread \\
\url{http://forum.rpg.net/showthread.php?206835} Where it all started }


\paragraph{The author was not personally involved in the ZeFRS project and lays no claims of ownership to said body of work. The same can be said for the original work by Zeb Cook. Parts of this work, most especially the skills/conflict resolution system are taken directly from ZeFRS itself. The author feels many other areas of the ZeFRS project have been modified enough to warrant a distinct publication. \\ \\
This work may be freely distributed as long as it remains complete and intact, which includes this credits page. The text of this work is designated as "copyleft," which means it can be used, modified, and reproduced at no cost. It would be appreciated, however, if credit would be given to the original author, Dave "Zeb" Cook, and those involved in the ZeFRS project.}

\tableofcontents

\chapter{Introduction}

While THIZ was inspired by a set of rules intended for a Swords n' Sorcery role-playing experience, THIZ strives to be a system that facilitates any and all settings and genres. The fluff is left to the Referee, while THIZ provides a full-fledged set of rules that strives to leave the players feeling powerful without bogging down gameplay. There are no races, very few restrictions with regard to magic, and a potentially limitless number of skills for player characters to master. 

\section*{The Obligatory Explanation of Role-playing Games}

Many people reading this rulebook likely already know what a role-playing game is. For those who are net yet part of this fortunate group, a role-playing game is a shared narrative, in which all participants have an opportunity (and perhaps an obligation) to tell the story as it unfolds. Each character has a role not only in the party but in 
the world as a whole. Each player's actions, reactions to events, and interactions with non-player characters (NPCs for the uninitiated) coalesce to form a living world, a compelling backdrop for a grand adventure (or series of adventures). \\
THIZ fits into this picture by adding an element of order and providing a means to not only define each character but also their ability to act upon the world around them. This rulebook provide a means for the referee (sometimes referred to as the Game Master) to arbitrate contests or conflicts between different characters. The Referee is the player who tells most of the story, providing the backdrop and setting the stage for the other players. Each player is in control of one or more characters, who act as their avatars in the game. 

\section*{Requisite Supplies}

You'll need a referee and at least one other player. A pencil and paper will be useful, as will the character sheet included at the end of this book. Additionally, you will need to be able to randomly generate a number between 1 and 100. This is normally achieved using a pair of ten-sided dice, often referred to as a d100, or percentile dice.  This set of dice typically has one die with numbers from 10 to 100 and one die that ranges from 0 to 9, but alternately you could use two dice with a similar range of numbers, as long as they are of a different color, and you consistently designate one color to serve as one part of the percentage. Finally, you will need a copy of the resolution table that is included in this book. This is used to resolve any conflicts or contests between characters. Once a d100 has been rolled and a random number generated, this number will be used in conjunction with the resolution table to determine the level of success of a character's actions. Enough talk. Let's get to the game!

\mainmatter
\pagenumbering{gobble}

\chapter{Character Creation}

\section{What Exactly is a \emph{Character?}}
As you may have guessed, this chapter intends to outline the process of creating a character. In order to create a character, it is necessary to understand what exactly defines a character in the world of THIZ. Characters in THIZ have a name and a homeland, both of which are mostly fluff; and a set of skills, which defines the limits of what they can do.

\section{Skills?}
Skills in THIZ are probably the most complicated aspect of the game. This complication arises from the many subgroups they are divided into. Skills are first divided into three broad categories: \textbf{Hard Skills}, which are the skills that directly affect a character's combat prowess, and \textbf{Soft Skills}, which are the skills that help a character perform useful actions outside of combat, and \textbf{Magic Skills}, which are the skills that a character utilizes to manifest (hopefully) useful magical effects. Hard Skills are then divided into two categories: \textbf{Mettle}, which are the skills that a character uses to avoid or endure attacks and \textbf{Efficacy}, which are the skills that help a character perform physical attacks. Soft Skills are divided into three categories as well: \textbf{Expertise}, which are the skills that govern a character's intellectual ability and critical thinking skills; \textbf{Awareness}, which are the skills that a character uses to perceive the world around them; and \textbf{Body}, which are skills that delineate a character's physical abilities and athleticism.

A Rating of 0 in a Skill denotes average human ability in that skill. A Rating of 7 in a Skill denotes an expert-level ability in that Skill's purview. A Rating of 30 is the highest a character can achieve, and represents the pinnacle of human achievement in that Skill's domain.

 \newpage
\begin{multicols}{2}
\section{Creating a Character}
Usually a character begins with a concept. Would you like to play as a taciturn warrior who charges into the battlefield wielding two spiked shields, bashing their enemies until either he or they stop moving? A fast-talking necromancer with a weakness for drink and a fear of the sun? An unusually short archer with a seething hatred for the fair-skinned folk of \emph{Dwyrain}?  Coming up with a basic concept can sometimes be the most difficult part of character creation. Sometimes is may be easier to choose which skills you'd like your next character to have and come up with the concept later, but establishing a back-story for your character is an integral step in the character creation process. It may not be very fun for the other players if they have to adventure with "Guard \#12." Record this back-story in the "The Story Begins" section of your character sheet. The following three sections will help you to flesh out this section.

\subsection{Your Character's Name}
Coming up with a name for your character is also an important step in creating a character. As your character's fame (or infamy) grows, their name will often precede them, spreading across the land with tales of their exploits and derring-do. As such, it would be a good idea to ensure that the name is enjoyable to you and fits into the setting of your campaign. Often, a character with a single name and an eponym is sufficient. If you need inspiration, as either a player or referee, \emph{Gary Gygax's Extraordinary Book of Names} is a great resource.

\subsection{Homeland}
Your character's homeland is the place of their birth. This will affect the way they look, their customs, and their native tongue. Their homeland may also restrict the skills they start the game with. A character born in the capital city of a very industrialized nation may not have any need to learn the arts of Tracking, for example. 

The homeland of your character is determined as much by the physical location of their birth as by the heritage(s) of their parents. 

Your character is automatically proficient in the common language of their homeland, as well as any one language which their parents use to communicate. They can speak both of these languages, but they do not necessarily have to be proficient in reading or writing either of them. At the referee's discretion, one of these languages may be a sort of \emph{lingua franca} used by all of the neighboring nations. 

\subsection{Parents}
Your character should have at least one parent or legal guardian, and while this book may refer to parents in the plural, it is perfectly acceptable for a character to have a single parent. Even it's simply the matron (or patron) of a rural orphanage, at some point your character had someone else to care for or mentor them. Decide on a name for your character's parents, as well as their careers. Careers can be based on any of the skills listed Chapter 2, or can be some profession of your own creation. Keep in mind that, at character creation, your character will have to invest at least one skill point in a skill that represents one of their parent's careers. This is meant to represent the career path that your character would have chosen had they not heeded the call of adventure. As such, if their back-story dictates that they have some sort of mentor that isn't one of their parents, a skill may be chosen to represent that person's career.

Now that you've finished choosing a name, homeland, and some details regarding your character's parents, you can fill in the section on your character sheet that reads "The Story Begins."

\section{Your Character's Skills}
Skills are the real meat of your character. They define the sum of your character's experiences and knowledge. At character creation, you have 20 skills points to allocate to your character. Each point may be exchanged at a 1:1 ratio to increase your rating in that skill. Each character begins with all skills at a rating of zero. You may allocate points to any skill in the table below as well as any skills that your referee may have added to the skill pool, with the following restrictions:
	\begin{itemize}
		\item{ \small You must allocate at least five points each on both \textbf{Hard Skills} and \textbf{Soft Skills}}
		\item{ \small You may allocate no more than 5 points on any one skill}
		\item{ \small You must allocate at least one point to skill representing a parent or mentor's profession}
		\item{ \small You must not allocate points to any skills which your referee has deemed incompatible with the setting}
\end{itemize}	

There is an additional restriction on choosing Magical Skills: for each unique magical skill your character begins play with, you must choose a \textbf{Weakness} for that character to also possess. Weaknesses incurred for this reason should be chosen from the Mental or Magical Weaknesses. More information on weaknesses can be found in Chapter 3.

When choosing skills, you should also keep in mind that for your character to learn a new skill after character creation, you must spend 5 skill points to get a rating of 1 in that skill. To learn more about each of these skills, refer to Chapter 2, where each is described in detail.

To further customize your character, you may choose to give them \textbf{Weaknesses} or \textbf{Quirks}. These are described in detail in Chapter 3. Choosing a Weakness (beyond those you may have incurred for learning a magical skill) grants you 5 more skill points to allocate to your character. Alternatively, adding a Quirk to your character costs 5 skill points. As you may have guessed, Weaknesses negatively affect your character: they may restrict what other skills they may learn or impose penalties under certain conditions. Quirks, on the other hand, may provide small perks or skill boosts when making certain resolution checks.
\end{multicols}
\newpage
\section{Table of Skills}


\begin{table}[h]
\begin{tabular}{|c|c|}
\hline
\rowcolor[HTML]{000000} 
{\color[HTML]{FFFFFF} \textbf{Hard Skills}} & {\color[HTML]{FFFFFF} Soft Skills}         \\ \hline
\rowcolor[HTML]{656565} 
{\color[HTML]{FFFFFF} \textbf{Mettle}}      & {\color[HTML]{FFFFFF} \textbf{Expertise}}  \\ 
\rowcolor[HTML]{9B9B9B} 
\multicolumn{1}{|l|}{\cellcolor[HTML]{9B9B9B}{\color[HTML]{000000} \textbf{\begin{tabular}[c]{@{}l@{}}\\ Reflexes {[}P{]}\\ Speed {[}P{]}\\ Strength {[}P{]}\\ Toughness {[}P{]}\\ Willpower {[}M{]}\end{tabular}}}} & \multicolumn{1}{l|}{\cellcolor[HTML]{9B9B9B}\textbf{\begin{tabular}[c]{@{}l@{}}Appraise {[}M{]}\\ Artisan (Choose one) {[}M{]} \\ Forgery {[}M{]} \\ Knowledge (Choose one) {[}M{]}\\ Lock-picking {[}M{]}\\ Persuasion {[}M{]}\\ Profession (Choose one) {[}M{]}\\Trade (Choose one) {[}P{]}\\ Trapping {[}M{]}\end{tabular}}}   \\ \hline
\rowcolor[HTML]{656565} 
{\color[HTML]{FFFFFF} \textbf{Efficacy}}    & {\color[HTML]{FFFFFF} \textbf{Body}}       \\ 
\rowcolor[HTML]{9B9B9B} 
\multicolumn{1}{|l|}{\cellcolor[HTML]{9B9B9B}\textbf{\begin{tabular}[c]{@{}l@{}}Dual-Wield [P]\\ Grappling {[}P{]}\\ Tactics {[}M{]} \\ Unarmed Combat {[}P{]}\\ Weapon Skill (Choose one) {[}P{]}\end{tabular}}}                                                & \multicolumn{1}{l|}{\cellcolor[HTML]{9B9B9B}\textbf{\begin{tabular}[c]{@{}l@{}}Acrobatics {[}P{]}\\ Climbing{[}P{]}\\  Sleight of Hand {[}P{]}\\ Stealth {[}P{]}\\ Swimming{[}P{]}\\ Throwing {[}P{]}\end{tabular}}}                                                      \\ \hline
\rowcolor[HTML]{333333} 
{\color[HTML]{FFFFFF} \textbf{Magic Skills}}  & \cellcolor[HTML]{656565}{\color[HTML]{FFFFFF} \textbf{Awareness}}    \\
\rowcolor[HTML]{9B9B9B} 
\multicolumn{1}{|l|}{\cellcolor[HTML]{9B9B9B}\textbf{\begin{tabular}[c]{@{}l@{}}Alchemy\\ Artifice \\ Druidism\\ Insight\\ Mentalism\\ Necromancy\\ Psychokinesis\\ Reiki\\ Shamanism\\ Summoning\end{tabular}}}                          & \multicolumn{1}{l|}{\cellcolor[HTML]{9B9B9B}\textbf{\begin{tabular}[c]{@{}l@{}}Animal Handling {[}M{]}\\ Fast Talk {[}M{]}\\ First Aid {[}M{]}\\ Magic Sense {[}P{]}\\ Navigation {[}M{]}\\ Perception {[}P{]}\\ Pilot (Choose One) {[}M{]}\\ Pocket-Picking {[}P{]}\\ Scrutiny {[}M{]}\\ Seduce {[}M{]}\\ Tracking {[}P{]} \\ Weather Sense {[}M{]}\end{tabular}}} \\ \hline
\end{tabular}
\end{table}

\section{General Skill Ratings}
In addition to the actual skill ratings your character possesses, they also have three general skill ratings: \textbf{Physical}, \textbf{Mental}, and \textbf{Magical}. These scores are calculated by taking your character's rating in the appropriate skills, summing these ratings, and dividing the result by ten, rounding down. The skills in the above chart marked with a [P] contribute to your character's Physical rating, whereas those marked with a [M] contribute to their Mental rating. All Magic Skills contribute to their Magical rating. 

\onecolumn
\section{Starting Equipment}
A character in THIZ typically starts with very little equipment. They are just beginning their life of adventure, often setting out from home to seek better fortunes elsewhere. A character typically starts play with a single weapon, 1d8 gold coins, and two pieces of mundane equipment. This equipment could be anything from a piece of armor, to a length of rope (100 feet) and a grapple, to a set of tailor's tools, to a small horse or mule, or anything else of that sort. 

\section{Luck}
Every player character in THIZ is destined for greatness. As such, fate itself may intervene before the character can come to an untimely end. Of course, the character can forge their own destiny, and, at times, it seems their whims dictate the events that unfold around them.  At character creation, the referee secretly rolls 1d10 for each player character. This number should not be revealed to the player, but the referee should remember this number for the duration of the campaign. To see how Luck can be used during play, please see the appropriate section in the chapter on \textbf{Character Advancement and Time Management}.

\chapter{Skills}
This chapter discusses the various skills available to your character in greater detail. Those marked with an [M] or [P] contribute to a character's Magical or Physical general skill rating. All Magic Skills contribute to their Magical general rating.

With the exception of Magic Skills, which cannot be used unless a character has learned them, a character makes resolution checks using the higher between their rating in the appropriate skill or that skill's governing general rating (Physical or Mental). For example, a character with no rating in Unarmed Combat and Physical rating of 4 makes resolution checks for Unarmed Combat against a base rating of 4 (situational modifiers may affect this rating as any other). A character attempting to make a check a check against an Artisan, Trade, or Profession skill for which they must use their General Rating (they are untrained in said Skill) takes a -5 to that General Rating for their lack of specialized training or knowledge.
\newpage
\begin{multicols}{2}

\section{Hard Skills}
\subsection{Mettle}
\subsubsection{Reflexes [P]}
This is the skill that a character uses to quickly avoid traps or react to the events that occur around them, most usually in battle. This skill also increases a character's initiative score. Traps are discussed further in the chapter entitled \textbf{Hazards}.

If a character delays their action in combat and wishes to take an action \emph{before} the event that they were waiting for, they must make a check against their Reflexes rating at the moment they wish to act. On a Yellow success or better, this character may make that action. Otherwise, they take their action at the end of the initiative pass.

\subsubsection{Speed [P]}
This skill determines a character's movement speed, and is also used by characters to avoid enemy attacks. Movement speed is discussed in the chapter entitled \textbf{Movement}.

\subsubsection{Strength [P]}
This skill determines how heavy of an object a character can lift, as well as how heavy of a load they can carry on their backs as they travel. A character can overhead lift 100 pounds plus an additional 10 pounds for every point of Strength. A character can lift double this amount off the ground, moving as if they had a Speed rating of 0. They can move in this way for the same amount of time they can Sprint (see the chapter on Movement).

Additionally, characters receive a +1 bonus to melee attack rolls for every 10 full points they have in Strength (1 at 10, 1 at 15, 2 at 20).

\subsubsection{Toughness [P]}
This is the skill that characters use to shrug off the effects of attacks they receive in combat, as well as the effects of poison or other hazards. The chapters entitled \textbf{Combat} and \textbf{Dangers} explain the use of this skill in more detail. This skill also governs how long a character can spend sprinting and jogging. See the chapter on \textbf{Movement} for more details.

\subsubsection{Willpower [M]}
Willpower governs a character's ability to resist their internal compulsions and desires, as well as those that others may attempt to impose upon them. Characters usually use Willpower to negate or lessen the effects of their Weaknesses. Willpower may also help a character ignore certain magical attacks.
\\
\\
\subsection{Efficacy}
\subsubsection{Dual-Wield [P]}
This skill allows a character to use a pair of any two one-handed weapons together in combat, usually attacking with both of them at once (see the chapter entitled Combat). A new instance of this skill must be learned and allocated its own points for each combination of weapons the character wishes to use. The character must also be trained in the use of both weapons. 

For example, a character can put 3 points into Dual-Wielding [Rapier, Pistol] and 5 points into Dual-Wielding [Falchion, Axe] to get a rating of 3 and 5 in those skills respectively. When taking this skill, the character must list the two weapons such that the weapon they are wielding in their main hand is listed first. A character suffers a -2 penalty for wielding weapons in the wrong hand. A character may declare a spiked piece of armor as part of a weapon combination, if they are trained in its use. 

\subsubsection{Grappling [P]}
This skill is used by characters when they grapple with or attempt to pin an opponent. Please see the chapter on Combat for more information on grappling.

\subsubsection{Tactics [M]}
This skill is for characters who wish to direct their ally's actions on the battlefield, informing them how best to defeat their enemies in any situation. Whenever three or more allies are in combat with a specific enemy, a character may use their combat turn to make a check against their Tactics skill. Based on their level of success, each ally in combat with that specific enemy at the time of this check receive a bonus on their next turn on any resolution checks involving combat with that enemy. This bonus is lost at the end of the next turn, and bonuses from multiple Tactics checks do not stack. The magnitude of this bonus ranges from +1 to +4, for Green through Black successes respectively. 

\subsubsection{Unarmed Combat [P]}
This skill is used by characters when they are attacking an enemy using a part of their body, eschewing the use of any weapons. Attacking in this manner normally cannot damage armored Adversaries. If a character has a Strength rating that is triple the Armor Bonus of the targeted area, however, they can deal damage to an armored Adversary. A character can mitigate this Strength requirement with the use of fist weapons, such as a cestus or spiked gauntlet. See the chapter on \textbf{Combat} for more information regarding Adversaries, as well as Unarmed Combat. 

\subsubsection{Weapon Skill [P]}
This skill is used by a character when they wish to attack or perform maneuvers with a weapon. A new instance of this skill should be chosen for every weapon the character is proficient in, each with its own rating determined by points that have been allocated to it specifically. Armor spikes are considered weapons for the purposes of this skill. Characters wishing to attack with them should allocate points to this skill in order to use them effectively.

\section{Soft Skills}

A good number of these skills can help the character earn money while outside of combat, especially in the time between campaigns. See the chapter entitled \textbf{Character Advancement and Time Management} in order to see how a character may earn money by pursuing a career. 
\subsection{Expertise}
\subsubsection{Appraise [M]}
This skill is used to appraise the value of trade goods and the works of artisans and craftsmen. A character that has a particular Artisan or Profession skill can check against their skill rating to appraise works (or materials) that fall under that skill's domain, but any character that does not have the appropriate profession or craft must use the Appraise skill to judge the value of an item. A character that has both Appraise and an appropriate Artisan or Profession skill may check against the higher of the two skill ratings, with a +2 bonus, in order to appraise something. 

The level of success of this check determines the width of the price range a character may determine for a particular object. A Green success would result in a range that is $\pm$ 20\% of the object's value, Yellow would result in $\pm$ 10\%, Red $\pm$ 5\%, and a Black success would give the object's exact value.
\subsubsection{Artisan [M]}
This skill is intended for character's that intend to pursue a more artistic profession. When learning this profession, a character must choose an artistic calling, such as Goldsmithing (making jewelry from gems and precious metals), Silversmithing (making silverware, flatware, hollowware, and other such items from gems and precious metals), Painting, Sculpture etc... A character may have multiple instances of the Artisan skill for each artistic pursuit they wish to follow. Of course, they must allocate points to each of these skills separately in order to increase their ratings. 

\subsubsection{Forgery [M]}
This skill governs a character's ability to create forged documents in any language they can read and write. In order for a character to forge a specific document, they must have a copy of that document to reference while creating the document. For each full 4-hour period a character has to work on their forgery, with a minimum of 1 (8 hours or less) and up to a maximum of 6 (24 hours or more), they may make a resolution check against their Forgery rating. For each check, the player should write down a 1, 2, 3, or 4 for a Green, Yellow, Red, or Black success, respectively. After all checks have been made, the player should sum the numbers they have written down. This sum is the forged document's \textbf{Forgery Score}. A Forgery Score of 0 means the character has failed; otherwise, they have successfully created a forgery. The Forgery Score is important if another character attempts to discover the ruse.

A character wishing to create a forgery of the product of a artisan or tradesman, they must have the appropriate Artisan or Trade skill. The process of creating a forgery with one of those skills is identical to that of creating a forgery using this skill.
\subsubsection{Knowledge [M]}
This skill represents the character's breadth of knowledge and experience with a subject. When learning this skill, pick any broad topic, such as a particular nation's (or world) history, battlefield tactics, herbalism, literature, etc... A character may have multiple instances of the Knowledge skill, representing their knowledge of different domains. As usual, they must allocate skill points to these skills separately in order to increase their ratings.

Whenever a character would search their breadth of knowledge for a particular fact, they may make a resolution check against their Knowledge rating for the topic under which that fact falls. The level of success required to remember a fact is determined by how specialized this particular fact is. For example, if a character wished to remember the minutiae of a particular year in a relatively unknown Classical poet's life, they would be required to make a Black success. At any other level of success, while they may remember something about that author's life or works from that year, they won't recall anything at the level of detail they may have hoped for.

\subsubsection{Lock-picking [M]}
This skill is used by characters in order to open locked doors and containers. A character should make a check against their Lock-picking rating for each tumbler in the lock. Any level of success means that the character has successfully set the pin in that tumbler. A Green success means that this process took 1 minute, Yellow 45 seconds, Red 30 seconds, and Black 10 seconds (or lower, at the referee's discretion). A single failure simply means that the character must try to set that pin again. Two failures in a row resets all the pins that had been set, forcing the character to start picking the lock over again. If the setting supports this, three failures in a row triggers an external alarm of some sort, possibly alerting nearby characters.
\subsubsection{Persuasion [M]}
This skill is used in order to convince a character (usually an NPC) to take an action that they normally wouldn't be inclined to, or to permanently change their viewpoint on a particular matter. A character usually refuses to take an action because of material, moral, or mortal concerns. 

Convincing a character to take an action that would cause them a small material loss will take a Green success, whereas an action that would incur a greater loss of wealth or other possessions would take a Yellow or Red success, depending on the magnitude of the cost. A Black success would only be required to convince a character to take an action that would cause them to lose the entirety of their wealth. 

Convincing a character to take an action that is against the tenets of their culture or religion might require anywhere from a Green to Black success, depending on the fervency of their beliefs, and the level of taboo that particular action holds. Convincing a zealot to make even a small slight against their god might take a Black success, while convincing a casual believer to condemn themselves to damnation might take a Red success. Cultural norms are usually easier to get a character to ignore, and so a Black success would likely only be required for something as grave to them as murder is to most cultures.

Convincing a character to take an action that would put themselves or a family member in mortal danger might take a Red or Black success, depending on how assured their destruction seems. Convincing them to put a stranger in mortal danger might take a Yellow or Red success, also depending on how assured their destruction seems. 
If a player roleplays their speech to the character particularly well, the referee may decide to lower the required level of success, at their discretion. Also at the referee's discretion, this skill may be checked against in order to move a crowd with a speech.

\subsubsection{Profession [M]}
This skill is for characters who wish to pursue a career in a more "learned" field, such as Education (educating others), Mathematics, Engineering, Architecture, Medicine, Surgery, etc... A character may have multiple instances of the Profession skill for each intellectual pursuit they wish to follow. Of course, they must allocate points to each of these skills separately in order to increase their ratings. 

As Surgery falls under this skill, please see the chapter entitled \textbf{Combat} in order to see its possible uses. 


\subsubsection{Trade [P]}
This skill represents a character's knowledge of and ability in a tradeskill, such as Woodworking, Carpentry, Plumbing (if setting-appropriate), Blacksmithing, etc... Professions that fall under this skill's purview are those that are typically considered "blue-collar."  A character may have multiple instances of the Trade skill for each trade they wish to pursue. Of course, they must allocate points to each of these skills separately in order to increase their ratings. 

\subsubsection{Trapping}
This skill is used by characters who wish to construct and lay traps. This skill also represents a character's knowledge and intuition of where best to lay traps. This skill can be used to construct or detect outdoor traps for creatures up to human size. Spike pits, snares, and net traps fall under the umbrella of this skill. Indoor or mechanical traps are not governed by this skill, which would more likely fall under Engineering (a subset of Profession).

If the trap is intended to catch unintelligent prey, the referee makes a check against the character's Trapping rating, and they catch some sort of prey on any degree of success. The quality and/or quantity of prey increases as the level of success gets greater. 
\\
\\
\subsection{Body}
\subsubsection{Acrobatics [P]}
This is a skill that is used by characters that wish to jump across gaps or over obstacles. This skill would also be used if a character wished to swing from a rope or perform any other such stunts. Please see the chapter entitled \textbf{Movement} for more information regarding Acrobatics.

A character wishing to long jump may make a resolution check against their Acrobatics skill. For a standing long jump, the character can jump 8, 9, 10, or 11 feet, for Green through Black successes respectively.
If the character can run at least their jogging distance before attempting the long jump, they can jump 20, 22, 24, or 26 feet, for Green through Black successes respectively.

A character attempting a vertical jump from standing can reach heights of 32, 34, 36, or 38 inches, for Green through Black successes. A character that can run at least their jogging distance before attempting a high jump can reach a height of 60, 68, 76, or 84 inches for Green through Black successes.
\subsubsection{Climbing}
This skill is used by a character when they are scaling towers or large walls. Any character can climb a short distance, but for a character to successfully climb for an extended period of time over a large distance, they need some knowledge and experience. The chapter on Movement has more information regarding climbing. 

\subsubsection{Sleight of Hand}
This skill governs the ability of a character to successfully make an object appear or disappear from within a pocket. The difficulty required for a character to hide or reveal a particular object successfully depends on the size of the object, according to the referee's discretion. Something discrete like a coin may require only a Green Success, whereas something the size of a human head or larger would require a Black success. Any level of success below the threshold established by the referee means that the item was spotted, at least in passing, by a nearby character that was actively observing the character attempting to use Sleight of Hand. A failure means that the character has dropped the item. 

A character may also attempt to cheat at a game of chance with their Sleight of Hand skill. Any level of success means that their cheating has so far gone undetected. 
Whenever a character makes a Sleight of Hand check, subtract the highest Perception rating among all the characters who are actively observing the character making the check. 

\subsubsection{Stealth}
This skill allows a character to attempt to hide from others or move unseen. A character wishing to use stealth to move must make a check against their stealth rating minus the highest Perception rating among all the characters actively searching for them (but not currently looking at them). If they succeed, they may move up to 1, 2, 3, or 4 times their walking speed before they must make a stealth check again. On a failure, they have been spotted. A character automatically fails a stealth attempt if they are being directly observed by a character that is searching for them or would immediately reveal their location to someone searching for them. If they are currently being observed, they must first move out of sight before attempting stealth. 

If a character is attempting to hide from someone pursuing them, that person doesn't currently have them in sight, they make make a check against their stealth rating to enter a hiding spot, hide an object they are holding, or hide a character that is adjacent to them into a hiding spot of sufficient size that is nearby.

 For each full one-minute period the character has to attempt to hide, they may make a resolution check against their Stealth skill. They may make a minimum of 1 check (if they have less than two minutes to hide) and a maximum of 6 (they have 6 minutes or more to hide). As the player makes each of these resolution checks, they should write down a 1, 2, 3, or 4 if they achieved a Green, Yellow, Red, or Black success, respectively. After all checks are made, the player should sum these numbers. This sum is the character's \textbf{Hide Score}. A Hide Score greater than 0 means the character has successfully hidden whatever it is they were attempting to hide. A Hide Score of 0 means the character has failed. The Hide Score is important if another character attempts to find the hidden character or object.

See the description of the skills \textbf{Perception} and \textbf{Scrutiny} to learn more about spotting a character that is hidden. 

\subsubsection{Swimming}
Without this skill, a character can only tread water or doggy paddle in the water. In order to move any appreciable distance in the water, or to stay afloat for an extended amount of time, a character must have a rating above 0 in the Swimming skill. Please see the chapter on Movement for more information.

\subsubsection{Throwing}
This skill is used whenever a character attempts to throw a non-weapon object across any distance. A character can make a check against this skill to throw a 16-pound object 64, 68, 72, or 76 inches on Green through Black successes, respectively. This distance is inversely correlated to the object's weight. If the object were to double in weight, the possible distance it could be thrown would be halved. 

This skill is meant to represent the maximum possible distance a character could throw an object, with ample time to prepare themselves for the throw. If this skill were to be used for combat in an improvised manner, an object of any size can only be thrown a quarter the distance it could normally be thrown.

\subsection{Awareness}
\subsubsection{Animal Handling [M]}
This skill is used by a character to train and command animals. This skill is most often used to control a mount in abnormal situations. To get a mounted creature to move towards a foe or obstacle that it is terrified of (most usually an exotic or supernatural beast), the character makes a check against their Animal Handling rating. On any level of success, the creature will move towards the foe, but at a maximum rate of 1/4, 1/2, 1, or twice its movement speed.

A character with this skill can rear an animal from birth, training it for any sort of purpose they wish (as long as the referee feels that purpose is not too complex for its intelligence). This training takes a period of one year for the creature to obey without fault or hesitation. This training can occur as the character travels on adventures, as long as the creature accompanies the character. 

\subsubsection{Fast Talk [M]}
This skill is perfect for characters who wish to master the art of the grift. This skill is used by a character to convince another character of a fact that they would not otherwise believe to be true. A character who has been duped by a successful use of the Fast Talk skill will realize the deception after 10+1d10 minutes. The level of success require to successfully swindle another character depends on the unbelievability of the lie, at the referee's discretion. The difficulty can also be lowered one step if the character has role-played their attempt particularly well, also at the referee's discretion. 

\subsubsection{First Aid [M]}
This skill is used by characters to quickly heal minor wounds that they or other characters have incurred during their adventures. Please see the chapter on Combat to read more about recovering from damage.

\subsubsection{Magic Sense [P]}
This skill is used by characters who, by some means or another (most likely due to some physical mutation or knowledge of telltale physical clues), are able to identify magical effects and magic users themselves. A character with points in this skill will be able to detect magical effects or objects at a distance of 10 feet * their rating in this skill. Additionally, a character may make a check against this skill in order to determine if another character has any magical abilities. A Black success can detect a character with the tiniest vestige of magic (a rating of 1 in a single magical skill), whereas a Yellow success can detect a character with a rating of 5 or higher in a single magical skill, Red can detect a character with a rating of 9 or higher, and Green  can detect a character with a rating of 15 or higher in a single magical skill or a general Magical rating of 1 or higher. A character using this skill simply knows that something is magical, they have no knowledge (through of this skill) of the nature or strength of that magic. 

\subsubsection{Navigation}
This skill allows characters to find their bearings while traveling. The level of success required to successfully reach a destination is determined by the number of known landmarks along the way and how cloudy the sky is (whether the sun is visible if traveling by day or an appropriate number of stars is visible if traveling by night). This skill can also be used to determine which direction the character is currently facing at night (a trivial task during the day if the sun is visible, impossible by use of this skill otherwise). To gain their bearings at night, a character must simply make a check against their Navigation rating at any level of success. Otherwise, on a failure, the character mistakenly believes they are going the direction they desire. The referee randomly determines which direction they are actually traveling.

\subsubsection{Perception [P]}
This skill is used whenever a character needs to quickly spot any feature in their surroundings, such as an approaching character at the top of a hill, an ambusher's elbow poking out of a nearby bush, or the telltale signs of a trap. The level of success required to spot an approaching character or party depends on the size of the party and the distance and environmental conditions between them. It is up to the referee's discretion if a character's check against their Perception check is sufficiently successful to spot the approaching party immediately. 

A character may also attempt to use Perception to immediately spot a character or object that they believe is hiding in the room. A character wishing to do so must make a check against their Perception rating. The player should write down a 1, 2, 3, or 4, for a Green, Yellow, Red, or Black success, respectively. If this number exceeds the Hide Score (see the section on \textbf{Stealth}) of the hidden object or character, they have found it. Otherwise, they must resort to use of the Scrutiny skill.

A character can also check against their Perception rating to survey their surroundings for interesting information. The referee should describe their surroundings in greater detail as their level of success increases. 


\subsubsection{Pilot [M]}
This skill governs a character's ability to pilot a craft or vehicle. Upon learning this skill, the character should name a specific class of vehicle, such as schooner, brigandine, or perhaps helicopter or tank in a more technologically advanced setting. For vehicles that the referee decides are sufficiently complicated, a character without the appropriate Pilot skill cannot pilot them whatsoever.

A character may have multiple instances of the Pilot skill for each type of vehicle they wish to be able to pilot. Of course, they must allocate points to each of these skills separately in order to increase their ratings.
See the chapter entitled Movement to learn more about the specific rules concerning vehicles. 

\subsubsection{Pocket-Picking [P]}
This skill is used to remove (or occasionally add) objects from the pocket of another character, most often coins or other valuables. A character can make a resolution check against this skill in order to pilfer an object from another character. On any level of success, they have successfully picked that character's pockets. In the case of coins, on a Green success, they have stolen 40\% of the coin that character is currently holding; on a Yellow success, 60\%; on a Red success, 80\%; on a Black success, 100\%.

A character can attempt to use pocket-picking to place an object small enough to fit in the palm of their hand into the pockets of another character by making a check against their Pocket-Picking rating. They successfully do so at any level of success on this check.

Normal failure simply means that the character was not swift enough and could not reach into the other character's pocket. If the player rolls 96-100, however, their pocket-picking attempt was not only unsuccessful, but was also detected by their intended target.

\subsubsection{Scrutiny [M]}
This skill represents the character's ability to search for clues, find hidden objects, and examine objects or documents to determine their authenticity. If a character wishes to search for a specific clue, they should use this skill.  A character that does not know what they are searching for cannot find it with this skill, and should use Perception to survey their surroundings for anything. To represent the difficulty of finding the clue, the referee should establish an integer, with 1 representing a trivial search and 24 representing a monumental undertaking. This may very well be the \textbf{Hide Score} of someone using the Stealth Skill. See the section on \textbf{Stealth} for more.

 The player should make a check against the character's Scrutiny rating, noting a 1, 2, 3, or 4 for a Green, Yellow, Red, or Black success, respectively. Subtract this number from that representing the difficulty of the search. If the difficulty is now 0 or lower, the search was successful. Else, the character may continue making checks in this manner until the difficulty has been reduced to 0 or lower, at which point they have found the clue. Each check represents 1 + 1d4 minutes passing as the character searches for the clue. Of course, at the referee's discretion, an event that prevents the character from continuing to search may happen at any time, and so the passage of time should be calculated between each resolution check. 

If a character wishes to find a hidden character or object, or discover a forgery, the process is very similar to that of finding a clue, the only difference being that the difficulty is replaced with a \textbf{Hide Score} or \textbf{Forgery Score}, as appropriate. Also, in the case of discovering forgeries, each check represents 1+1d4 hours of Scrutiny, rather than minutes. 

\subsubsection{Seduce [M]}
This skill is used by a character to use their appearance and wiles to get a character to lower their guard or act in a manner that may get them into legal or personal trouble afterwards. A character cannot seduce another character that is not normally attracted to their gender or race, and they cannot use Seduce to get them to do something they would never possibly do otherwise. A character requires a Green success to Seduce a character that is already friendly towards them, a Yellow success to seduce an acquaintance, a Red success to seduce a stranger, and Black success to Seduce someone that dislikes them. Someone who has a deep personal hatred for the character or is currently in combat with the character cannot be seduced by them.

\subsubsection{Tracking [P]}
Tracking is the art of finding and following a trail left behind by some sort of of quarry, whether human or animal. When a character wishes to follow the trail left behind by another character, they must make a check against their Tracking Skill. Any level of success means they have found some sort of trail they can follow.

A character with training in the Tracking skill may use their knowledge to attempt to hide their trail. They must make a check against their Tracking rating. On any level of success, any pursuers must subtract this character's Tracking rating from their own when making a check to follow the trail. 

Characters using the Tracking skill move at a reduced rate of speed due to the additional effort of following or hiding a trail. On a Green success, they can travel at 25\% of their normal rate of movement. This speed increases by 25\% with each additional level of success, up to 100\% on a Black success.

For every full hour a trail is cold, any character wishing to follow it suffers a -1 to their Tracking rating when making the skill check. Weather, terrain, and the number of characters traveling together may also add or subtract from the pursuer's Tracking rating, at the referee's discretion.

\subsubsection{Weather Sense [M]}
A character may make a resolution check against their Weather Sense rating to determine the weather conditions for the next 12 hours. A Green success or better will allow them to determine the type of precipitation coming (if any), and a Red success or better will also inform them of the general amount of precipitation (if any). A Yellow success or better will also allow them to determine wind speed and direction. A character may make a Weather Sense check every 6 hours.

\section{Magic Skills}
Magic skills are placed into their own category since they have utility both inside and outside of battle. Each Magic Skill's description will give a brief summary of the sort of effects that fall under the scope of that skill, as well as a list of example effects and the level of success required to manifest that effect. This level of success will be noted by a [G] for a Green Success, [Y] for Yellow, [R] for Red, and a [B] for Black. Players should note that the stresses of battle are not conducive to intricate spells or rituals, and that magic used in combat generally causes diminished effects for a shorter duration than magic used outside of combat. Players should also keep in mind that checks can be made against a particular Magic Skill in order to identify whether something is the effect of that magic skill. Please see the Chapter entitled \textbf{Magic} in order to learn more about how magic is used both in and out of combat. 

\subsubsection{Alchemy}
This skill is the one of the few magic skills that has little use in combat. It allows a character to transmute metals from one to another, and to brew potions and other magical concoctions. In order to transmute a metal, the character must have a significant amount of the metal in question, and any appropriate reagents dictated by the referee. Depending on how exotic the base metal and the desired metal are, these reagents may be rare enough to require the alchemist to pay an exorbitant sum or go on a quest.

If the character has the proper materials for a transmutation, they may make a check against their Alchemy skill. Their transmutation attempt has an efficiency as indicated in the list below, according to their level of success. This efficiency is given as a percentage of mass transmuted. If, for example, a character is attempting to transmute 30 ounces of \emph{Oolite} into Platinum, a Green success would result in 3 ounces being transmuted, and the rest destroyed in the attempt. 

\textbf{Example Alchemy Effects}: 
	\begin{itemize}
		\setlength{\itemsep}{0cm}%
  		\setlength{\parskip}{0cm}%
		\item{ \small 10\% transmutation efficiency [G]}
		\item{ \small Sleeping potion or powder [G]}
		\item{ \small Vomit-inducing philter [G]}
		\item{ \small Gas that evokes feelings of love [Y]}
		\item{ \small Gas that causes intense vertigo [Y]}
		\item{ \small 40\% transmutation efficiency [Y]}
		\item{ \small Knockout gas [R]}
		\item{ \small 70\% transmutation efficiency [R]}
		\item{ \small Deadly vapors [B]}
		\item{ \small 100\% transmutation efficiency [B]}
	\end{itemize}

\subsubsection{Artifice}
A character can use Artifice to imbue crafted objects with magical effects. They can also use Artifice to create Golems, mindless objects that are suited for automatically performing simple tasks, and Egos, items that have a mind of their own. Using artifice in combat imbues items that already exist with temporary effects that last on the order of a few combat turns. Using artifice outside of combat is done alongside another skill, usually Blacksmithing, Goldsmithing, Sculpture, or Woodworking, and results in an object that is permanently imbued with some magical effect.

\textbf{Example Out-of-combat Artifice Effects}: 
	\begin{itemize}
		\setlength{\itemsep}{0cm}%
  		\setlength{\parskip}{0cm}%
		\item{ \small A child-sized Golem that slowly moves light objects from one place to another [G]}
		\item{ \small An object that permanently casts off a dim light [G]}
		\item{ \small An object that glows in the immediate presence of a particular creature [Y]}
		\item{ \small A ring that gives a +1 to a single soft skill [R]}
		\item{ \small A weapon that always adds +2 to its wielder's attack roll [R]}
		\item{ \small Armor that provides a +2 against a specific damage type (like piercing) [R]}
	\end{itemize}

\textbf{Example In-combat Artifice Effects}: 
	\begin{itemize}
		\setlength{\itemsep}{0cm}%
  		\setlength{\parskip}{0cm}%
  		\item{ \small Repairing a nearby broken weapon or piece of armor [G]}
  		\item{ \small Making a single piece of armor provide +1 against a specific damage type for 1+1d4 turns [G]}
  		\item{ \small Making a single weapon add +1 to its wielder's attack roll for 2+1d6 turns [Y]}
  		\item{ \small Making all nearby swords sharper, adding +1 to their wielders' attack rolls for 1+1d4 turns [R]}
  		\item{ \small Imbuing a weapon with fire, dealing +2 fire damage for 1+1d4 turns [R]}
	\end{itemize}
Black successes could cause effects similar to those above (in or out of combat), but at greater magnitudes, left to the referee's discretion.

\subsubsection{Druidism}
Druidism is borne of a spellcaster spending much time in nature, communing with the spirits of the forest. As such, Druids have a modicum of control over the plants and animals they encounter in the wild. Of course, creatures that are disharmonious with nature, such a demons, creatures from another plane of existence, or manufactured creatures do not fall under the purview of the Druidism skill.

\textbf{Example Out-of-combat Druidism Effects}: 
	\begin{itemize}
		\setlength{\itemsep}{0cm}%
  		\setlength{\parskip}{0cm}%
		\item{ \small Sense nearby animals [G]}
		\item{ \small Ask a single simple question of the surrounding trees [G]}
		\item{ \small Command a nearby animal to carry out a simple duty [Y]}
		\item{ \small Repel all nearby animals from your presence as you travel [Y]}
		\item{ \small Command a nearby monstrous creature to carry out a simple duty [R]}
		\item{ \small See through the eyes of a nearby creature [R]}
		\item{ \small Communicate through the mouth of a nearby creature [B]}
	\end{itemize}
	
\textbf{Example Out-of-combat Druidism Effects}: 
	\begin{itemize}
		\setlength{\itemsep}{0cm}%
  		\setlength{\parskip}{0cm}%
		\item{ \small Pacify a nearby animal [G]}
		\item{ \small Cause a nearby root to trip a character [G]}
		\item{ \small Cause roots and grasses to reduce the Speed of nearby characters by 2 for 1+1d4 turns [Y]}
		\item{ \small Cause a nearby animal to fight for your side [Y]}
		\item{ \small Cause roots and grasses to reduce the Speed of nearby enemies by 2 for 1+1d4 turns [R]}
		\item{ \small Cause a nearby neutral monstrous creature to fight for your side [R]}
		\item{ \small Cause a nearby tree to rise and fight for your side [B]}
	\end{itemize}

\subsubsection{Insight}
Insight is the art of gleaning information about events that are distant in space, time, or both. This information could be the result of divine revelation or just a spellcaster's natural ability. In order to affect an Insight spell effect that targets a specific person, the caster must have an object that is connected to that person in some way (they previously owned it, it was once a part of their body). If a spellcaster uses Insight to glean information about the future, it is dependent on the referee's discretion and the setting as to whether or not this future can be changed.

\textbf{Example Insight Effects}:
	\begin{itemize}
		\setlength{\itemsep}{0cm}%
  		\setlength{\parskip}{0cm}%
		\item{ \small See OR hear an event distant in space OR time [G]}
		\item{ \small See OR hear the events currently happening around a specific target [Y]}
		\item{ \small See AND hear an event distant in space OR time[Y]}
		\item{ \small Ask a deity or other external force for advice [R]}
		\item{ \small Predict a future event or the outcome of an impending choice/conflict [B]}
	\end{itemize}

\subsubsection{Mentalism}
Mentalism is what allows a spellcaster to move \emph{information} from one mind to another. This includes, but is not necessarily limited to, reading minds, communicating telepathically, mind control, and creating illusions. The range in which a spellcaster can use Mentalism on another character is about the size of a large city. If the spellcaster has a very strong personal connection with the intended target, they may be able to communicate telepathically outside of this range. The complexity of a task, as well as a target's inclination to carry out such a task, may increase or decrease the required level of success to force them to perform that task, according to the referee's discretion.

In combat, a Mentalist can also make a target feel as if they have suffered a specific wound. This ability can be used on any character within range. Such a wound wound is identical to any other wound, except that its effects last until the character has has spent 10+1d10 minutes resting outside of combat. A character who wishes to use mentalism on any character in combat must make a check against (Mentalism - target's Willpower). The spellcaster may then inflict a wound on the target equal in magnitude to their level of success, on a body area of their choice. Alternately, they may impart an effect similar to those in the "In-combat" chart below.  

\textbf{Example Out-of-Combat Mentalism Effects}:
	\begin{itemize}
		\setlength{\itemsep}{0cm}%
  		\setlength{\parskip}{0cm}%
		\item{ \small Speak to a specific target telepathically [G]}
		\item{ \small Read the mind of a willing target [G]}
		\item{ \small Detect the presence of nearby sentient creatures [G]}
		\item{ \small Visit a target in their dreams, capable of only speech [Y]}
		\item{ \small Cause a character to carry out a simple task of your choice [R]}
		\item{ \small Read the mind of an unwilling target [R]}
		\item{ \small Cause a lingering illusion to manifest itself in an area [B]}
	\end{itemize}
	
\textbf{Example In-Combat Mentalism Effects}:
	\begin{itemize}
		\setlength{\itemsep}{0cm}%
  		\setlength{\parskip}{0cm}%
		\item{ \small Grant yourself a +2 to avoid a specific character's next action [G]}
		\item{ \small Cause a character to ignore you as an attack target the next 1+1d4 turns [Y]}
		\item{ \small Grant yourself a +2 to avoid all attacks for the next 1+1d4 turns [R]}
		\item{ \small Pacify a character for 1+1d4 turns [B]}
	\end{itemize}
	
	
\subsubsection{Necromancy}
Necromancy is one of the more perverse arts a spellcaster can pursue. It is the art of raising and forcibly commanding the dead. Necromancers can resurrect recently fallen foes and force them to fight by their side for a short time. Or they can resurrect a long-dead pirate king, forcing him to reveal the location of his buried hoard. Necromancers can also use their dark arts to weaken the living in melee range, sapping their life force with nigromantic magics. In order to contact a specific spirit, or to raise any creature bodily from the dead, the spellcaster must have a piece of their body or an object they were very strongly emotionally attached to.

If a spirit is contacted, they will be compelled to answer any three questions asked by the spellcaster, but they may choose to answer additional questions beyond the first three. Spirits are tied to a physical location or object, and if the spellcaster leaves the area or object, the spirit will dissipate and must be re-conjured.
 
If a dead character (human or animal) is resurrected bodily, they will perform a single deed (limited by the mental, physical, and magical characteristics they possessed in life) for a limited amount of time (if they are repeating a simple task or it has end state, such as collecting firewood indefinitely or standing guard) or until completion (if the task has an end state, such as moving a set amount of cargo to a set location or painting a house), after which they will die once more. Human characters that are resurrected again will be hostile to the spellcaster, and must be convinced or coerced to act (perhaps through Mentalism). 

Attempting to cast any Necromancy spells, regardless of success, causes the character to suffer a Green wound.

\textbf{Example Out-of-Combat Necromancy Effects}:
	\begin{itemize}
		\setlength{\itemsep}{0cm}%
  		\setlength{\parskip}{0cm}%
		\item{ \small Contact a nearby random spirit, capable of answering Yes/No questions  [G]}
		\item{ \small Conjure a specific spirit, capable of answering Yes/No questions [Y]}
		\item{ \small Contact a random nearby spirit, capable of answering any question [Y]}
		\item{ \small Resurrect a large natural animal or monster for 1+1d4 hours [R]}
		\item{ \small Resurrect a human for 1+1d4 hours [B]}
	\end{itemize}
	
\textbf{Example In-Combat Necromancy Effects}:
	\begin{itemize}
		\setlength{\itemsep}{0cm}%
  		\setlength{\parskip}{0cm}%
		\item{ \small Cause a nearby skull or other skeletal remains to move about, filling nearby sensible foes with Terror (see the chapter on Magic)  [G]}
		\item{ \small Cause a character to suffer a Yellow wound  [Y]}
		\item{ \small Conjure a black cloud with a 15-foot radius that causes all characters to suffer -2 to their Speed while within and for 1+1d4 turns afterward [Y]}
		\item{ \small Cause a nearby uncontrolled undead or spiritual character to come under your command for 1+1d4 turns [R]}
		\item{ \small Resurrect a human foe that has died in this combat; it will attack the nearest character, whether former friend or foe [B]}
	\end{itemize}
 
\subsubsection{Psychokinesis}
This skill governs a character's ability to move objects with sheer force of will. This skill is unique in that its use in and out of combat is essentially the same. Characters with this skill are theoretically capable of moving objects of any size at any speed, but their ability to do so is limited by their confidence, experience, and ability to visualize such movements. Characters using this ability are not normally capable of fine manipulations, such as engraving jewelry or defusing a bomb (although a Black success can achieve almost anything). Characters with this skill receive a +2 to all checks to attack with a thrown weapon. 

If a character desires to move (or stop) an object or character, they may make a check against Psychokinesis or (Psychokinesis - target's Speed) in the case of a moving target. They can effect a change in the object's momentum, in foot-pounds per second, equal to 50, 100, 200, or 400, whether they achieve a Green, Yellow, Red, or Black success, respectively. The object will maintain movement with their new momentum as long as the character can see the object and maintains concentration on the object, forgoing any other actions. If their concentration is broken forcibly, such as if they suffered a physical blow, the object begins moving as it would naturally, most likely under the effect of physical laws such as gravity. If they chose to end their concentration willingly, they may choose to let the object fall or make another check against Psychokinesis if they wish to hit a specific target. Of course, they must make a check against (Psychokinesis - target's Speed) if they wish to hit a target capable of avoiding the attack. The level of success to hit the target depends on the target's size relative to the projectile being moved, according to the referee's discretion. 

In times of great desire or desperation, a character can achieve a momentum far greater than those indicated above for their level of success, although they will incur a physical toll afterwards (such as a Yellow wound, a temporary stun, or falling unconscious) as adjudicated by the referee.

\subsubsection{Reiki}
Reiki is the skill of manipulating the flow of latent magical energy present in all beings. Characters using this skill must be within melee range in order to affect another character with it. If they wish to heal a character, they must get within melee range of that character and make a check against Reiki. They heal the character of one wound that is of equal or lower magnitude to their level of success (a Yellow success heals a Yellow or Green wound). Alternatively, a Green or higher success can alleviate the negative effects of a Specific Wound for 1+1d4 combat turns. 

In order to damage an enemy using Reiki, a character must first roll to hit the enemy character using their Unarmed Combat skill. They then make a check against their (Reiki - target's Toughness) skill, inflicting a wound to their enemy equal in magnitude to their level of success. 

\subsubsection{Shamanism}
Whereas Necromancy is the skill of dominating the undead and subjugating them to the spellcaster's will, Shamanism is the skill of interacting harmoniously with the undead. That being said, Shamans cannot coerce the dead to act in any way, so their relationships are often those of mutual benefit. They can, however, incapacitate unruly spirits or corporeal undead. They can do the same to controlled undead, but they must make a check against (Shamanism - the Willpower of the undead's controller).

Characters that practice Shamanism are skilled in the art of conferring with the spirits of the dead, and many Shamans often allow spirits with whom they have a deep personal bond to possess their bodies for a short time. Masters of this skill may also enable a spirit to manifest itself physically in their presence. In order for a Shaman to induce these magical effects, they must be holding an object to which a spirit has strong ties, and the spirit itself must allow the spellcaster to do so. 

The spirits that a Shaman interacts with are characters for all intents and purposes. They have skills, and probably quirks and weaknesses as well. When possessed by a spirit, a Shaman receives a bonus to all the skills in which a spirit has a rating (positive or negative), equal to 1/2 the magnitude of that rating. A spirit that is physically manifest has all the skills it possessed in life, and can use them to act on the outside world. A Shaman typically must go on a harrowing quest to find an object to which a spirit has strong ties. The referee (and possibly the player) should take time to flesh out a spirits skills, personality, and desires, as they will likely be the Shaman's close companion.

The effects that can be achieve by making a Shamanism check are similar whether the characters is in or out of combat. In combat, manifesting a spirit has a duration on the order of turns, whereas it lasts on the order of hours out of combat. If a spirit enters or leaves combat, convert the units of time they have left from hours to turns (or vice versa, as appropriate) at a one-to-one ratio. 

\textbf{Example Shamanism Effects}: 
	\begin{itemize}
		\setlength{\itemsep}{0cm}%
  		\setlength{\parskip}{0cm}%
		\item{ \small Detect nearby spirits [G]}
		\item{ \small Speak with nearby spirits [G]}
		\item{ \small Incapacitate undead for 1+1d4 turns [Y]}
		\item{ \small Allow a spirit to possess the caster [R]}
		\item{ \small Allow a spirit to physically manifest for 2+1d6 hours/turns [B]}
	\end{itemize}

\subsubsection{Summoning}
A Summoner is a magician adept at contacting and summoning beings from beyond their own plane of existence. Frequently referred to as \emph{demons}, these unnatural creatures are usually powerful and almost always malign. Drawn from the innumerable realities outside our own, these creatures possess powers as diverse as the caster can imagine. Summoning will require rare, often costly reagents. It is not unusual for an aspiring Summoner to go on a dangerous quest in search of these infernal materials.

A Summoner most often employs their craft when in need of assistance with a specific task. In order to call a creature from beyond the veil, a Summoner should make a check against their Summoning skill and state what task they wish for that creature to accomplish. Most often, something suited for that task will appear. Creatures summoned in this manner, however, are always unwilling to perform the duty asked of them by a summon; in fact, they are often rancorous at the inconvenience of being taken from their home. As such, a Summoner should ensure that it can either placate or destroy whatever they may summon. 

In order to placate a summoned creature, the Summoner should have a gift of some sort. For lesser demons, this may be as simple as a shiny object or the souls of the enemies Summoner's enemies (if summoned to do combat). Demons of greater import, however, are proud, and will require much more of the Summoner. Most often their service will come at great cost to the Summoner -- an eye or the Summoner's soul, for instance. Alternately, they may be interested in exotic or intricate devices, a personal favor from the Summoner, or perhaps a slave that the Summoner has managed to capture. 

Summoned creatures are not of the Summoner's world. As a result, they will inevitably be drawn back to their dominion. A more successful summoning tie a creature to the Summoner's plane for a longer period of time. Additionally, unfortunately for the Summoner, a greater success when summoning will draw a creature of greater power (and pride). Such a baleful art does not come without its many risks. If a Summoner rolls a 96-100 on their Summoning check, whatever they summoned is horribly maddened or angered by the summoning. This creature will immediately attack the Summoner, and will stop at nothing to kill them. It will chase the Summoner tirelessly, utilizing whatever abilities and cunning it has at its disposal, and will only stop when one of them is dead.

\textbf{Example tasks for Summoned creatures}: 
	\begin{itemize}
		\setlength{\itemsep}{0cm}%
  		\setlength{\parskip}{0cm}%
		\item{ \small Shadow someone for the Summoner, spy for a short time, transport the Summoner or an object [G]}
		\item{ \small Fight with natural or manufactured weapons, stand guard for 1+1d4 hours, intimidate those around the Summoner  [Y]}
		\item{ \small Fight with a magical weapon or limited magical ability, teach the Summoner unknown Lore, stand guard for 1+1d4 days, use a Skill for the Summoner at an expert level (a score of 7) [R]}
		\item{ \small Teach the Summoner a new Skill, stand guard for 1+1d4 years, fight with magical skills, disguise as another person for 1+1d4 years, imprison someone eternally  [B]}
	\end{itemize}
	
\section{Creating Skills}
As the setting demands, the referee could create additional skills for characters to take. These skills should function similarly to skills, in that the skill can possess a rating that affects the character's success as a skill. If the referee wishes to add something they think should be a skill, but a "rating" doesn't quite make sense for it, perhaps the addition should be made to be a Quirk or a house-rule. 
\end{multicols}

\chapter{Weaknesses}

This chapter discusses the various weaknesses a character can possess. Weaknesses are divided into three categories: \textbf{General Weaknesses, Mental Weaknesses, and Magical Weaknesses}. General Weaknesses represent weaknesses of the character's body, whereas Mental Weaknesses are quirks of the character's personality or thinking process that adversely affect their interactions with others. Magical Weaknesses are representative of the adverse effects of studying and performing magical arts.

As stated in the chapter on Character Creation, choosing to give your character a Weakness at that time grants you 5 additional Skill Points (per Weakness chosen) to allocate towards your character's starting skills. A player choosing to give their character a weakness for this reason may only choose General or Mental Weaknesses for their character. 

Also discussed in Character Creation are the Weaknesses incurred for possessing Magical Skills. For every Magical Skill a character begins play with (and every Magical Skill they learn thereafter), they must incur a Weakness. These Weaknesses should be chosen from those denoted as Mental or Magical Weaknesses.

The player (and referee) should keep in mind that certain Weaknesses may preclude training in particular Skills, and may prevent their use entirely. The player should also keep in mind that they will be expected to play their character as if they actually possess these Weaknesses. While possibly challenging, this can make for engaging and interesting characters and gameplay.  
\newpage
\begin{multicols}{2}
\section{General Weaknesses}
\subsubsection{Allergy}
The character has a serious and violent allergy to a substance. When choosing this Weakness, the player should name a substance that is common enough for this to truly be a Weakness, but not something that will kill the character immediately. If the character should ever come into close proximity with their allergen, they must test Toughness. On a failure, they suffer the effects of their allergy, depending on the method of contact. Skin contact with their allergen results in an unsightly, painful rash. They take a -3 to all Skills involving the use or movement of that area of their skin for 1d10 hours. For example, an allergic reaction on one's arms would result in a -3 to most Weapon Skills, Trade/Artisan/Profession Skills, etc. If a character were to accidentally swallow their allergen, however, the consequences could be much more dire. If they were to fail their Toughness check in this situation, their eyes will swell shut and their airway will close. Unless they (or another character nearby) can quickly induce vomiting or perform some sort of medical procedure, they will begin suffocating. Please see the Chapter on \textbf{Hazards} for more information about suffocation damage. Even if a character were to prevent their death by suffocation, their puffy eyes give them a -5 to any tasks requiring sight for 1d6 hours.
\subsubsection{Color Blind}
Characters with this Weakness have an inability to distinguish between different colors. This may result in a -3 to using Perception to detect subtle differences in shade (such as when a hidden enemy is wearing camouflage). This may also result in ridicule or ire when the character fails to coordinate the colors of their own outfit. 
\subsubsection{Missing Limb}
The character is missing a limb. While this will certainly hamper the character physically, the player \emph{(and referee)} should keep in mind that this Weakness could have social implications as well. Characters with this Weakness suffer a -5 to all uses of a Skill that requires the use of both arms. While a character missing a limb can learn to do many things that an able-bodied character can do, Dual-Wielding is expressly forbidden for a character missing an arm. A character missing a leg that does not have some sort of prosthetic is considered to have a Speed of -2, regardless of their original rating in that Skill. A character in an appropriate setting with a prosthetic that functions as well as the missing limb is no longer considered to have this Weakness.
\subsubsection{Night-Blind}
A character with this Weakness cannot see very well (or at all) at night. Characters with this Weakness cannot use Navigation to travel at night, and they suffer a -2 to any action requiring sight performed at night. 
\subsubsection{Physically Dependent}
The character requires some sort of substance to function normally, at the player's discretion. Perhaps they have an illness that requires frequent consumption of \emph{Degiik} leaves. Perhaps they have abused Synthehol for too long. Whatever the reason, if the character goes more than three days without consuming this substance, they suffer a -1 to all Skills. Every two days thereafter, this penalty increases by 1. If the character's Toughness skill reaches a rating of 0, they go unconscious. If their Toughness Skill reaches a negative value equal to its original rating, they die. 

If the character begins ingesting this substance again, the penalty is reduced by 2 for each day they consume it, until it has been reduced to 0. If the penalty reached a magnitude over 3, they suffer the excess as a permanent reduction to their Toughness Skill. For example, if the character went 11 days without meeting their needs, they will have suffered a -5 to all Skills. Once they satisfy their itch and reduce the penalty to 0, their Toughness Skill is nonetheless permanently reduced by 2. 

\subsubsection{Reckless}
The character is a person of action. Unwilling or unable to stop and think of the consequences of their actions, they often leap into the maw of danger. Characters with this Weakness cannot pass up their turn in combat if physically able to act, nor can they delay an action. 

\subsubsection{Silent. Possibly Deadly.}
A character with this Weakness is a person of few (if any) words. They are indisposed to or incapable of communicating even with their comrades. Of course, while this may be beneficial if they were to be interrogated, this also means their reticence may prevent them from pointing out any danger they have spotted. 

\subsubsection{Tin Ear}
This character is incapable of distinguishing between musical notes. They cannot possess or even test against any Skills involving the musical arts. Their inability to appreciate music may also prove detrimental in social situations. 

\subsubsection{Unlucky}
A character with this Weakness is possessed of particularly ill fortune. Whenever they make a check against a skill and get a 98-100, something particularly bad happens to them: their pantaloons fall down while addressing a diplomat; a piton slips out of the cliff face they are attempting to scale; their grenade bounces off a wall and rolls toward their feet. 

\subsubsection{Weak Constitution}
This character has a particularly feeble stomach, and they tire easily. When making a Toughness check against poison, they suffer a -3. Additionally, they can Sprint as if their Toughness was 3 ranks lower than it actually is.

\section{Mental Weaknesses}
\subsubsection{Compulsion}
When choosing this Weakness, the character should choose a circumstance that compels a character to perform a specific action. Before leaving the inn, do they open and close the door three times? Perhaps before bedding down outdoors, they must completely bury their cooking fire and all remains of their meal. Whatever they case may be, if an opportunity arises where they feel they must act out their compulsion, and circumstance prevents them from doing so, they suffer a -3 to all Skills for the next 1d6 hours. If the character falls asleep during this period, the penalty continues the rest of its term once they wake. At the referee's discretion, if the character feels their compulsion once again during this period, and they can act on it, they no longer suffer the penalty. 

\subsubsection{Dependent}
The character is dependent upon another character. Their self-worth is measured solely in this other character's opinion of them. If they are asked to act in a manner that could harm that character's opinion of them, they must pass a Willpower check or be unable to do so. If that character should begin to dislike or distrust them, they suffer a -3 to all Skills for 1d8 hours. After this period, they must find a new character upon which to become dependent. Additionally, if the character upon which they are dependent is in a combat with them, the character must make a successful Willpower check or delay their turn until the other character has taken an action. 

\subsubsection{Hatred}
The character has an intense dislike for a specific group. This can be members of another race, spell-casters, a specific species of creature, a political group, or any other group of significant size and distribution. If the character were never to see this group, this wouldn't really be a Weakness. The character will generally steadfastly refuse to help members of this group, and will often seek any opportunity to harm them. If asked to do so by a friend or trusted ally, the character must pass a Willpower check in order to help a member of this group or to stop themselves from harming or outright killing a member of this group they have in their power.
\subsubsection{Obsession}
When taking this Weakness, the player must choose an ideal, a group, an object, or any other thing with which their character can become obsessed, such as a field of knowledge. When presented with an opportunity to help, further the cause of, learn about, or generally interact with their obsession, the character must pass a Willpower check in order to prevent themselves from doing so. If they fail to do so, not even the prospect of harming or betraying their allies will prevent them from feeding their obsession. 
\subsubsection{Phobia}
When choosing this Weakness, the player should choose something of which their character will be terribly frightened. This could be anything their character could encounter frequently enough to impact gameplay. Examples include: magic, water, heights, or gold (particularly bad for a player whose enjoys collecting wealth). When presented with a situation that requires approaching or interacting with an object of their phobia, the character must pass a Willpower check or flee/freeze on the spot, at the player's discretion. If their phobia is something like heights or water, they must pass frequent Willpower checks to continue acting while in or around their phobia, or be frozen in place. If they are in combat with something they fear, even if the character passes their Willpower check, the suffer a -2 to all Skill checks targeting their phobia.

Additionally, the character cannot learn Skills that would require them to interact frequently with their phobia. A fear of heights precludes learning Climbing, for instance. Similarly, a fear of water prevents learning Swimming or Piloting any watercraft. 
\subsubsection{Shifty}
The character lacks certain social cues or acts in a manner that lends them an air of deceit. This could be the result of a poor childhood, past trauma, or years spent shut in a library with only tomes for company. On any NPC Reaction checks, the character's level of Success is being treated as one lower (a Green Success is still a Green Success, however). 
\subsubsection{Tic}
The character has an observable motor or phonic tic. This could be pronounced, repetitive eye-blinking or throat-clearing, finger-snapping, or uttering a short phrase. While occasionally their tic can be suppressed, in times of great stress, nervousness, or happiness, the character must make a Red Success on a Willpower check or suffer an outburst. Of course, such a situation could have social implications for the character. 
\subsubsection{Vice}
When choosing this Weakness, the player should choose a substance or action that the character cannot resist, such as gambling, alcohol, shopping for clothes, or \emph{fraternizing} with members of either gender. Whenever they enter a town after an extended time away, the character must pass a Willpower check or pursue their vice. Of course, if they wish to do so, no check needs to be made.

Once a character has gotten a taste of their vice, if they wish to stop, they must pass a Willpower check in order to do so. Each time they satiate their need (such as each time they finish a drink), they may make a new Willpower check. If the character is financially or physically unable to continue, they may also stop. 

Additionally, when presented or bribed with an opportunity to satisfy their "need," but circumstance leads them to prefer not to do so, the character must pass a Willpower check or take the opportunity. This check need not be made in combat or other situations where it does not make sense for the character to drop everything and go shop for new clothes, for example. On the other hand, this does mean that the character can potentially be caused to act in an abnormal or even detrimental manner. 

\section{Magical Weaknesses}
\subsubsection{Altered Biology}
As a result of arcane experimentation or perhaps some boon from a demonic lord, the character's biology is now alien. The player should choose a mundane, normally inedible substance, such as sand or cloth. The character now receives their sustenance from that substance. As such, this substance should be relatively common. The character no longer has any need to consume normal fare, and in fact it will most likely make them ill. Instead, much to the disgust of their tablemates, they must consume whatever substance the player has chosen. 

\subsubsection{Animal Aversion}
The character evokes skittish or violent responses from mundane animals, and even from humans with particularly high Perception. Any animal (human with Perception of 7 or greater) within 20 feet of the character can sense their \emph{abnormality}. Domestic animals, or any wild creatures that tend to be skittish, will attempt to flee from the character's presence. Predatory creatures, however, may attempt to do the character harm. Characters with this weakness cannot learn the Animal Handling skill, and they cannot use Druidism to interact with mundane creatures, save on a Black success. 

\subsubsection{Disfigured}
The character's body has been rendered misshapen by their arcane dabblings. These changes should be blatant, but can possibly be hidden with enough effort. Examples include immobile supernumerary limbs, clawed hands, discolored or abnormally textured skin, and ornate, over-large tattoos. 

\subsubsection{Haunted}
A character with this Weakness is dogged by the persistent company of unseen presences. The character is always accompanied by whispering voices. These voices seem to have hushed conversations with each other, although they are never quite loud enough to glean meaning from. Characters with this Weakness often find others unwilling to remain alone in a room with them very long, and many who have heard the voices previously find it hard to spend time with the character even in a boisterous tavern hall.  

\subsubsection{Poltergeist}
The character has garnered the attentions of some mischievous force or spirit. Whenever the character packs up after staying in one area for more than a single night, something mundane of theirs (which includes a coinpurse), chosen by the referee, goes missing. The character does not notice that the item is missing until they look for it specifically. At the referee's discretion, the item may reappear some time later when they no longer need it. 

\subsubsection{Restless Sleep}
The character frequently awakes somewhere far from their bed, apparently pursuing some goal which they can no longer recall. Perhaps at the behest of some magical being whose machinations the character cannot possibly comprehend. Whether they remained in their bed the previous night or not, the character feels groggy every morning, taking a -2 to all skills for the first 1d3 hours after waking.  

\subsubsection{Visitation}
The character often finds themselves in the dreams (usually nightmares) of those sleeping in their general vicinity. While amongst civilization, the character's nocturnal wanderings may land them in the dreams of complete strangers, who may often recognize them during their waking hours. Whether their recognition is well-received is determined by the character's actions and perceived intentions in the dream.  
\end{multicols}

\chapter{Quirks}
Quirks are small modifiers to specific Skills that a player can purchase at Character Creation for their character. Quirks are a good way to garner bonuses for your character while adding to their backstory. Quirks often provide situational bonuses or specific specializations for a particular Skill, although some fundamentally alter the character's abilities. Players should keep in mind that any Weaknesses that prevent learning a particular Skill also prevent purchasing Quirks that modify their use of that particular skill. Additionally, purchasing a Quirk that modifies a specific Skill requires that Skill to be known at Character Creation.  

Most Quirks cost two Skill Points. Those that cost a different amount will have their cost listed above their description. 

\newpage
\begin{multicols}{2}
\subsubsection{Berserk}
\textbf{\small Cost: 5 Skill Points}

Except in clear cases of mortal damage (decapitation, magical annihilation, etc), the player may spend a Luck Point to keep fighting beyond the point of death. If the character falls in combat with something they particularly hate (the target of a Hatred Weakness), the player must spend this Luck Point (if they have any remaining). A character in this state receives many bonuses and weaknesses.

The Berserk state confers the following bonuses: The character may continue to ignore the effect of any wounds they incur until the end of the combat. Additionally, they receive a +3 bonus to both Strength checks and Willpower checks for the duration of combat. The character also receives a +1 bonus to melee attack rolls for every 5 full points in Strength, as opposed to the usual 10. Lastly, the character can ignore any Phobias they have.

The Berserk state has the following drawbacks: The character must focus their attention on the target of any Hatred Weaknesses they have, ignoring other enemies until their Hated targets are dead. If they Hate multiple targets in the combat, they must attack the closest first. Additionally, the character remains in a blood-trance continues once all enemies have been vanquished. The closest living creature in sight becomes their next target, continuing until the character can no longer see a living creature, at which point they succumb to their wounds. 

\subsubsection{Celestial Guidance}
The character is an expert at Navigating using either the Sun or the stars. Choose either day or night. The character receives a +3 to their Navigation score during the chosen time period. The character does not receive this benefit if they cannot see the sky, often due to weather conditions or simply being indoors.

\subsubsection{Exotic Accent}
The character speaks with an accent that no one can quite guess. A character with this Quirk receives no penalties or bonuses when interacting with NPCs due to their homeland. An NPC interested in undiscovered or unknown lands may be more inclined to interact with the character, however. 

\subsubsection{Expert Tracker}
When choosing this Quirk, the player should choose between mundane animals, magical creatures, and human beings. For the purposes of Tracking the chosen type of creature, the character may treat the trail as if it were 3 hours more fresh than it actually were. This does not allow for a trail to be treated as if it had a negative age.

\subsubsection{Forgiving Fans}
This Quirk cannot be taken by those that also possess the Polarizing Personality Quirk.

Any time a character with this Quirk would lose a point of Fame, they may make a check against their current Fame value. On a Black success, they don't lose any Fame. 

\subsubsection{Iron Constitution}
Characters with this Quirk may increase the time they have to find the antidote for a poison by 25\%, rounding down to the nearest quarter-unit of time. For example, a poison that takes effect in 5 minutes now takes effect in 6.25 minutes, while a poison that takes effect in 3 hours now takes effect in 3 hours and 45 minutes. This Quirk has no effect on the wound inflicted by the poison. See the chapter entitled \textbf{Hazards} in order to find out more about Poisons. 

\subsubsection{Light Touch}
A character with this Quirk has particularly nimble fingers. Their attempts at picking pockets are only detected on a result of 98-100, rather than 96-100.     

\subsubsection{Luck Everlasting}
Any time the player chooses to spend a Luck Point, the referee must make a roll against that character's current Luck Point total. On a Black success, the Luck Point has its intended effect, but remains unspent. The result of this check should not be revealed. 

\subsubsection{Marathon Runner}
The character can jog for three more hours or sprint for three more minutes than their Toughness or General Physical rating would allow. 

\subsubsection{Military Background}
Characters with this Quirk must have a rating of at least 1 in the Weapon Skill for the standard-issue weapon of their homeland's military. 

Past time spent in the military has taught the character a variety of useful things. Not least of which being how to belittle subordinates. The character gets a +3 bonus to using Fast Talk or Persuade while interacting with non-hostile, rank-and-file members of a militia or military. 

\subsubsection{Polarizing Personality}
\textbf{\small Cost: 5 Skill Points}

The character's strong force of will and charismatic personality ensures that anyone they interact with quickly develops a very strong opinion of them, whether it be positive or negative; rare is the person who feels ambivalent about the character. Not every NPC is quick to reveal their personal feelings about the character, especially if they are particularly strong the character may often discover a an unknown ally or a secret foe. In the case of someone with the "Void-touched" weakness, all creatures that would normally dislike the character due to their unusual properties (magical and magic-using creatures) now feel an extreme hatred for them.

Additionally, whenever the character gains fame, roll against their new fame point total. On a black success, they gain an additional point of fame. A gain of fame in this manner does not allow the player to roll for another point of fame.

\subsubsection{Quick Study}
\textbf{\small Cost: 5 Skill Points}

The amount of time required to study any subject (usually for the purposes of gaining a new Skill), is reduced by 1/3rd. This has no effect on how many Skills you may study at the same time, nor does it affect the cost or penalties associated with learning a new Skill. See the chapter on \textbf{Character Advancement and Time Management} to find out more about studying new Skills.

\subsubsection{Sleepless}
\textbf{\small Cost: 5 Skill Points}
The character needs very little sleep on most nights. Normally, the character needs only 2 hours of sleep to feel fully rested. Even if they desired to sleep longer, they usually find themselves unable to do so. At the start of the campaign, and at the start of any session before which a significant amount of time was skipped, the referee should roll 1d8+4. This roll need not be revealed to the player. On the night after this number of days has passed, the character's lack of sleep seems to catch up with them, and they sleep for a full 8 hours. While sleeping on this particular night, the character is unusually difficult to rouse. At the referee's discretion, the player may spend a Luck Point to rouse from their slumber. After whatever event that caused them to wake has passed, however, they must resume their sleep.

\subsubsection{Street-Smarts}
A childhood spent in an urban environment has given the character a keen eye for danger and an ear for interesting information. The character receives a +3 to Perception Skill checks while in a city environment. 

\subsubsection{Trained for War}
A character wishing to choose this Quirk must have a rating of least 1 in a Weapon Skill at Character Creation.

The character gets a +2 column shift to their General Physical Rating when using a weapon in the same family as their starting weapon for which they are not trained. This score cannot exceed the highest score the character has for a specific weapon in the chosen weapon family. A family of weapons is a category such as flexible weapons (whips, flails, etc); 1-handed blades; 2-handed blunt weapons (greathammers); pole weapons (lance, spear, guisarme, etc).

\subsubsection{Triage Specialist}
The character is a practiced hand at treating combat wounds. Perhaps they were a medic in their nation's army or a surgeon in a war-torn region. Nonetheless, they gain a +3 bonus to First-Aid Skill checks or other Skill checks used to treat wounds that resulted from combat. This grants no bonus to healing wounds inflicted by the environment (fire, falling, nettles, etc) that were taken in the course of combat. However, this will grant a bonus to healing wounds inflicted by caltrops or other devices that are generally used to wound enemy combatants.

\subsubsection{Void-Touched}
\textbf{\small Cost: 5 Skill Points}

A character with Quirk is considered by most to be a particularly pitiful creature. The character, by some twist of fate, is completely immune to the effects of magic, both helpful and harmful. Additionally, all characters within 20 feet of this creature receive a -2 to all Magical Skill checks used to manifest magical effects. Finally, a character with this quirk cannot learn to use any Magic Skills.

Additionally, all creatures can sense that the character is, in some way, subtly \emph{different} than those around them. Most characters will treat the character more gruffly than they would a normal stranger. Magically-attuned creatures, however, experience a strong feeling of disgust when in the presence Void-Touched characters, as they are an antithesis of everything they believe in. Creatures that have points in the Reiki or Magic Sense Skills, or any creature with a General Magic Rating of 2 or greater fall into the previous category. Demons, spirits, and any other creatures that the referee deems appropriate may also fall into the above category.
\end{multicols}

\chapter{The Resolution Chart}
This chapter concerns the method by which conflicts are resolved in THIZ. These conflicts are resolved with dice rolls, and the results of those rolls are found by consulting the Resolution Chart. Depicted at the end of this book in many forms, this chart dictates how successful any character is whenever they use a Skill. The player makes a check by rolling a 1d100 and consulting the proper row of the resolution chart.

\section{Reading the Resolution Chart}
The resolution chart consists of multiple rows and six columns. The leftmost column, which contains a white background or a white header, denotes the rating against which a check is being made. For example, if a character is attempting to swim through rough seas and has a rating of 4 in the Swimming skill, the player will consult the row whose left-most column contains a '4' when making a check. Once the proper row has been consulted, the player can then determine the level of success. This is represented by the columns whose background colors or headers are black, red, yellow, green, or grey. These colors represent black, red, yellow, or green success and failure, respectively. 

There are many variations of the resolution chart depicted in the back of the book. They represent the roll required to attain a certain level of success in two different ways. Some of the charts contain a range of numbers in their columns, whereas some contain a single number. The charts containing a range of numbers depict the numbers between which your d100 roll must fall in order to achieve that level of success. On the charts with a single number, your level of success is that of the column which depicts the smallest number greater than your roll. 

For example, in the Swimming example above, if the player rolled a 40, their level of success (with a rating of 4), would be Yellow.

\chapter{Combat}
\begin{multicols}{2}
Combat may be the meat and potatoes of your group's tabletop adventures, or it may be the occasional, unfortunate side effect of poor roleplay. Either way, there inevitably comes a time when your characters must test their mettle and cross swords (figuratively or literally) with their foes. 

\section{Get the Mooks out of the Way!}
In the world of THIZ, players will encounter two types of enemies: Mooks and Adversaries. 

Mooks are the chaff which your characters must clear on their way to the adversaries. Hired thugs, guardsmen, militia-men, sailors, and other such fighters who certainly die by the sword, but don't quite live like it, as your characters do. 

Adversaries are characters that are on par with those of the players. Armed with only your Skills and your wits, you will almost certainly square off against adversaries during your adventures. Unlike combat with Mooks, meeting an Adversary in pitched battle is not a guaranteed victory. 

\section{How Combat Works}
Combat in THIZ is composed of the following discrete components:
\begin{enumerate}
  \setlength{\itemsep}{0cm}%
  \setlength{\parskip}{0cm}%
  \item Determining surprise
  \item Determining initiative
  \item Declare and resolve actions in initiative passes
  \item Repeat 2 \& 3 until combat is over
\end{enumerate}

\section{Determine Surprise}
In most cases, some combatants are more prepared for combat than others. Some fighters may be caught completely unawares, or perhaps were distracted whilst a nearby argument escalated into a brawl. Whatever the case may be, all fighters begin combat by checking against Speed + Perception. Depending on the result of this check, each combatant gets 0 through 4 points, for Failure through Black success, respectively. This value is their \textbf{Preparedness}.  

Whichever side has the combatant with the highest Preparedness may take a full Initiative Pass of actions against their enemies. In the case of a tie, compare the total Preparedness of either party. If there is still a tie, have the characters with the highest Preparedness check Preparedness again, until the tie is resolved.

In some special cases, such as an ambush, the referee may declare that one side is unprepared, in which case the other side takes a full Initiative Pass of actions against the other. 

If a character enters combat after it has already began, they must declare an intended target. The character and their intended target then determine their Preparedness. The character with the highest preparedness may take a free action against the other. The newly arrived character may afterward take part in regular combat, starting with the following Initiative Pass.

\section{Determining Initiative and Initiative Passes}
In order to determine Initiative, players roll 1d10 and add either Speed + Perception or their General Physical rating, whichever is higher.

Once all characters have determined their Initiative, they each take a 
\textbf{turn} in initiative order. A turn in combat lasts 5 seconds, and permits a character to take a single action. Possible actions are outlined below, in the section entitled \textbf{Actions}, but typical actions include: moving, attacking, or casting a spell. 

Once all characters in the combat have acted in initiative order, they subtract 10 from their Initiative score. If any characters have an initiative above 0, they may act again, in order of their new Initiative. This process repeats until no characters have an initiative  above 0. This cycle, beginning with all characters checking Initiative and ending with all characters possessing an Initiative of 0 or lower, is called an \textbf{Initiative Pass}. 

If combat is not resolved (i.e. there are combatants alive and willing to fight on both sides), then all characters determine Initiative again and the process starts over. 

\section{Actions}
There are six general actions that a character can take on their turn during combat: Move, Attack, Defend, Use Magic, Use Maneuvers, or Wait. Below follows a description of these actions and how they work. 

\subsubsection{Move}
A character can move using whatever means are available to them and appropriate to the circumstances. The distance they can move is dictated by the character's Speed rating. See the Chapter on Movement for more information.

\subsubsection{Attack}
Attacks are actions that attempt to harm the target in some way. There are two types of attacks: Missile Attacks, and Melee Attacks.

\textbf{Melee attacks} are any attacks made with a character's melee weapon or fists. When making a Melee Attack, the character must be within 5 feet of the target. The attacking player must declare a body location they wish to target: Head, Gut, Chest, Right or Left Arm, or Right or Left Leg. The sections below entitled \textbf{Determining a Hit} and \textbf{Damage} describe how to resolve the intent of a melee attack. 

A \textbf{Missile Attack} is slightly more complex. This is any attack made with a ranged weapon of some kind: bow, crossbow, gun, thrown weapon, etc. In order to make a Missile Attack, the character must be  out of the melee range of any enemy combatants, and within the range of their ranged weapon. Unlike with melee attacks, the body location targeted by a missile attack is normally determined randomly by rolling 1d100 and consulting the table below. Alternately, the player can choose to declare the body location they will be targeting. In this case, the player must get a Black success when determining their hit, or else the attack misses. As with melee attacks, see the sections "Determining a Hit" and "Damage" to see how to resolve the intent of a missile attack. 


\begin{center}
\begin{tabular}{|l|l|}
\hline
\rowcolor[HTML]{9B9B9B} 
\textbf{Die Roll} & \textbf{Location} \\ \hline
01-10             & Left Leg          \\ \hline
\rowcolor[HTML]{EFEFEF} 
11-20             & Right Leg         \\ \hline
21-45             & Gut               \\ \hline
\rowcolor[HTML]{EFEFEF} 
46-70             & Chest             \\ \hline
71-80             & Left Arm          \\ \hline
\rowcolor[HTML]{EFEFEF} 
81-90             & Right Arm         \\ \hline
90-100            & Head              \\ \hline
\end{tabular}
\end{center}

\subsubsection{Defend}
When a character chooses to defend, they add their Weapon Skill (for the weapon they have equipped) to their Speed for the first attack made against them before their next turn. 

\subsubsection{Use Magic}
This action is taken when a character attempts to manifest a magical effect using one of their Magic Skills. See the chapter entitled \textbf{Magic} to learn more about using magic in combat.

\subsubsection{Use Maneuvers}
Maneuvers are actions such as Bull Rush, Disarm, Trip, or other actions that affect a target without damaging them. Individual Maneuvers are discussed later in this chapter. 

\subsubsection{Wait}
A character can choose to wait on their turn. If they do so, they must state an action they wish to perform and a condition that will cause them to act. If that condition occurs before their next turn, they may perform their action. If they wish to act before the trigger occurs, they must check Speed. On a success, they may act before the triggering condition occurs. Otherwise, they act after.

\section{Determining a Hit}
In order to determine a hit with an attack, you simply make a resolution check against (your Weapon Skill with your equipped weapon or General Physical rating + that weapon's Attack Bonus, if any) - (your target's Speed or General Physical rating + your target's Armor Bonus on the targeted body region, if any). On any level of success, you have hit your target. For example, an attacker armed with a steel sword and with a Sword Skill rating of 3 attacks a target with Speed 4 and bronze armor. He checks against an adjusted rating of (3 [his Sword Skill] + 4 [his sword's Attack Bonus]) - (4 [his opponent's Speed + 1 [his opponent's armor bonus]), which equals 2. The player rolls a d100 and checks the resolution chart for a rating of 2.

See the section below, "Damage," to find out more about how this hit can affect your target. The section \textbf{Combat Equipment} discusses Attack Bonuses and Armor Bonuses. 

\section{Damage}
Characters in THIZ have seven body sections, as discussed above: Head, Chest, Gut, Right and Left Arm, and Right and Left Leg. Each of these body sections can be targeted, and each of them can receive wounds. Whenever a body section has been successfully hit by an attacker, the character receiving the hit makes a check against their Toughness rating. If their level of success exceeds that of the hit, they receive no wound. If their level of success matches that of the hit, the level of success of the hit is reduced by one. 

Depending on the level of success of the hit, the character receives a Green, Yellow, Red or Black Wound to the stricken body location, for a Green through Black Success, respectively. A character can receive any number of Green Wounds, but only one each of the other types of wounds. If they receive a second wound of the same color to the same body location, it is stepped up to the next color of wound. 

The effect of wounds on particular body locations is outlined in the charts below:

\begin{center}
\scalebox{0.7}{
	\begin{tabular}{l|c|c|}
\cline{2-3}
 & G/Y/R & Black \\ \hline
\multicolumn{1}{|l|}{Head} & -1/2/4 to all Skills & Incapacitated \\ \hline
\multicolumn{1}{|l|}{Chest} & -1/2/4 to Initiative & Incapacitated \\ \hline
\multicolumn{1}{|l|}{Gut} & -1/2/4 Toughness & Incapacitated \\ \hline
\multicolumn{1}{|l|}{Arm} & -1/2/4 to Skills with arm & Useless \\ \hline
\multicolumn{1}{|l|}{Leg} & -1/2/4 Speed & Useless \\ \hline
\end{tabular}
}
\end{center}

In the case of non-lethal damage (attacker is using bare fists), an Incapacitated character is simply unconscious (they cannot act, Speed is 0). In the case of lethal damage, they are dying. They must check Toughness. On a failure, they die. On a Black success, they need not check Toughness any more. A character can attempt to use First Aid to stabilize an Incapacitated character, which they do on any level of success. Also in the case of lethal damage, a limb with a Black success can be removed, at the referee's discretion. 

\section{Combat Equipment}
This book attempts to separate the setting and the mechanics as much as possible. As such, rather than listing specific weapon or armor types, specific materials, and specific costs, this book leaves those to the referee and the players to determine as appropriate for their setting. The book does, however, provide a suggestion for comparing the efficacy of weapons and armor, relative to each other. 

Weapons and armor are made of materials, and these materials have relative hardness and resistances relative to each other. As such, it should be easy and intuitive for the referee to identify four or five materials that are common to their setting and rank them from 1 to 4 (or 5), on a scale of increasing hardness. The value assigned to a specific material is the Attack or Armor bonus for a weapon or a piece of armor made with that material, respectively. In addition to an Armor Bonus, pieces of Armor also cover one or more body locations. Whatever design or style the armor has is up to the referee and their setting. 

For example, a futuristic setting could commonly feature objects made of Carbon Fiber, Steel, Titanium, and Plasteel. If the referee assigned these materials a hardness of 1, 2, 3, and 4, respectively, then a sword made of Titanium would have an Attack Bonus of 3. Alternately, in a medieval setting featuring Leather, Bronze, Iron, and Steel materials, each with a hardness of 1, 2, 3, or 4, respectively, a piece of bronze armor would have an Armor Bonus of 2. 

If the setting allows for the use of ranged weapons of any sort, the referee may declare Attack Bonuses for these weapons as well. If the weapon is low-velocity, such as a bow, the Attack Bonus is 1, so long as the missiles are made of a material with a hardness of at least 1. If the missiles are low-velocity but designed to pierce armor, such as a bodkin point arrow, the Attack Bonus is 2. High velocity missiles (bullets fired from a gun or faster), both traditional and armor-piercing, have an Attack Bonus of 3 and 4 respectively. 

In special circumstances, such as weapons or armor made of a very rare material, or magical/technological weapons that aren't technically made of any material, it is up to the referee's discretion to declare an Attack or Armor Bonus, relative to the traditional materials of the setting. 

\section{Unarmed Combat}
Characters fighting without melee weapons, using their bare fists, can only damage armored Adversaries if their Strength rating is triple that of the Adversary's Armor Bonus. A character uses their Unarmed Combat Skill to make attacks such as these; this Skill can also be used if the character is equipped with certain weapons, such as a cestus or knuckledusters. If using one of these weapons or something similar, and the Attack Bonus of that weapon is 1 or higher, the character can damage armored Adversaries regardless of their Strength rating.  

\section{Two-Weapon Fighting}
A character fighting with two weapons must have the Dual-Wield skill for the two weapons they wish to attack with, or suffer penalties. The character may check their Weapon skill normally to attack with their main-hand weapon, but they must check the lower of their Weapon Skill or Dual-Wield rating to attack with their second weapon. The attack made with their second weapon is resolved after the turn of the character with the lowest initiative. 

A character attempting to attack with two weapons without the appropriate Dual-Wield skill suffers a -3 to attacking with both weapons this turn, as well as a -2 to their Speed rating until the end of the next Initiative Pass. 

\section{Maneuvers}
Maneuvers are special actions that can be performed in Combat whose primary intent are not to cause harm. 

\subsubsection{Grappling}
A character can attempt to grapple another character whether or not they have the Grappling skill. Of course, having training will increase a character's chance of success. In order to grapple another character, the character must make a check against their Grappling or General Physical rating, minus the Speed or General Physical rating of their target. This check is referred to as a Grappling check. 

On a success, they are grappling the target. On subsequent turns, they can make another Grappling check and, on a success, can continue grappling the target and make a free Unarmed Combat attack, throw the target 5 feet per 5 full points of Strength, or pin the target to the floor. A thrown target always lands in the Prone position.

A character that begins their turn being Grappled can either make a Grappling check to reverse the grapple or a Strength check to break free. In either case, the intended effect only occurs if their level of success exceeds that of the character that Grappled them. 

\subsubsection{Tripping}
A character can attempt to trip another character, making them drop to the floor in the Prone position. The character must make a check against their Weapon skill or General Physical rating minus the target's Speed or General Physical rating. Particular weapons or objects grant a bonus to tripping. These include flexible weapons like whips and weapons with hooks, such as guisarmes.

\subsubsection{Dropping Prone}
A character may find themselves in the Prone position as a result of the actions of another character or environmental effects, or they may choose to drop prone themselves. Moving to or from the Prone position is an action that takes up a single turn. Characters attempting to hit a character in the Prone position with a missile attack receive a -3 penalty. Characters who attempt to hit a character in the Prone position with a melee weapon, however, receive a +3 bonus. 

\subsubsection{Disarming}
A character can attempt to knock a weapon from another character's grasp, if they are in melee range with their target. To do so, they check against their Weapon skill or General Physical rating minus their target's Weapon skill or General Physical rating. On any level of success, the weapon drops to the ground at their target's feet. On a Black success, the weapon drops to the ground at most 10 feet away from the target, in a random direction. Flexible weapons such as whips and flails grant a +3 bonus to attempts to disarm a target.

\subsubsection{Sundering}
A character attempting to sunder the equipment of a character must declare the piece of equipment they intend to sunder (a weapon or the armor of a body location). A sunder attempt also requires two rolls: the first is a check to see if the character hits the intended target, and the second is to determine if they were able to damage it. The first check is made against the character's Weapon skill or General Physical rating minus the Speed or General Physical rating of the target. The second check is against the character's Strength or General Physical rating. 

If both of these checks have any level of success, the target equipment is damaged, at least to some extent. The target piece of equipment's Attack Bonus or Armor Bonus (as appropriate) is reduced by the Attack Bonus of the character's weapon. A piece of equipment whose Bonus (Attack or Armor) has been reduced to 0 is destroyed. On a Black success on the Strength check inflicts a Green wound on the body location nearest the target piece of equipment.  

For example: Gala of Iondren uses her trusty steel lucerne hammer to attempt to sunder the bronze helmet of a would-be assassin. She successfully hits his helmet and gets a Green success on her Strength check. Her weapon, being made of steel, has an Attack Bonus of 3, which reduces the bronze helmet's Armor Bonus of 1 to -2. As its Armor Bonus is reduced to 0 or lower, the helmet is destroyed. 

\section{Burst/Full-Auto Fire}
In a setting where weapons are capable of burst or fully-automatic fire, the character can score multiple hits against their target. If they attack using burst or full-auto fire and wound their target at any level above Green, they also give the target a wound of each color below that. For example, if a character fires a burst of 5 shots and scores a Red wound on their target, they also score a Yellow and Green wound on that target. The player must roll on the missile attack chart for each of these wounds to determine which body location they effect. If a character's firearm is only capable of a three-round burst and they score a Black wound, they only inflict a Black, Red, and Yellow wound on the target. 

\end{multicols}


\appendix
\chapter{Resolution Charts}
This appendix contains multiple renditions of the resolution chart, as there has been no established consensus as to which is best. I consider the chart contained in ZeFRS, by which this set of rules was inspired, too large and hard to read. The chapter on \textbf{The Resolution Chart} contains guidelines for reading the various iterations of the resolution chart, as well as how to determine the success of a Skill check using the resolution chart. 


\begin{table}[h]
\begin{tabular}{
>{\columncolor[HTML]{FFFFFF}}l 
>{\columncolor[HTML]{000000}}c 
>{\columncolor[HTML]{FE0000}}c 
>{\columncolor[HTML]{F8FF00}}c 
>{\columncolor[HTML]{34FF34}}c 
>{\columncolor[HTML]{EFEFEF}}c }
{\color[HTML]{000000} \textbf{-26 to -30}} & {\color[HTML]{FFFFFF} \textbf{1 .. 4}}  & {\color[HTML]{FFFFFF}\textbf{5 .. 10}}  & \textbf{11 .. 18} & \textbf{19 .. 30} & \textbf{31 .. 100} \\
\textbf{-21 to -25}                        & {\color[HTML]{FFFFFF} \textbf{1 .. 4}}  & {\color[HTML]{FFFFFF}\textbf{5 .. 10}}  & \textbf{11 .. 19} & \textbf{20 .. 31} & \textbf{32 .. 100} \\
\textbf{-16 to -20}                        & {\color[HTML]{FFFFFF} \textbf{1 .. 4}}  & {\color[HTML]{FFFFFF}\textbf{5 .. 11}}  & \textbf{12 .. 20} & \textbf{21 .. 32} & \textbf{33 .. 100} \\
\textbf{-11 to -15}                        & {\color[HTML]{FFFFFF} \textbf{1 .. 4}}  & {\color[HTML]{FFFFFF}\textbf{5 .. 11}}  & \textbf{12 .. 21} & \textbf{22 .. 33} & \textbf{34 .. 100} \\
\textbf{-10}                               & {\color[HTML]{FFFFFF} \textbf{1 .. 4}}  & {\color[HTML]{FFFFFF}\textbf{5 .. 12}}  & \textbf{13 .. 22} & \textbf{23 .. 34} & \textbf{35 .. 100} \\
\textbf{-9}                                & {\color[HTML]{FFFFFF} \textbf{1 .. 5}}  & {\color[HTML]{FFFFFF}\textbf{6 .. 12}}  & \textbf{13 .. 23} & \textbf{24 .. 36} & \textbf{37 .. 100} \\
\textbf{-8}                                & {\color[HTML]{FFFFFF} \textbf{1 .. 5}}  & {\color[HTML]{FFFFFF}\textbf{6 .. 13}}  & \textbf{14 .. 24} & \textbf{25 .. 38} & \textbf{39 .. 100} \\
\textbf{-7}                                & {\color[HTML]{FFFFFF} \textbf{1 .. 5}}  & {\color[HTML]{FFFFFF}\textbf{6 .. 13}}  & \textbf{14 .. 25} & \textbf{26 .. 40} & \textbf{41 .. 100} \\
\textbf{-6}                                & {\color[HTML]{FFFFFF} \textbf{1 .. 6}}  & {\color[HTML]{FFFFFF}\textbf{7 .. 14}}  & \textbf{15 .. 26} & \textbf{27 .. 42} & \textbf{43 .. 100} \\
\textbf{-5}                                & {\color[HTML]{FFFFFF} \textbf{1 .. 6}}   & {\color[HTML]{FFFFFF}\textbf{7 .. 14}}  & \textbf{15 .. 27} & \textbf{28 .. 44} & \textbf{45 .. 100} \\
\textbf{-4}                                & {\color[HTML]{FFFFFF} \textbf{1 .. 6}}   & {\color[HTML]{FFFFFF}\textbf{7 .. 15}}  & \textbf{16 .. 28} & \textbf{29 .. 46} & \textbf{47 .. 100} \\
\textbf{-3}                                & {\color[HTML]{FFFFFF} \textbf{1 .. 6}}   & {\color[HTML]{FFFFFF}\textbf{7 .. 15}}  & \textbf{16 .. 29} & \textbf{30 .. 48} & \textbf{49 .. 100} \\
\textbf{-2}                                & {\color[HTML]{FFFFFF} \textbf{1 .. 7}}  & {\color[HTML]{FFFFFF}\textbf{8 .. 16}}  & \textbf{17 .. 30} & \textbf{31 .. 50} & \textbf{51 .. 100} \\
\textbf{-1}                                & {\color[HTML]{FFFFFF} \textbf{1 .. 7}}  & {\color[HTML]{FFFFFF}\textbf{8 .. 16}}  & \textbf{17 .. 31} & \textbf{32 .. 52} & \textbf{53 .. 100} \\
\textbf{0}                                 & {\color[HTML]{FFFFFF} \textbf{1 .. 7}}  & {\color[HTML]{FFFFFF}\textbf{8 .. 17}}  & \textbf{18 .. 33} & \textbf{34 .. 54} & \textbf{55 .. 100} \\
\textbf{1}                                 & {\color[HTML]{FFFFFF} \textbf{1 .. 7}}  & {\color[HTML]{FFFFFF}\textbf{8 .. 18}}  & \textbf{19 .. 35} & \textbf{36 .. 57} & \textbf{58 .. 100} \\
\textbf{2}                                 & {\color[HTML]{FFFFFF} \textbf{1 .. 8}}  & {\color[HTML]{FFFFFF}\textbf{9 .. 19}}  & \textbf{20 .. 37} & \textbf{38 .. 60} & \textbf{59 .. 100} \\
\textbf{3}                                 & {\color[HTML]{FFFFFF} \textbf{1 .. 8}}  & {\color[HTML]{FFFFFF}\textbf{9 .. 20}}  & \textbf{21 .. 39} & \textbf{40 .. 63} & \textbf{64 .. 100} \\
\textbf{4}                                 & {\color[HTML]{FFFFFF} \textbf{1 .. 9}}  & {\color[HTML]{FFFFFF}\textbf{10 .. 21}} & \textbf{22 .. 41} & \textbf{41 .. 66} & \textbf{67 .. 100} \\
\textbf{5}                                 & {\color[HTML]{FFFFFF} \textbf{1 .. 9}}  & {\color[HTML]{FFFFFF}\textbf{10 .. 22}} & \textbf{23 .. 43} & \textbf{44 .. 69} & \textbf{70 .. 100} \\
\textbf{6}                                 & {\color[HTML]{FFFFFF} \textbf{1 .. 10}} & {\color[HTML]{FFFFFF}\textbf{11 .. 23}} & \textbf{24 .. 45} & \textbf{46 .. 72} & \textbf{73 .. 100} \\
\textbf{7}                                 & {\color[HTML]{FFFFFF} \textbf{1 .. 10}} & {\color[HTML]{FFFFFF}\textbf{11 .. 25}} & \textbf{26 .. 47} & \textbf{48 .. 75} & \textbf{76 .. 100} \\
\textbf{8}                                 & {\color[HTML]{FFFFFF} \textbf{1 .. 11}} & {\color[HTML]{FFFFFF}\textbf{12 .. 27}} & \textbf{28 .. 49} & \textbf{50 .. 78} & \textbf{79 .. 100} \\
\textbf{9}                                 & {\color[HTML]{FFFFFF} \textbf{1 .. 12}} & {\color[HTML]{FFFFFF}\textbf{13 .. 29}} & \textbf{30 .. 51} & \textbf{52 .. 81} & \textbf{82 .. 100} \\
\textbf{10}                                & {\color[HTML]{FFFFFF} \textbf{1 .. 13}} & {\color[HTML]{FFFFFF}\textbf{14 .. 31}} & \textbf{32 .. 53} & \textbf{54 .. 84} & \textbf{85 .. 100} \\
\textbf{11 to 15}                          & {\color[HTML]{FFFFFF} \textbf{1 .. 14}} & {\color[HTML]{FFFFFF}\textbf{15 .. 33}} & \textbf{34 .. 56} & \textbf{57 .. 88} & \textbf{89 .. 100} \\
\textbf{16 to 20}                          & {\color[HTML]{FFFFFF} \textbf{1 .. 15}} & {\color[HTML]{FFFFFF}\textbf{16 .. 35}} & \textbf{36 .. 59} & \textbf{60 .. 92} & \textbf{93 .. 100} \\
\textbf{21 to 25}                          & {\color[HTML]{FFFFFF} \textbf{1 .. 16}} & {\color[HTML]{FFFFFF}\textbf{17 .. 37}} & \textbf{38 .. 62} & \textbf{63 .. 96} & \textbf{97 .. 100} \\
\textbf{26 to 30}                          & {\color[HTML]{FFFFFF} \textbf{1 .. 18}} & {\color[HTML]{FFFFFF}\textbf{19 .. 40}} & \textbf{41 .. 66} & \textbf{67 .. 96} & \textbf{97 .. 100}
\end{tabular}
\end{table}


\newpage

\begin{table}[h]
\begin{tabular}{
>{\columncolor[HTML]{FFFFFF}}l 
>{\columncolor[HTML]{000000}}c 
>{\columncolor[HTML]{FE0000}}c 
>{\columncolor[HTML]{F8FF00}}c 
>{\columncolor[HTML]{34FF34}}c 
>{\columncolor[HTML]{EFEFEF}}c }
{\color[HTML]{000000} \textbf{-26 to -30}} & {\color[HTML]{FFFFFF} \textbf{1 .. 4}}  & {\color[HTML]{FFFFFF}\textbf{5 .. 10}}  & \textbf{11 .. 18} & \textbf{19 .. 30} & \textbf{31 .. 100} \\	\hline
\textbf{-21 to -25}                        & {\color[HTML]{FFFFFF} \textbf{1 .. 4}}  & {\color[HTML]{FFFFFF}\textbf{5 .. 10}}  & \textbf{11 .. 19} & \textbf{20 .. 31} & \textbf{32 .. 100} \\	\hline
\textbf{-16 to -20}                        & {\color[HTML]{FFFFFF} \textbf{1 .. 4}}  & {\color[HTML]{FFFFFF}\textbf{5 .. 11}}  & \textbf{12 .. 20} & \textbf{21 .. 32} & \textbf{33 .. 100} \\	\hline
\textbf{-11 to -15}                        & {\color[HTML]{FFFFFF} \textbf{1 .. 4}}  & {\color[HTML]{FFFFFF}\textbf{5 .. 11}}  & \textbf{12 .. 21} & \textbf{22 .. 33} & \textbf{34 .. 100} \\	\hline
\textbf{-10}                               & {\color[HTML]{FFFFFF} \textbf{1 .. 4}}  & {\color[HTML]{FFFFFF}\textbf{5 .. 12}}  & \textbf{13 .. 22} & \textbf{23 .. 34} & \textbf{35 .. 100} \\	\hline
\textbf{-9}                                & {\color[HTML]{FFFFFF} \textbf{1 .. 5}}  & {\color[HTML]{FFFFFF}\textbf{6 .. 12}}  & \textbf{13 .. 23} & \textbf{24 .. 36} & \textbf{37 .. 100} \\	\hline
\textbf{-8}                                & {\color[HTML]{FFFFFF} \textbf{1 .. 5}}  & {\color[HTML]{FFFFFF}\textbf{6 .. 13}}  & \textbf{14 .. 24} & \textbf{25 .. 38} & \textbf{39 .. 100} \\	\hline
\textbf{-7}                                & {\color[HTML]{FFFFFF} \textbf{1 .. 5}}  & {\color[HTML]{FFFFFF}\textbf{6 .. 13}}  & \textbf{14 .. 25} & \textbf{26 .. 40} & \textbf{41 .. 100} \\	\hline
\textbf{-6}                                & {\color[HTML]{FFFFFF} \textbf{1 .. 6}}  & {\color[HTML]{FFFFFF}\textbf{7 .. 14}}  & \textbf{15 .. 26} & \textbf{27 .. 42} & \textbf{43 .. 100} \\	\hline
\textbf{-5}                                & {\color[HTML]{FFFFFF} \textbf{1 .. 6}}   & {\color[HTML]{FFFFFF}\textbf{7 .. 14}}  & \textbf{15 .. 27} & \textbf{28 .. 44} & \textbf{45 .. 100} \\	\hline
\textbf{-4}                                & {\color[HTML]{FFFFFF} \textbf{1 .. 6}}   & {\color[HTML]{FFFFFF}\textbf{7 .. 15}}  & \textbf{16 .. 28} & \textbf{29 .. 46} & \textbf{47 .. 100} \\	\hline
\textbf{-3}                                & {\color[HTML]{FFFFFF} \textbf{1 .. 6}}   & {\color[HTML]{FFFFFF}\textbf{7 .. 15}}  & \textbf{16 .. 29} & \textbf{30 .. 48} & \textbf{49 .. 100} \\	\hline
\textbf{-2}                                & {\color[HTML]{FFFFFF} \textbf{1 .. 7}}  & {\color[HTML]{FFFFFF}\textbf{8 .. 16}}  & \textbf{17 .. 30} & \textbf{31 .. 50} & \textbf{51 .. 100} \\	\hline
\textbf{-1}                                & {\color[HTML]{FFFFFF} \textbf{1 .. 7}}  & {\color[HTML]{FFFFFF}\textbf{8 .. 16}}  & \textbf{17 .. 31} & \textbf{32 .. 52} & \textbf{53 .. 100} \\	\hline
\textbf{0}                                 & {\color[HTML]{FFFFFF} \textbf{1 .. 7}}  & {\color[HTML]{FFFFFF}\textbf{8 .. 17}}  & \textbf{18 .. 33} & \textbf{34 .. 54} & \textbf{55 .. 100} \\	\hline
\textbf{1}                                 & {\color[HTML]{FFFFFF} \textbf{1 .. 7}}  & {\color[HTML]{FFFFFF}\textbf{8 .. 18}}  & \textbf{19 .. 35} & \textbf{36 .. 57} & \textbf{58 .. 100} \\	\hline
\textbf{2}                                 & {\color[HTML]{FFFFFF} \textbf{1 .. 8}}  & {\color[HTML]{FFFFFF}\textbf{9 .. 19}}  & \textbf{20 .. 37} & \textbf{38 .. 60} & \textbf{59 .. 100} \\	\hline
\textbf{3}                                 & {\color[HTML]{FFFFFF} \textbf{1 .. 8}}  & {\color[HTML]{FFFFFF}\textbf{9 .. 20}}  & \textbf{21 .. 39} & \textbf{40 .. 63} & \textbf{64 .. 100} \\	\hline
\textbf{4}                                 & {\color[HTML]{FFFFFF} \textbf{1 .. 9}}  & {\color[HTML]{FFFFFF}\textbf{10 .. 21}} & \textbf{22 .. 41} & \textbf{41 .. 66} & \textbf{67 .. 100} \\	\hline
\textbf{5}                                 & {\color[HTML]{FFFFFF} \textbf{1 .. 9}}  & {\color[HTML]{FFFFFF}\textbf{10 .. 22}} & \textbf{23 .. 43} & \textbf{44 .. 69} & \textbf{70 .. 100} \\	\hline
\textbf{6}                                 & {\color[HTML]{FFFFFF} \textbf{1 .. 10}} & {\color[HTML]{FFFFFF}\textbf{11 .. 23}} & \textbf{24 .. 45} & \textbf{46 .. 72} & \textbf{73 .. 100} \\	\hline
\textbf{7}                                 & {\color[HTML]{FFFFFF} \textbf{1 .. 10}} & {\color[HTML]{FFFFFF}\textbf{11 .. 25}} & \textbf{26 .. 47} & \textbf{48 .. 75} & \textbf{76 .. 100} \\	\hline
\textbf{8}                                 & {\color[HTML]{FFFFFF} \textbf{1 .. 11}} & {\color[HTML]{FFFFFF}\textbf{12 .. 27}} & \textbf{28 .. 49} & \textbf{50 .. 78} & \textbf{79 .. 100} \\	\hline
\textbf{9}                                 & {\color[HTML]{FFFFFF} \textbf{1 .. 12}} & {\color[HTML]{FFFFFF}\textbf{13 .. 29}} & \textbf{30 .. 51} & \textbf{52 .. 81} & \textbf{82 .. 100} \\	\hline
\textbf{10}                                & {\color[HTML]{FFFFFF} \textbf{1 .. 13}} & {\color[HTML]{FFFFFF}\textbf{14 .. 31}} & \textbf{32 .. 53} & \textbf{54 .. 84} & \textbf{85 .. 100} \\	\hline
\textbf{11 to 15}                          & {\color[HTML]{FFFFFF} \textbf{1 .. 14}} & {\color[HTML]{FFFFFF}\textbf{15 .. 33}} & \textbf{34 .. 56} & \textbf{57 .. 88} & \textbf{89 .. 100} \\	\hline
\textbf{16 to 20}                          & {\color[HTML]{FFFFFF} \textbf{1 .. 15}} & {\color[HTML]{FFFFFF}\textbf{16 .. 35}} & \textbf{36 .. 59} & \textbf{60 .. 92} & \textbf{93 .. 100} \\	\hline
\textbf{21 to 25}                          & {\color[HTML]{FFFFFF} \textbf{1 .. 16}} & {\color[HTML]{FFFFFF}\textbf{17 .. 37}} & \textbf{38 .. 62} & \textbf{63 .. 96} & \textbf{97 .. 100} \\ 	\hline
\textbf{26 to 30}                          & {\color[HTML]{FFFFFF} \textbf{1 .. 18}} & {\color[HTML]{FFFFFF}\textbf{19 .. 40}} & \textbf{41 .. 66} & \textbf{67 .. 96} & \textbf{97 .. 100}
\end{tabular}
\end{table}


\newpage


\begin{table}[h]
\begin{tabular}{
>{\columncolor[HTML]{FFFFFF}}l 
>{\columncolor[HTML]{000000}}c 
>{\columncolor[HTML]{FE0000}}c
>{\columncolor[HTML]{F8FF00}}c 
>{\columncolor[HTML]{34FF34}}c 
>{\columncolor[HTML]{EFEFEF}}c }
{\color[HTML]{000000} \textbf{-26 to -30}} & {\color[HTML]{FFFFFF} \textbf{4}}  & {\color[HTML]{FFFFFF}\textbf{10}}  & \textbf{18} & \textbf{30} & \textbf{100} \\ 	
\textbf{-21 to -25}                        & {\color[HTML]{FFFFFF} \textbf{4}}  & {\color[HTML]{FFFFFF}\textbf{10}}  & \textbf{19} & \textbf{31} & \textbf{100} \\	
\textbf{-16 to -20}                        & {\color[HTML]{FFFFFF} \textbf{4}}  & {\color[HTML]{FFFFFF}\textbf{11}}  & \textbf{20} & \textbf{32} & \textbf{100} \\	
\textbf{-11 to -15}                        & {\color[HTML]{FFFFFF} \textbf{4}}  & {\color[HTML]{FFFFFF}\textbf{11}}  & \textbf{21} & \textbf{33} & \textbf{100} \\	
\textbf{-10}                               & {\color[HTML]{FFFFFF} \textbf{4}}  & {\color[HTML]{FFFFFF}\textbf{12}}  & \textbf{22} & \textbf{34} & \textbf{100} \\	
\textbf{-9}                                & {\color[HTML]{FFFFFF} \textbf{5}}  & {\color[HTML]{FFFFFF}\textbf{12}}  & \textbf{23} & \textbf{36} & \textbf{100} \\	
\textbf{-8}                                & {\color[HTML]{FFFFFF} \textbf{5}}  & {\color[HTML]{FFFFFF}\textbf{13}}  & \textbf{24} & \textbf{38} & \textbf{100} \\	
\textbf{-7}                                & {\color[HTML]{FFFFFF} \textbf{5}}  & {\color[HTML]{FFFFFF}\textbf{13}}  & \textbf{25} & \textbf{40} & \textbf{100} \\	
\textbf{-6}                                & {\color[HTML]{FFFFFF} \textbf{6}}  & {\color[HTML]{FFFFFF}\textbf{14}}  & \textbf{26} & \textbf{42} & \textbf{100} \\	
\textbf{-5}                                & {\color[HTML]{FFFFFF} \textbf{6}}   & {\color[HTML]{FFFFFF}\textbf{14}}  & \textbf{27} & \textbf{44} & \textbf{100} \\	
\textbf{-4}                                & {\color[HTML]{FFFFFF} \textbf{6}}   & {\color[HTML]{FFFFFF}\textbf{15}}  & \textbf{28} & \textbf{46} & \textbf{100} \\	
\textbf{-3}                                & {\color[HTML]{FFFFFF} \textbf{6}}   & {\color[HTML]{FFFFFF}\textbf{15}}  & \textbf{29} & \textbf{48} & \textbf{100} \\		
\textbf{-2}                                & {\color[HTML]{FFFFFF} \textbf{7}}  & {\color[HTML]{FFFFFF}\textbf{16}}  & \textbf{30} & \textbf{50} & \textbf{100} \\	
\textbf{-1}                                & {\color[HTML]{FFFFFF} \textbf{7}}  & {\color[HTML]{FFFFFF}\textbf{16}}  & \textbf{31} & \textbf{52} & \textbf{100} \\	
\textbf{0}                                 & {\color[HTML]{FFFFFF} \textbf{7}}  & {\color[HTML]{FFFFFF}\textbf{17}}  & \textbf{33} & \textbf{54} & \textbf{100} \\	
\textbf{1}                                 & {\color[HTML]{FFFFFF} \textbf{7}}  & {\color[HTML]{FFFFFF}\textbf{18}}  & \textbf{35} & \textbf{57} & \textbf{100} \\	
\textbf{2}                                 & {\color[HTML]{FFFFFF} \textbf{8}}  & {\color[HTML]{FFFFFF}\textbf{19}}  & \textbf{37} & \textbf{60} & \textbf{100} \\	
\textbf{3}                                 & {\color[HTML]{FFFFFF} \textbf{8}}  & {\color[HTML]{FFFFFF}\textbf{20}}  & \textbf{39} & \textbf{63} & \textbf{100} \\	
\textbf{4}                                 & {\color[HTML]{FFFFFF} \textbf{9}}  & {\color[HTML]{FFFFFF}\textbf{21}} & \textbf{41} & \textbf{66} & \textbf{100} \\	
\textbf{5}                                 & {\color[HTML]{FFFFFF} \textbf{9}}  & {\color[HTML]{FFFFFF}\textbf{22}} & \textbf{43} & \textbf{69} & \textbf{100} \\	
\textbf{6}                                 & {\color[HTML]{FFFFFF} \textbf{10}} & {\color[HTML]{FFFFFF}\textbf{23}} & \textbf{45} & \textbf{72} & \textbf{100} \\	
\textbf{7}                                 & {\color[HTML]{FFFFFF} \textbf{10}} & {\color[HTML]{FFFFFF}\textbf{25}} & \textbf{47} & \textbf{75} & \textbf{100} \\	
\textbf{8}                                 & {\color[HTML]{FFFFFF} \textbf{11}} & {\color[HTML]{FFFFFF}\textbf{27}} & \textbf{49} & \textbf{78} & \textbf{100} \\	
\textbf{9}                                 & {\color[HTML]{FFFFFF} \textbf{12}} & {\color[HTML]{FFFFFF}\textbf{29}} & \textbf{51} & \textbf{81} & \textbf{100} \\	
\textbf{10}                                & {\color[HTML]{FFFFFF} \textbf{13}} & {\color[HTML]{FFFFFF}\textbf{31}} & \textbf{53} & \textbf{84} & \textbf{100} \\	
\textbf{11 to 15}                          & {\color[HTML]{FFFFFF} \textbf{14}} & {\color[HTML]{FFFFFF}\textbf{33}} & \textbf{56} & \textbf{88} & \textbf{100} \\	
\textbf{16 to 20}                          & {\color[HTML]{FFFFFF} \textbf{15}} & {\color[HTML]{FFFFFF}\textbf{35}} & \textbf{59} & \textbf{92} & \textbf{100} \\	
\textbf{21 to 25}                          & {\color[HTML]{FFFFFF} \textbf{16}} & {\color[HTML]{FFFFFF}\textbf{37}} & \textbf{62} & \textbf{96} & \textbf{100} \\	
\textbf{26 to 30}                          & {\color[HTML]{FFFFFF} \textbf{18}} & {\color[HTML]{FFFFFF}\textbf{40}} & \textbf{66} & \textbf{96} & \textbf{100}	
\end{tabular}
\end{table}

\newpage

\begin{table}[h]
\begin{tabular}{
>{\columncolor[HTML]{FFFFFF}}l 
>{\columncolor[HTML]{000000}}c 
>{\columncolor[HTML]{FE0000}}c
>{\columncolor[HTML]{F8FF00}}c 
>{\columncolor[HTML]{34FF34}}c 
>{\columncolor[HTML]{EFEFEF}}c }
{\color[HTML]{000000} \textbf{-26 to -30}} & {\color[HTML]{FFFFFF} \textbf{4}}  & {\color[HTML]{FFFFFF}\textbf{10}}  & \textbf{18} & \textbf{30} & \textbf{100} \\ 	\hline
\textbf{-21 to -25}                        & {\color[HTML]{FFFFFF} \textbf{4}}  & {\color[HTML]{FFFFFF}\textbf{10}}  & \textbf{19} & \textbf{31} & \textbf{100} \\	\hline
\textbf{-16 to -20}                        & {\color[HTML]{FFFFFF} \textbf{4}}  & {\color[HTML]{FFFFFF}\textbf{11}}  & \textbf{20} & \textbf{32} & \textbf{100} \\	\hline
\textbf{-11 to -15}                        & {\color[HTML]{FFFFFF} \textbf{4}}  & {\color[HTML]{FFFFFF}\textbf{11}}  & \textbf{21} & \textbf{33} & \textbf{100} \\	\hline
\textbf{-10}                               & {\color[HTML]{FFFFFF} \textbf{4}}  & {\color[HTML]{FFFFFF}\textbf{12}}  & \textbf{22} & \textbf{34} & \textbf{100} \\	\hline
\textbf{-9}                                & {\color[HTML]{FFFFFF} \textbf{5}}  & {\color[HTML]{FFFFFF}\textbf{12}}  & \textbf{23} & \textbf{36} & \textbf{100} \\	\hline
\textbf{-8}                                & {\color[HTML]{FFFFFF} \textbf{5}}  & {\color[HTML]{FFFFFF}\textbf{13}}  & \textbf{24} & \textbf{38} & \textbf{100} \\	\hline
\textbf{-7}                                & {\color[HTML]{FFFFFF} \textbf{5}}  & {\color[HTML]{FFFFFF}\textbf{13}}  & \textbf{25} & \textbf{40} & \textbf{100} \\	\hline
\textbf{-6}                                & {\color[HTML]{FFFFFF} \textbf{6}}  & {\color[HTML]{FFFFFF}\textbf{14}}  & \textbf{26} & \textbf{42} & \textbf{100} \\	\hline
\textbf{-5}                                & {\color[HTML]{FFFFFF} \textbf{6}}   & {\color[HTML]{FFFFFF}\textbf{14}}  & \textbf{27} & \textbf{44} & \textbf{100} \\	\hline
\textbf{-4}                                & {\color[HTML]{FFFFFF} \textbf{6}}   & {\color[HTML]{FFFFFF}\textbf{15}}  & \textbf{28} & \textbf{46} & \textbf{100} \\	\hline
\textbf{-3}                                & {\color[HTML]{FFFFFF} \textbf{6}}   & {\color[HTML]{FFFFFF}\textbf{15}}  & \textbf{29} & \textbf{48} & \textbf{100} \\	\hline	
\textbf{-2}                                & {\color[HTML]{FFFFFF} \textbf{7}}  & {\color[HTML]{FFFFFF}\textbf{16}}  & \textbf{30} & \textbf{50} & \textbf{100} \\	\hline
\textbf{-1}                                & {\color[HTML]{FFFFFF} \textbf{7}}  & {\color[HTML]{FFFFFF}\textbf{16}}  & \textbf{31} & \textbf{52} & \textbf{100} \\	\hline
\textbf{0}                                 & {\color[HTML]{FFFFFF} \textbf{7}}  & {\color[HTML]{FFFFFF}\textbf{17}}  & \textbf{33} & \textbf{54} & \textbf{100} \\	\hline
\textbf{1}                                 & {\color[HTML]{FFFFFF} \textbf{7}}  & {\color[HTML]{FFFFFF}\textbf{18}}  & \textbf{35} & \textbf{57} & \textbf{100} \\	\hline
\textbf{2}                                 & {\color[HTML]{FFFFFF} \textbf{8}}  & {\color[HTML]{FFFFFF}\textbf{19}}  & \textbf{37} & \textbf{60} & \textbf{100} \\	\hline
\textbf{3}                                 & {\color[HTML]{FFFFFF} \textbf{8}}  & {\color[HTML]{FFFFFF}\textbf{20}}  & \textbf{39} & \textbf{63} & \textbf{100} \\	\hline
\textbf{4}                                 & {\color[HTML]{FFFFFF} \textbf{9}}  & {\color[HTML]{FFFFFF}\textbf{21}} & \textbf{41} & \textbf{66} & \textbf{100} \\	\hline
\textbf{5}                                 & {\color[HTML]{FFFFFF} \textbf{9}}  & {\color[HTML]{FFFFFF}\textbf{22}} & \textbf{43} & \textbf{69} & \textbf{100} \\	\hline
\textbf{6}                                 & {\color[HTML]{FFFFFF} \textbf{10}} & {\color[HTML]{FFFFFF}\textbf{23}} & \textbf{45} & \textbf{72} & \textbf{100} \\	\hline
\textbf{7}                                 & {\color[HTML]{FFFFFF} \textbf{10}} & {\color[HTML]{FFFFFF}\textbf{25}} & \textbf{47} & \textbf{75} & \textbf{100} \\	\hline
\textbf{8}                                 & {\color[HTML]{FFFFFF} \textbf{11}} & {\color[HTML]{FFFFFF}\textbf{27}} & \textbf{49} & \textbf{78} & \textbf{100} \\	\hline
\textbf{9}                                 & {\color[HTML]{FFFFFF} \textbf{12}} & {\color[HTML]{FFFFFF}\textbf{29}} & \textbf{51} & \textbf{81} & \textbf{100} \\	\hline
\textbf{10}                                & {\color[HTML]{FFFFFF} \textbf{13}} & {\color[HTML]{FFFFFF}\textbf{31}} & \textbf{53} & \textbf{84} & \textbf{100} \\	\hline
\textbf{11 to 15}                          & {\color[HTML]{FFFFFF} \textbf{14}} & {\color[HTML]{FFFFFF}\textbf{33}} & \textbf{56} & \textbf{88} & \textbf{100} \\	\hline
\textbf{16 to 20}                          & {\color[HTML]{FFFFFF} \textbf{15}} & {\color[HTML]{FFFFFF}\textbf{35}} & \textbf{59} & \textbf{92} & \textbf{100} \\	\hline
\textbf{21 to 25}                          & {\color[HTML]{FFFFFF} \textbf{16}} & {\color[HTML]{FFFFFF}\textbf{37}} & \textbf{62} & \textbf{96} & \textbf{100} \\	\hline
\textbf{26 to 30}                          & {\color[HTML]{FFFFFF} \textbf{18}} & {\color[HTML]{FFFFFF}\textbf{40}} & \textbf{66} & \textbf{96} & \textbf{100}	
\end{tabular}
\end{table}

\newpage

\begin{table}[h]
\begin{tabular}{lccccc}
\multicolumn{1}{c|}{} & \cellcolor[HTML]{333333}{\color[HTML]{FFFFFF} \textbf{B}} & \cellcolor[HTML]{FE0000}{\color[HTML]{FFFFFF}\textbf{R}} & \multicolumn{1}{c|}{\cellcolor[HTML]{F8FF00}\textbf{Y}} & \multicolumn{1}{c|}{\cellcolor[HTML]{34FF34}\textbf{G}} & \multicolumn{1}{c|}{\cellcolor[HTML]{C0C0C0}\textbf{F}} \\
\rowcolor[HTML]{FFFFFF} 
{\color[HTML]{000000} \textbf{-26 to -30}} & {\color[HTML]{333333} 1 .. 4}                             & 5 .. 10                            & 11 .. 18                                               & 19 .. 30                                               & 31 .. 100                                              \\
\rowcolor[HTML]{EFEFEF} 
\textbf{-21 to -25}                        & {\color[HTML]{333333} 1 .. 4}                             & 5 .. 10                            & 11 .. 19                                               & 20 .. 31                                               & 32 .. 100                                              \\
\rowcolor[HTML]{FFFFFF} 
\textbf{-16 to -20}                        & {\color[HTML]{333333} 1 .. 4}                             & 5 .. 11                            & 12 .. 20                                               & 21 .. 32                                               & 33 .. 100                                              \\
\rowcolor[HTML]{EFEFEF} 
\textbf{-11 to -15}                        & {\color[HTML]{333333} 1 .. 4}                             & 5 .. 11                            & 12 .. 21                                               & 22 .. 33                                               & 34 .. 100                                              \\
\rowcolor[HTML]{FFFFFF} 
\textbf{-10}                               & {\color[HTML]{333333} 1 .. 4}                             & 5 .. 12                            & 13 .. 22                                               & 23 .. 34                                               & 35 .. 100                                              \\
\rowcolor[HTML]{EFEFEF} 
\textbf{-9}                                & {\color[HTML]{333333} 1 .. 5}                             & 6 .. 12                            & 13 .. 23                                               & 24 .. 36                                               & 37 .. 100                                              \\
\rowcolor[HTML]{FFFFFF} 
\textbf{-8}                                & {\color[HTML]{333333} 1 .. 5}                             & 6 .. 13                            & 14 .. 24                                               & 25 .. 38                                               & 39 .. 100                                              \\
\rowcolor[HTML]{EFEFEF} 
\textbf{-7}                                & {\color[HTML]{333333} 1 .. 5}                             & 6 .. 13                            & 14 .. 25                                               & 26 .. 40                                               & 41 .. 100                                              \\
\rowcolor[HTML]{FFFFFF} 
\textbf{-6}                                & {\color[HTML]{333333} 1 .. 6}                             & 7 .. 14                            & 15 .. 26                                               & 27 .. 42                                               & 43 .. 100                                              \\
\rowcolor[HTML]{EFEFEF} 
\textbf{-5}                                & {\color[HTML]{333333} 1 .. 6}                             & 7 .. 14                            & 15 .. 27                                               & 28 .. 44                                               & 45 .. 100                                              \\
\rowcolor[HTML]{FFFFFF} 
\textbf{-4}                                & {\color[HTML]{333333} 1 .. 6}                             & 7 .. 15                            & 16 .. 28                                               & 29 .. 46                                               & 47 .. 100                                              \\
\rowcolor[HTML]{EFEFEF} 
\textbf{-3}                                & {\color[HTML]{333333} 1 .. 6}                             & 7 .. 15                            & 16 .. 29                                               & 30 .. 48                                               & 49 .. 100                                              \\
\rowcolor[HTML]{FFFFFF} 
\textbf{-2}                                & {\color[HTML]{333333} 1 .. 7}                             & 8 .. 16                            & 17 .. 30                                               & 31 .. 50                                               & 51 .. 100                                              \\
\rowcolor[HTML]{EFEFEF} 
\textbf{-1}                                & {\color[HTML]{333333} 1 .. 7}                             & 8 .. 16                            & 17 .. 31                                               & 32 .. 52                                               & 53 .. 100                                              \\
\rowcolor[HTML]{FFFFFF} 
\textbf{0}                                 & {\color[HTML]{333333} 1 .. 7}                             & 8 .. 17                            & 18 .. 33                                               & 34 .. 54                                               & 55 .. 100                                              \\
\rowcolor[HTML]{EFEFEF} 
\textbf{1}                                 & {\color[HTML]{333333} 1 .. 7}                             & 8 .. 18                            & 19 .. 35                                               & 36 .. 57                                               & 58 .. 100                                              \\
\rowcolor[HTML]{FFFFFF} 
\textbf{2}                                 & {\color[HTML]{333333} 1 .. 8}                             & 9 .. 19                            & 20 .. 37                                               & 38 .. 60                                               & 59 .. 100                                              \\
\rowcolor[HTML]{EFEFEF} 
\textbf{3}                                 & {\color[HTML]{333333} 1 .. 8}                             & 9 .. 20                            & 21 .. 39                                               & 40 .. 63                                               & 64 .. 100                                              \\
\rowcolor[HTML]{FFFFFF} 
\textbf{4}                                 & {\color[HTML]{333333} 1 .. 9}                             & 10 .. 21                           & 22 .. 41                                               & 41 .. 66                                               & 67 .. 100                                              \\
\rowcolor[HTML]{EFEFEF} 
\textbf{5}                                 & {\color[HTML]{333333} 1 .. 9}                             & 10 .. 22                           & 23 .. 43                                               & 44 .. 69                                               & 70 .. 100                                              \\
\rowcolor[HTML]{FFFFFF} 
\textbf{6}                                 & {\color[HTML]{333333} 1 .. 10}                            & 11 .. 23                           & 24 .. 45                                               & 46 .. 72                                               & 73 .. 100                                              \\
\rowcolor[HTML]{EFEFEF} 
\textbf{7}                                 & {\color[HTML]{333333} 1 .. 10}                            & 11 .. 25                           & 26 .. 47                                               & 48 .. 75                                               & 76 .. 100                                              \\
\rowcolor[HTML]{FFFFFF} 
\textbf{8}                                 & {\color[HTML]{333333} 1 .. 11}                            & 12 .. 27                           & 28 .. 49                                               & 50 .. 78                                               & 79 .. 100                                              \\
\rowcolor[HTML]{EFEFEF} 
\textbf{9}                                 & {\color[HTML]{333333} 1 .. 12}                            & 13 .. 29                           & 30 .. 51                                               & 52 .. 81                                               & 82 .. 100                                              \\
\rowcolor[HTML]{FFFFFF} 
\textbf{10}                                & {\color[HTML]{333333} 1 .. 13}                            & 14 .. 31                           & 32 .. 53                                               & 54 .. 84                                               & 85 .. 100                                              \\
\rowcolor[HTML]{EFEFEF} 
\textbf{11 to 15}                          & {\color[HTML]{333333} 1 .. 14}                            & 15 .. 33                           & 34 .. 56                                               & 57 .. 88                                               & 89 .. 100                                              \\
\rowcolor[HTML]{FFFFFF} 
\textbf{16 to 20}                          & {\color[HTML]{333333} 1 .. 15}                            & 16 .. 35                           & 36 .. 59                                               & 60 .. 92                                               & 93 .. 100                                              \\
\rowcolor[HTML]{EFEFEF} 
\textbf{21 to 25}                          & {\color[HTML]{333333} 1 .. 16}                            & 17 .. 37                           & 38 .. 62                                               & 63 .. 96                                               & 97 .. 100                                              \\
\rowcolor[HTML]{FFFFFF} 
\textbf{26 to 30}                          & {\color[HTML]{333333} 1 .. 18}                            & 19 .. 40                           & 41 .. 66                                               & 67 .. 96                                               & 97 .. 100                                             
\end{tabular}
\end{table}

\newpage


\begin{table}[h]
\begin{tabular}{l|c|c|c|c|c|}
\cline{2-6}
\multicolumn{1}{c|}{} & \cellcolor[HTML]{333333}{\color[HTML]{FFFFFF} \textbf{B}} & \cellcolor[HTML]{FE0000}{\color[HTML]{FFFFFF}\textbf{R}} & \multicolumn{1}{c|}{\cellcolor[HTML]{F8FF00}\textbf{Y}} & \multicolumn{1}{c|}{\cellcolor[HTML]{34FF34}\textbf{G}} & \multicolumn{1}{c|}{\cellcolor[HTML]{C0C0C0}\textbf{F}} \\ \hline
\rowcolor[HTML]{FFFFFF} 
\multicolumn{1}{|l|}{\cellcolor[HTML]{FFFFFF}{\color[HTML]{000000} \textbf{-26 to -30}}} & {\color[HTML]{333333} 1 .. 4}                             & 5 .. 10                            & 11 .. 18                                                & 19 .. 30                                                & 31 .. 100                                               \\ \hline
\rowcolor[HTML]{EFEFEF} 
\multicolumn{1}{|l|}{\cellcolor[HTML]{EFEFEF}\textbf{-21 to -25}}                        & {\color[HTML]{333333} 1 .. 4}                             & 5 .. 10                            & 11 .. 19                                                & 20 .. 31                                                & 32 .. 100                                               \\ \hline
\rowcolor[HTML]{FFFFFF} 
\multicolumn{1}{|l|}{\cellcolor[HTML]{FFFFFF}\textbf{-16 to -20}}                        & {\color[HTML]{333333} 1 .. 4}                             & 5 .. 11                            & 12 .. 20                                                & 21 .. 32                                                & 33 .. 100                                               \\ \hline
\rowcolor[HTML]{EFEFEF} 
\multicolumn{1}{|l|}{\cellcolor[HTML]{EFEFEF}\textbf{-11 to -15}}                        & {\color[HTML]{333333} 1 .. 4}                             & 5 .. 11                            & 12 .. 21                                                & 22 .. 33                                                & 34 .. 100                                               \\ \hline
\rowcolor[HTML]{FFFFFF} 
\multicolumn{1}{|l|}{\cellcolor[HTML]{FFFFFF}\textbf{-10}}                               & {\color[HTML]{333333} 1 .. 4}                             & 5 .. 12                            & 13 .. 22                                                & 23 .. 34                                                & 35 .. 100                                               \\ \hline
\rowcolor[HTML]{EFEFEF} 
\multicolumn{1}{|l|}{\cellcolor[HTML]{EFEFEF}\textbf{-9}}                                & {\color[HTML]{333333} 1 .. 5}                             & 6 .. 12                            & 13 .. 23                                                & 24 .. 36                                                & 37 .. 100                                               \\ \hline
\rowcolor[HTML]{FFFFFF} 
\multicolumn{1}{|l|}{\cellcolor[HTML]{FFFFFF}\textbf{-8}}                                & {\color[HTML]{333333} 1 .. 5}                             & 6 .. 13                            & 14 .. 24                                                & 25 .. 38                                                & 39 .. 100                                               \\ \hline
\rowcolor[HTML]{EFEFEF} 
\multicolumn{1}{|l|}{\cellcolor[HTML]{EFEFEF}\textbf{-7}}                                & {\color[HTML]{333333} 1 .. 5}                             & 6 .. 13                            & 14 .. 25                                                & 26 .. 40                                                & 41 .. 100                                               \\ \hline
\rowcolor[HTML]{FFFFFF} 
\multicolumn{1}{|l|}{\cellcolor[HTML]{FFFFFF}\textbf{-6}}                                & {\color[HTML]{333333} 1 .. 6}                             & 7 .. 14                            & 15 .. 26                                                & 27 .. 42                                                & 43 .. 100                                               \\ \hline
\rowcolor[HTML]{EFEFEF} 
\multicolumn{1}{|l|}{\cellcolor[HTML]{EFEFEF}\textbf{-5}}                                & {\color[HTML]{333333} 1 .. 6}                             & 7 .. 14                            & 15 .. 27                                                & 28 .. 44                                                & 45 .. 100                                               \\ \hline
\rowcolor[HTML]{FFFFFF} 
\multicolumn{1}{|l|}{\cellcolor[HTML]{FFFFFF}\textbf{-4}}                                & {\color[HTML]{333333} 1 .. 6}                             & 7 .. 15                            & 16 .. 28                                                & 29 .. 46                                                & 47 .. 100                                               \\ \hline
\rowcolor[HTML]{EFEFEF} 
\multicolumn{1}{|l|}{\cellcolor[HTML]{EFEFEF}\textbf{-3}}                                & {\color[HTML]{333333} 1 .. 6}                             & 7 .. 15                            & 16 .. 29                                                & 30 .. 48                                                & 49 .. 100                                               \\ \hline
\rowcolor[HTML]{FFFFFF} 
\multicolumn{1}{|l|}{\cellcolor[HTML]{FFFFFF}\textbf{-2}}                                & {\color[HTML]{333333} 1 .. 7}                             & 8 .. 16                            & 17 .. 30                                                & 31 .. 50                                                & 51 .. 100                                               \\ \hline
\rowcolor[HTML]{EFEFEF} 
\multicolumn{1}{|l|}{\cellcolor[HTML]{EFEFEF}\textbf{-1}}                                & {\color[HTML]{333333} 1 .. 7}                             & 8 .. 16                            & 17 .. 31                                                & 32 .. 52                                                & 53 .. 100                                               \\ \hline
\rowcolor[HTML]{FFFFFF} 
\multicolumn{1}{|l|}{\cellcolor[HTML]{FFFFFF}\textbf{0}}                                 & {\color[HTML]{333333} 1 .. 7}                             & 8 .. 17                            & 18 .. 33                                                & 34 .. 54                                                & 55 .. 100                                               \\ \hline
\rowcolor[HTML]{EFEFEF} 
\multicolumn{1}{|l|}{\cellcolor[HTML]{EFEFEF}\textbf{1}}                                 & {\color[HTML]{333333} 1 .. 7}                             & 8 .. 18                            & 19 .. 35                                                & 36 .. 57                                                & 58 .. 100                                               \\ \hline
\rowcolor[HTML]{FFFFFF} 
\multicolumn{1}{|l|}{\cellcolor[HTML]{FFFFFF}\textbf{2}}                                 & {\color[HTML]{333333} 1 .. 8}                             & 9 .. 19                            & 20 .. 37                                                & 38 .. 60                                                & 59 .. 100                                               \\ \hline
\rowcolor[HTML]{EFEFEF} 
\multicolumn{1}{|l|}{\cellcolor[HTML]{EFEFEF}\textbf{3}}                                 & {\color[HTML]{333333} 1 .. 8}                             & 9 .. 20                            & 21 .. 39                                                & 40 .. 63                                                & 64 .. 100                                               \\ \hline
\rowcolor[HTML]{FFFFFF} 
\multicolumn{1}{|l|}{\cellcolor[HTML]{FFFFFF}\textbf{4}}                                 & {\color[HTML]{333333} 1 .. 9}                             & 10 .. 21                           & 22 .. 41                                                & 41 .. 66                                                & 67 .. 100                                               \\ \hline
\rowcolor[HTML]{EFEFEF} 
\multicolumn{1}{|l|}{\cellcolor[HTML]{EFEFEF}\textbf{5}}                                 & {\color[HTML]{333333} 1 .. 9}                             & 10 .. 22                           & 23 .. 43                                                & 44 .. 69                                                & 70 .. 100                                               \\ \hline
\rowcolor[HTML]{FFFFFF} 
\multicolumn{1}{|l|}{\cellcolor[HTML]{FFFFFF}\textbf{6}}                                 & {\color[HTML]{333333} 1 .. 10}                            & 11 .. 23                           & 24 .. 45                                                & 46 .. 72                                                & 73 .. 100                                               \\ \hline
\rowcolor[HTML]{EFEFEF} 
\multicolumn{1}{|l|}{\cellcolor[HTML]{EFEFEF}\textbf{7}}                                 & {\color[HTML]{333333} 1 .. 10}                            & 11 .. 25                           & 26 .. 47                                                & 48 .. 75                                                & 76 .. 100                                               \\ \hline
\rowcolor[HTML]{FFFFFF} 
\multicolumn{1}{|l|}{\cellcolor[HTML]{FFFFFF}\textbf{8}}                                 & {\color[HTML]{333333} 1 .. 11}                            & 12 .. 27                           & 28 .. 49                                                & 50 .. 78                                                & 79 .. 100                                               \\ \hline
\rowcolor[HTML]{EFEFEF} 
\multicolumn{1}{|l|}{\cellcolor[HTML]{EFEFEF}\textbf{9}}                                 & {\color[HTML]{333333} 1 .. 12}                            & 13 .. 29                           & 30 .. 51                                                & 52 .. 81                                                & 82 .. 100                                               \\ \hline
\rowcolor[HTML]{FFFFFF} 
\multicolumn{1}{|l|}{\cellcolor[HTML]{FFFFFF}\textbf{10}}                                & {\color[HTML]{333333} 1 .. 13}                            & 14 .. 31                           & 32 .. 53                                                & 54 .. 84                                                & 85 .. 100                                               \\ \hline
\rowcolor[HTML]{EFEFEF} 
\multicolumn{1}{|l|}{\cellcolor[HTML]{EFEFEF}\textbf{11 to 15}}                          & {\color[HTML]{333333} 1 .. 14}                            & 15 .. 33                           & 34 .. 56                                                & 57 .. 88                                                & 89 .. 100                                               \\ \hline
\rowcolor[HTML]{FFFFFF} 
\multicolumn{1}{|l|}{\cellcolor[HTML]{FFFFFF}\textbf{16 to 20}}                          & {\color[HTML]{333333} 1 .. 15}                            & 16 .. 35                           & 36 .. 59                                                & 60 .. 92                                                & 93 .. 100                                               \\ \hline
\rowcolor[HTML]{EFEFEF} 
\multicolumn{1}{|l|}{\cellcolor[HTML]{EFEFEF}\textbf{21 to 25}}                          & {\color[HTML]{333333} 1 .. 16}                            & 17 .. 37                           & 38 .. 62                                                & 63 .. 96                                                & 97 .. 100                                               \\ \hline
\rowcolor[HTML]{FFFFFF} 
\multicolumn{1}{|l|}{\cellcolor[HTML]{FFFFFF}\textbf{26 to 30}}                          & {\color[HTML]{333333} 1 .. 18}                            & 19 .. 40                           & 41 .. 66                                                & 67 .. 96                                                & 97 .. 100                                               \\ \hline
\end{tabular}
\end{table}

\newpage

\begin{table}[h]
\begin{tabular}{lccccc}
\multicolumn{1}{c|}{} & \cellcolor[HTML]{333333}{\color[HTML]{FFFFFF} \textbf{B}} & \cellcolor[HTML]{FE0000}{\color[HTML]{FFFFFF}\textbf{R}} & \multicolumn{1}{c|}{\cellcolor[HTML]{F8FF00}\textbf{Y}} & \multicolumn{1}{c|}{\cellcolor[HTML]{34FF34}\textbf{G}} & \multicolumn{1}{c|}{\cellcolor[HTML]{C0C0C0}\textbf{F}} \\ 
\rowcolor[HTML]{FFFFFF} 
\multicolumn{1}{l}{\cellcolor[HTML]{FFFFFF}{\color[HTML]{000000} \textbf{-26 to -30}}} & {\color[HTML]{333333}4}	&10	& 18	& 30	& 100	\\ 
\rowcolor[HTML]{EFEFEF} 
\multicolumn{1}{l}{\cellcolor[HTML]{EFEFEF}\textbf{-21 to -25}}   & {\color[HTML]{333333}4}  	&10		& 19		& 31		& 100     \\ 
\rowcolor[HTML]{FFFFFF} 
\multicolumn{1}{l}{\cellcolor[HTML]{FFFFFF}\textbf{-16 to -20}}  	& {\color[HTML]{333333}4}  	&11       & 20      & 32      & 100     \\ 
\rowcolor[HTML]{EFEFEF} 
\multicolumn{1}{l}{\cellcolor[HTML]{EFEFEF}\textbf{-11 to -15}}   & {\color[HTML]{333333}4} 	&11       & 21      & 33      & 100     \\ 
\rowcolor[HTML]{FFFFFF} 
\multicolumn{1}{l}{\cellcolor[HTML]{FFFFFF}\textbf{-10}}    	    & {\color[HTML]{333333}4} 	&12       & 22      & 34      & 100     \\ 
\rowcolor[HTML]{EFEFEF} 
\multicolumn{1}{l}{\cellcolor[HTML]{EFEFEF}\textbf{-9}} 			& {\color[HTML]{333333}5}     &12       & 23      & 36      & 100     \\ 
\rowcolor[HTML]{FFFFFF} 
\multicolumn{1}{l}{\cellcolor[HTML]{FFFFFF}\textbf{-8}}			& {\color[HTML]{333333}5}     &13       & 24      & 38      & 100     \\ 
\rowcolor[HTML]{EFEFEF} 
\multicolumn{1}{l}{\cellcolor[HTML]{EFEFEF}\textbf{-7}}			& {\color[HTML]{333333}5}     &13       & 25      & 40      & 100     \\ 
\rowcolor[HTML]{FFFFFF} 
\multicolumn{1}{l}{\cellcolor[HTML]{FFFFFF}\textbf{-6}}			& {\color[HTML]{333333}6}     &14       & 26      & 42      & 100     \\ 
\rowcolor[HTML]{EFEFEF} 
\multicolumn{1}{l}{\cellcolor[HTML]{EFEFEF}\textbf{-5}}			& {\color[HTML]{333333}6}     &14       & 27      & 44      & 100     \\ 
\rowcolor[HTML]{FFFFFF} 
\multicolumn{1}{l}{\cellcolor[HTML]{FFFFFF}\textbf{-4}}           & {\color[HTML]{333333}6}     &15       & 28      & 46      & 100     \\ 
\rowcolor[HTML]{EFEFEF} 
\multicolumn{1}{l}{\cellcolor[HTML]{EFEFEF}\textbf{-3}}           & {\color[HTML]{333333}6}     &15       & 29      & 48      & 100     \\ 
\rowcolor[HTML]{FFFFFF} 
\multicolumn{1}{l}{\cellcolor[HTML]{FFFFFF}\textbf{-2}}           & {\color[HTML]{333333}7}     &16       & 30      & 50      & 100     \\ 
\rowcolor[HTML]{EFEFEF} 
\multicolumn{1}{l}{\cellcolor[HTML]{EFEFEF}\textbf{-1}}           & {\color[HTML]{333333}7}     &16       & 31      & 52      & 100     \\ 
\rowcolor[HTML]{FFFFFF} 
\multicolumn{1}{l}{\cellcolor[HTML]{FFFFFF}\textbf{0}}            & {\color[HTML]{333333}7}     &17       & 33      & 54      & 100     \\ 
\rowcolor[HTML]{EFEFEF} 
\multicolumn{1}{l}{\cellcolor[HTML]{EFEFEF}\textbf{1}}            & {\color[HTML]{333333}7}     &18       & 35      & 57      & 100     \\ 
\rowcolor[HTML]{FFFFFF} 
\multicolumn{1}{l}{\cellcolor[HTML]{FFFFFF}\textbf{2}}            & {\color[HTML]{333333}8}     &19       & 37      & 60      & 100     \\ 
\rowcolor[HTML]{EFEFEF} 
\multicolumn{1}{l}{\cellcolor[HTML]{EFEFEF}\textbf{3}}            & {\color[HTML]{333333}8}     &20       & 39      & 63      & 100     \\ 
\rowcolor[HTML]{FFFFFF} 
\multicolumn{1}{l}{\cellcolor[HTML]{FFFFFF}\textbf{4}}            & {\color[HTML]{333333}9}     & 21      & 41      & 66      & 100     \\ 
\rowcolor[HTML]{EFEFEF} 
\multicolumn{1}{l}{\cellcolor[HTML]{EFEFEF}\textbf{5}}            & {\color[HTML]{333333}9}     & 22      & 43      & 69      & 100     \\ 
\rowcolor[HTML]{FFFFFF} 
\multicolumn{1}{l}{\cellcolor[HTML]{FFFFFF}\textbf{6}}            & {\color[HTML]{333333}10}    & 23      & 45      & 72      & 100     \\ 
\rowcolor[HTML]{EFEFEF} 
\multicolumn{1}{l}{\cellcolor[HTML]{EFEFEF}\textbf{7}}            & {\color[HTML]{333333}10}    & 25      & 47      & 75      & 100     \\ 
\rowcolor[HTML]{FFFFFF} 
\multicolumn{1}{l}{\cellcolor[HTML]{FFFFFF}\textbf{8}}            & {\color[HTML]{333333}11}    & 27      & 49      & 78      & 100     \\ 
\rowcolor[HTML]{EFEFEF} 
\multicolumn{1}{l}{\cellcolor[HTML]{EFEFEF}\textbf{9}}            & {\color[HTML]{333333}12}    & 29      & 51      & 81      & 100     \\ 
\rowcolor[HTML]{FFFFFF} 
\multicolumn{1}{l}{\cellcolor[HTML]{FFFFFF}\textbf{10}}           & {\color[HTML]{333333}13}    & 31      & 53      & 84      & 100     \\ 
\rowcolor[HTML]{EFEFEF} 
\multicolumn{1}{l}{\cellcolor[HTML]{EFEFEF}\textbf{11 to 15}}     & {\color[HTML]{333333}14}    & 33      & 56      & 88      & 100     \\ 
\rowcolor[HTML]{FFFFFF} 
\multicolumn{1}{l}{\cellcolor[HTML]{FFFFFF}\textbf{16 to 20}}     & {\color[HTML]{333333}15}    & 35      & 59      & 92      & 100     \\ 
\rowcolor[HTML]{EFEFEF} 
\multicolumn{1}{l}{\cellcolor[HTML]{EFEFEF}\textbf{21 to 25}}     & {\color[HTML]{333333}16}    & 37      & 62      & 96      & 100     \\ 
\rowcolor[HTML]{FFFFFF} 
\multicolumn{1}{l}{\cellcolor[HTML]{FFFFFF}\textbf{26 to 30}}     & {\color[HTML]{333333}18}    & 40      & 66      & 96      & 100     \\ 
\end{tabular}
\end{table}

\newpage

\begin{table}[h]
\begin{tabular}{l|c|c|c|c|c|}
\cline{2-6}
\multicolumn{1}{c|}{} & \cellcolor[HTML]{333333}{\color[HTML]{FFFFFF} \textbf{B}} & \cellcolor[HTML]{FE0000}{\color[HTML]{FFFFFF}\textbf{R}} & \multicolumn{1}{c|}{\cellcolor[HTML]{F8FF00}\textbf{Y}} & \multicolumn{1}{c|}{\cellcolor[HTML]{34FF34}\textbf{G}} & \multicolumn{1}{c|}{\cellcolor[HTML]{C0C0C0}\textbf{F}} \\ \hline
\rowcolor[HTML]{FFFFFF} 
\multicolumn{1}{|l|}{\cellcolor[HTML]{FFFFFF}{\color[HTML]{000000} \textbf{-26 to -30}}} & {\color[HTML]{333333}4}	&10	& 18	& 30	& 100	\\ \hline
\rowcolor[HTML]{EFEFEF} 
\multicolumn{1}{|l|}{\cellcolor[HTML]{EFEFEF}\textbf{-21 to -25}}   & {\color[HTML]{333333}4}  	&10		& 19		& 31		& 100     \\ \hline
\rowcolor[HTML]{FFFFFF} 
\multicolumn{1}{|l|}{\cellcolor[HTML]{FFFFFF}\textbf{-16 to -20}}  	& {\color[HTML]{333333}4}  	&11       & 20      & 32      & 100     \\ \hline
\rowcolor[HTML]{EFEFEF} 
\multicolumn{1}{|l|}{\cellcolor[HTML]{EFEFEF}\textbf{-11 to -15}}   & {\color[HTML]{333333}4} 	&11       & 21      & 33      & 100     \\ \hline
\rowcolor[HTML]{FFFFFF} 
\multicolumn{1}{|l|}{\cellcolor[HTML]{FFFFFF}\textbf{-10}}    	    & {\color[HTML]{333333}4} 	&12       & 22      & 34      & 100     \\ \hline
\rowcolor[HTML]{EFEFEF} 
\multicolumn{1}{|l|}{\cellcolor[HTML]{EFEFEF}\textbf{-9}} 			& {\color[HTML]{333333}5}     &12       & 23      & 36      & 100     \\ \hline
\rowcolor[HTML]{FFFFFF} 
\multicolumn{1}{|l|}{\cellcolor[HTML]{FFFFFF}\textbf{-8}}			& {\color[HTML]{333333}5}     &13       & 24      & 38      & 100     \\ \hline
\rowcolor[HTML]{EFEFEF} 
\multicolumn{1}{|l|}{\cellcolor[HTML]{EFEFEF}\textbf{-7}}			& {\color[HTML]{333333}5}     &13       & 25      & 40      & 100     \\ \hline
\rowcolor[HTML]{FFFFFF} 
\multicolumn{1}{|l|}{\cellcolor[HTML]{FFFFFF}\textbf{-6}}			& {\color[HTML]{333333}6}     &14       & 26      & 42      & 100     \\ \hline
\rowcolor[HTML]{EFEFEF} 
\multicolumn{1}{|l|}{\cellcolor[HTML]{EFEFEF}\textbf{-5}}			& {\color[HTML]{333333}6}     &14       & 27      & 44      & 100     \\ \hline
\rowcolor[HTML]{FFFFFF} 
\multicolumn{1}{|l|}{\cellcolor[HTML]{FFFFFF}\textbf{-4}}           & {\color[HTML]{333333}6}     &15       & 28      & 46      & 100     \\ \hline
\rowcolor[HTML]{EFEFEF} 
\multicolumn{1}{|l|}{\cellcolor[HTML]{EFEFEF}\textbf{-3}}           & {\color[HTML]{333333}6}     &15       & 29      & 48      & 100     \\ \hline
\rowcolor[HTML]{FFFFFF} 
\multicolumn{1}{|l|}{\cellcolor[HTML]{FFFFFF}\textbf{-2}}           & {\color[HTML]{333333}7}     &16       & 30      & 50      & 100     \\ \hline
\rowcolor[HTML]{EFEFEF} 
\multicolumn{1}{|l|}{\cellcolor[HTML]{EFEFEF}\textbf{-1}}           & {\color[HTML]{333333}7}     &16       & 31      & 52      & 100     \\ \hline
\rowcolor[HTML]{FFFFFF} 
\multicolumn{1}{|l|}{\cellcolor[HTML]{FFFFFF}\textbf{0}}            & {\color[HTML]{333333}7}     &17       & 33      & 54      & 100     \\ \hline
\rowcolor[HTML]{EFEFEF} 
\multicolumn{1}{|l|}{\cellcolor[HTML]{EFEFEF}\textbf{1}}            & {\color[HTML]{333333}7}     &18       & 35      & 57      & 100     \\ \hline
\rowcolor[HTML]{FFFFFF} 
\multicolumn{1}{|l|}{\cellcolor[HTML]{FFFFFF}\textbf{2}}            & {\color[HTML]{333333}8}     &19       & 37      & 60      & 100     \\ \hline
\rowcolor[HTML]{EFEFEF} 
\multicolumn{1}{|l|}{\cellcolor[HTML]{EFEFEF}\textbf{3}}            & {\color[HTML]{333333}8}     &20       & 39      & 63      & 100     \\ \hline
\rowcolor[HTML]{FFFFFF} 
\multicolumn{1}{|l|}{\cellcolor[HTML]{FFFFFF}\textbf{4}}            & {\color[HTML]{333333}9}     & 21      & 41      & 66      & 100     \\ \hline
\rowcolor[HTML]{EFEFEF} 
\multicolumn{1}{|l|}{\cellcolor[HTML]{EFEFEF}\textbf{5}}            & {\color[HTML]{333333}9}     & 22      & 43      & 69      & 100     \\ \hline
\rowcolor[HTML]{FFFFFF} 
\multicolumn{1}{|l|}{\cellcolor[HTML]{FFFFFF}\textbf{6}}            & {\color[HTML]{333333}10}    & 23      & 45      & 72      & 100     \\ \hline
\rowcolor[HTML]{EFEFEF} 
\multicolumn{1}{|l|}{\cellcolor[HTML]{EFEFEF}\textbf{7}}            & {\color[HTML]{333333}10}    & 25      & 47      & 75      & 100     \\ \hline
\rowcolor[HTML]{FFFFFF} 
\multicolumn{1}{|l|}{\cellcolor[HTML]{FFFFFF}\textbf{8}}            & {\color[HTML]{333333}11}    & 27      & 49      & 78      & 100     \\ \hline
\rowcolor[HTML]{EFEFEF} 
\multicolumn{1}{|l|}{\cellcolor[HTML]{EFEFEF}\textbf{9}}            & {\color[HTML]{333333}12}    & 29      & 51      & 81      & 100     \\ \hline
\rowcolor[HTML]{FFFFFF} 
\multicolumn{1}{|l|}{\cellcolor[HTML]{FFFFFF}\textbf{10}}           & {\color[HTML]{333333}13}    & 31      & 53      & 84      & 100     \\ \hline
\rowcolor[HTML]{EFEFEF} 
\multicolumn{1}{|l|}{\cellcolor[HTML]{EFEFEF}\textbf{11 to 15}}     & {\color[HTML]{333333}14}    & 33      & 56      & 88      & 100     \\ \hline
\rowcolor[HTML]{FFFFFF} 
\multicolumn{1}{|l|}{\cellcolor[HTML]{FFFFFF}\textbf{16 to 20}}     & {\color[HTML]{333333}15}    & 35      & 59      & 92      & 100     \\ \hline
\rowcolor[HTML]{EFEFEF} 
\multicolumn{1}{|l|}{\cellcolor[HTML]{EFEFEF}\textbf{21 to 25}}     & {\color[HTML]{333333}16}    & 37      & 62      & 96      & 100     \\ \hline
\rowcolor[HTML]{FFFFFF} 
\multicolumn{1}{|l|}{\cellcolor[HTML]{FFFFFF}\textbf{26 to 30}}     & {\color[HTML]{333333}18}    & 40      & 66      & 96      & 100     \\ \hline
\end{tabular}
\end{table}


\end{document}